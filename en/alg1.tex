\documentclass[12pt]{article}

\usepackage{theorem,amsmath,amssymb}



\addtolength{\topmargin}{-23mm}
\addtolength{\textheight}{60mm}
\addtolength{\oddsidemargin}{-20mm}
\addtolength{\textwidth}{40mm}

\def\eqref#1{(\ref{#1})}
\newcommand{\goth}{\mathfrak}
\newcommand{\arrow}{{\:\longrightarrow\:}}
\def\1{\sqrt{-1}\:}
\newcommand{\restrict}[1]{{\left|_{{\phantom{|}\!\!}_{#1}}\right.}}

\renewcommand{\bar}{\overline}
\renewcommand{\phi}{\varphi}
\renewcommand{\epsilon}{\varepsilon}
\renewcommand{\geq}{\geqslant}
\renewcommand{\leq}{\leqslant}

\def\rad{\operatorname{\sf rad}}
\def\tr{\operatorname{\sf tr}}
\def\rk{\operatorname{\sf rk}}
\def\Alt{\operatorname{\sf Alt}}
\def\Sym{\operatorname{\sf Sym}}
\def\Id{\operatorname{\sf Id}}
\def\Hom{\operatorname{Hom}}
\def\Map{\operatorname{Map}}
\def\Gal{\operatorname{Gal}}
\def\Aut{\operatorname{Aut}}
\newcommand{\End}{\operatorname{End}}
\newcommand{\Mat}{\operatorname{Mat}}

\newcommand{\coker}{\operatorname{Coker}}

\def\chpoly{\operatorname{\sf Chpoly}}
\def\minpoly{\operatorname{\sf Minpoly}}

\def\cchar{\operatorname{\sf char}}

\def\Z{{\mathbb Z}}
\def\R{{\mathbb R}}
\def\C{{\mathbb C}}
\def\Q{{\mathbb Q}}
\def\N{{\mathbb N}}
\def\F{{\mathbb F}}

\def\Re{\operatorname{Re}}
\def\Im{\operatorname{Im}}

\makeatletter
\theoremstyle{definition}

\newtheorem{zadacha}{������}[section]
\newtheorem{opredelenie}{�����������}[section]
\newtheorem*{ukazanie}{��������}%[section]
\newtheorem*{zamechanie}{���������}%[section]

%\renewcommand{\labelenumi}{\ralph{enumi}.}
\renewcommand{\labelenumi}{\alph{enumi}.}
\newcommand{\subs}[1]{{\bigskip\centerline{\bf\large #1}\bigskip}}
\newcommand{\sttr}{{\bf(*)}}
\newcommand{\shrk}{{\bf(!)}}
\newcommand{\doublesttr}{{\bf(**)}}

\newcommand{\listok}[2]{%
\setcounter{page}{1}
\renewcommand{\@oddhead}{\hfil #2 \hfil}
\renewcommand{\@evenhead}{\hfil #2 \hfil}
\section*{#2}
\refstepcounter{section}
\setcounter{section}{#1}
}

\@addtoreset{equation}{section}
\renewcommand{\theequation}{\thesection.\arabic{equation}}

\let\oldllim=\lim
\def\lim{\oldllim\limits}
\makeatother


\begin{document}

%%%%%%%%%%%%%%%%%%%%%%%%%%%%%%%%%%%%%%%%%%%%%%%%

\listok{1}{ALGEBRA 1: groups, rings and fields}

\subs{Groups}

The set of pairs $(a,b)$, where $a$ is an element of $A$ and $b$ is an
element of $B$ is called a {\bf product} of $A$ and $B$ and is denoted
by $A \times B$.  A mapping $f:A \to B$ from a set $A$ to a set $B$
is called an {\bf injective mapping} or {\bf injection} or {\bf
  one-to-one mapping} (these are synonyms), if it maps different
elements of the set $A$ to different elements of the set $B$. A
mapping is called a {\bf surjective mapping} or a {\bf surjection} or
an {\bf onto mapping} if for every element $x$ of set $B$ there exists
at least one element of $A$ that is mapped to $x$. A mapping is called
{\bf bijective mapping} or {\bf bijection} or a {\bf one-to-one
  mapping} if it is surjective and injective.

Let $A$ be some set, either finite or infinite. Let $S(A)$ denote
the set of all bijective mappings from $A$ to itself. If $f, g$ are
two such mappings then they can be ``multiplied'' using composition
$f\circ g$:
$$
f \circ g (a) = f(g(a)).
$$ A set $S(A)$ endowed with this operation is called ``permutation
group (or substitution group)'' or more precisely ``the group of
permutations of $A$''. Identity permutation is denoted by $1_A$
(or $\Id_A$ as well).

$S(A)$ is also called a {\bf symmetric group}. If $A$ is a finite set
of $n$ elements then $S(A)$ is denoted by $S_n$.

Permutations can be represented by tables; for example, a permutation
$1 \mapsto 3$, $2 \mapsto 4$, $3 \mapsto 1$, $4 \mapsto 2$ of
$1,2,3,4$ can be written down as $\sigma =
\left(\begin{array}{cccc}1&2&3&4\\3&4&1&2\end{array}\right)$. The
numbers are written in the ascending order in the upper line, their
images are written in the lower line.

\begin{zadacha}
  Compute the composition
$$
\left(\begin{array}{ccccc}1&2&3&4&5\\4&5&2&1&3\end{array}\right) \circ 
\left(\begin{array}{ccccc}1&2&3&4&5\\3&5&4&1&2\end{array}\right).
$$
\end{zadacha}

\begin{zadacha}
\begin{enumerate}
\item How many permutations of $1,2,\dots,5$ are there? How many of
  them leave $1$ unchanged?
\item How many of them map $1$ to $5$?
\item How many permutations are there such that $\sigma(1) <
  \sigma(2)$? 
\item How many permutations are there such that $\sigma(1) < \sigma(2)
  < \sigma(3)$?
\end{enumerate}
\end{zadacha}

\begin{zadacha} 
How many elements are there in $S(A)$  if $A$ is a finite set of $n$
elements?
\end{zadacha}

\begin{zadacha}  
Is it true that $f\circ g = g\circ f$ for any $f, g$?
\end{zadacha}

For any permutation $f\in S(A)$ (``$\in$'' means that the element
belongs to the set) there exists a unique ``inverse permutation''
$f^{-1}$, i.e.\ a permutation such that $f\circ f^{-1} = f^{-1}\circ f
= 1_A$.

A {\bf cyclic permutation} of a set $a,b,c,d, \dots, w$ maps $a$ to
$b$, $b$ to $c$, $c$ to $d$ and so on.  Such a permutation is denoted
by $(a,b,c,d,\dots,w)$.  The number of elements in the brackets is its
{ \bf order}. {\bf Transposition} is a cyclic permutation of order 2;
it permutes two elements and leaves all other elements unchanged.

\begin{zadacha}
Let $\sigma = (123)$, $\tau=(34)$. Calculate $\tau \circ \sigma \circ
\tau^{-1}$.
\end{zadacha}

\begin{zadacha}
Prove that permutation (of a finite set) is a product of transpositions.
\end{zadacha}

\begin{zadacha}[*] Is it possible that a product of odd number of
  transpositions be an identity permutation?
\end{zadacha}

\begin{ukazanie} What happens to a polynomial 
$(x_1-x_2)(x_1-x_3)\dots(x_1-x_n)(x_2-x_3)\dots$ (a product of $(x_i -
x_j)$ for all $i < j$) when  $x_i$ and $x_j$ are permuted?
\end{ukazanie}

A permutation group $S(A)$ is endowed with the following structure:
operation of multiplication of permutations, operation of taking
inverse of a permutation, the identity permutation. It is useful to
axiomatize this structure.

\begin{opredelenie}
  Let $G$ be a set with the following operations defined on it: $f, g
  \mapsto f\cdot g$ (``multiplication''), $f\mapsto f^{-1}$ (``taking
  inverse'') and let the ``identity element'' $1_G$ be defined as
  well. Let the following axioms be satisfied:
\begin{enumerate}
\item ``Associativity'': $(f\cdot g) \cdot h= f\cdot (g \cdot h)$
for all $f$, $g$, $h$.

\item ``Identity'': $f \cdot 1_G = 1_G \cdot f = f$ for all $f$.

\item ``Inverse element'': $f \cdot f^{-1} = f^{-1} \cdot f = 1_G$
for all $f$.
\end{enumerate}
In this case we call $G$ a {\bf group}.
\end{opredelenie}

A subset of $G$ which is closed under these operations is called a
{\bf subgroup of $G$ }.

\begin{zadacha} Consider a group $G$. Prove that for any two elements
  $f$ and $g$ of it
\begin{enumerate}
\item If $fg=f$ or $gf=f$ then $g=1$;
\item If $fg=1$ or $gf=1$ then $g=f^{-1}$.
\end{enumerate}
\end{zadacha}

\begin{zamechanie}
  This means that to define a group structure on a set $G$ it suffices
  to define an operation of multiplication. Identity element and
  operation of taking inverse are uniquely determined by it and can be
  then reconstructed.
\end{zamechanie}

\begin{zadacha}\label{grp}
Are these sets (with indicated operations) groups?
\begin{enumerate}
\item Natural numbers with an operation of addition;
\item Integer numbers with an operation of addition;
\item Integer numbers with an operation of multiplication;
\item Rational numbers with an operation of multiplication;
\item Real numbers with an operation of addition;
\item Real numbers with an operation of multiplication;
\item\sttr Planar motions with an operation of composition;
\item Numbers strictly greater than $-1$ and strictly less than $1$ with
  an operation defined by the formula $u * v = (u+v)/(1+uv)$ (check
  that the operation is defined correctly);
\item Figures (sets of points) on a plane with an operation of union;
\item\sttr Figures (sets of points) on a plane with an operation of
  symmetric difference: $A * B$ contains points that belong to
  precisely to one of the figures ($A$ or $B$);
\item\sttr Mappings from a fixed set $C$ into a fixed group $G$ with
  an operation $(f \cdot g)(s) = f(s)g(s)$.
\end{enumerate}
\end{zadacha}

``Product of groups'' $G_1$ and $G_2$ is a set of pairs $(g_1,
g_2)$, $g_1\in G_1, g_2\in G_2$ with an operation
$$
(g_1, g_2) \cdot (g_1', g_2') =(g_1\cdot g_1', g_2\cdot g_2')
$$
A mapping $f:G \to G'$ from a group $G$ into a group $G'$ is called a
{\bf homomorphism} if it preserves the multiplication: $f(g_1 \cdot
g_2) = f(g_1) \cdot f(g_2)$. A homomorphism is called a {\bf
monomorphism}, if it is injective, an {\bf epimorphism} if it is
surjective and an {\bf isomorphism} if it is bijective. Groups $G$,
$G'$ are {\bf isomorphic} if there exists an isomorphism between
them. An isomorphism of a group into itself is called an {\bf
automorphism}.

\begin{zadacha} Prove that if $f:G \to G'$ is a homomorphism of groups
  then for any $g \in G$ $f(1_G) = 1_{G'}$ and
  $f(g^{-1})=(f(g)^{-1})$.
\end{zadacha}

\begin{opredelenie}
  If a homomorphism $G \to S(A)$ of a group $G$ into a group $S(A)$ of
  permutations of a set $A$ is defined then one says that {\bf $G$
    acts on a set $A$} (and indeed every element of $G$ permutes the
  elements of $A$ somehow). The action of $G$ on $A$ can be thought of
  as a mapping $G \times A \overset{\rho}{\to} A$, $a, g \mapsto
  \rho(g, a)$. Sometimes the notation for an action of a group on a set
  is even simpler: $a, g \mapsto g(a)$.
\end{opredelenie}

\begin{zadacha} Prove that every group admits an injective
  homomorphism into a permutation group (of a not necessarily finite
  set).
\end{zadacha}

\begin{ukazanie} Think about what the meaning of the following phrase
  can be: ``Group $G$ acts on itself by multiplication on the left''.
\end{ukazanie}

\begin{zadacha} Is it true that
\begin{enumerate}
\item every group that consists of two elements is isomorphic to 
  permutation group $S_2$;
\item every group that consists of six elements is isomorphic either
  to permutation group $S_3$ or to the product of two non-trivial
  (i.e. those that have more than one element) groups.
\end{enumerate}
\end{zadacha}

\begin{zadacha}[*] Prove that permutation group $S_n$ is not
  isomorphic to the product of two non-trivial groups.
\end{zadacha}

\begin{zadacha}
Let $G$ be a group and $g \in G$ its element. Is it true that the
sequence $g,g^2,g^3,\dots$ is periodic? Is it true if  $G$ is a finite
group?
\end{zadacha}

Let $n$ be a positive integer. One says that $g \in G$ is an {\bf
element of order $n$} in a group $G$ if $g^n = 1_G$ but $g^k \neq 1_G$
for any $k < n$.

\begin{zadacha}[!]
Consider a finite group of $n$ elements. Prove that $n$ is divisible
by the order of every element of that group.
\end{zadacha}

\begin{ukazanie}
Consider the action of the group on itself by multiplication on the
left.
\end{ukazanie}

\begin{zadacha}[*]
  Consider a group with an even number of elements. Prove that it
  contains an element of order 2.
\end{zadacha}

\begin{zadacha}[*]
  Is it true that
\begin{enumerate}
\item the group $D_{12}$ of rotations of a regular 12-gon is
  isomorphic to a product $D_{6}\times S_2$ where $D_6$ is a group of
  rotations of a regular hexagon;
\item the group $D_6$ is isomorphic to a product $D_{3}\times S_2$
  where $D_3$ is a group of rotations of a triangle.
\end{enumerate}
\end{zadacha}

A group is called {\bf commutative} or an {\bf Abelian group} if
$f\cdot g = g \cdot f$ for all $f, g$. Two elements $f, g$ {\bf
  commute} if $f\cdot g = g \cdot f$.

\begin{zadacha}
Which groups out of those considered in the exercise~\ref{grp} are
commutative?
\end{zadacha}

\begin{zadacha}[*]
\begin{enumerate}

\item 
A {\bf center} of a group $G$ is a set consisting of all the elements
$g\in G$ such that $g g' = g' g$ for all $g'\in G$. Prove that center
is a subgroup.

\item Consider a group $G$ such that there exist an element in it of
  order $>2$. Consider a subgroup $G'$ such that all elements $g\in G$
  that do not belong to $G'$ have order 2. Give an example of such a
  situation (or prove that it is not possible). Is $G$ always finite
  when the mentioned conditions hold?\label{pa}

\item In the conditions of a previous question prove that $G'$ is an
  abelian group

\item Let $G'$ contain a center $G$. Prove that a group $G$ is
  uniquely (up to an isomorphism) determined by the $G'$ subgroup if
  the condition from \thetag{\ref{pa}} holds. ($G'$ is called then a
  dihedral group)

\item Consider a dihedral group $G$ corresponding to an abelian group
$G'$ as shown above. Let $G'$ be a product of $S_2$ and some other
abelian group: $G' = S_2 \times G''$. Prove that $G$ is a product of
$S_2$ and a dihedral group.
\end{enumerate}
\end{zadacha}

\subs{Rings and fields}

Consider real numbers, integer numbers and finite decimal
fractions. There are the following operations defined on these structures
\begin{enumerate}
\item Addition which is commutative and makes a group out of a set
  (addition is designated as ``+''; taking an inverse element is
  designated as ``-'')

\item Multiplication which is also commutative but does not make a
  group out of any of the considered sets because some elements are
  non-invertible (multiplication is designated by a dot; the dot is
  often omitted: one writes $xy$ instead of $x\cdot y$).
\end{enumerate}
It is useful to axiomatize these structures.

\begin{opredelenie} 
Let $R$ be a set with two operations $a, b \mapsto a+b$
(addition) and $a, b \mapsto a\cdot b$ (multiplication). Let elements
$0$ and $1$ (zero and identity) be defined in $R$. If the following
holds then $R$ is called a {\bf ring}:
\begin{enumerate}
\item $R$ is a commutative group with respect to the operation of
  addition, $0$ is a the identity element in this group

\item $1$ is an identity with respect to multiplication: $1 \cdot a = a
\cdot 1 =a$ for all $a$.

\item Associativity for multiplication: $a \cdot (b \cdot c) = (a \cdot
  b) \cdot c$.

\item Distributivity: $a\cdot (b+c) = a \cdot b + a\cdot c$.
\end{enumerate}
If the multiplication is commutative then one says that a ring $R$ is
commutative. If, moreover, the multiplication is invertible for all
$a\neq 0$, i.e. $R \backslash \{0\}$ is a group with respect to
multiplication then $R$ is called a {\bf field}.
\end{opredelenie}

In this chapter as well as in several following chapters we will
consider only commutative rings and we will omit the word
``commutative'' for brevity; unless it is stated otherwise explicitly
all the rings are assumed to be commutative.

\begin{zadacha}\label{rings}
  Are the following sets (equipped with natural operations unless they
  are specified explicitly) the rings:
\begin{enumerate}
\item natural numbers
\item integer numbers
\item even integer numbers
\item rational numbers
\item irrational numbers
\item finite decimal fractions
\item pairs of integer numbers with the coordinatewise addition and
multiplication
\item pairs of integer numbers with the coordinatewise addition and
  multiplication defined by the formula
  $(a,b)\cdot(c,d)=(ac-bd,ad+bc)$
\item\sttr\label{sqrt2} pairs of rational numbers with the
coordinatewise addition and multiplication defined by the formula
$(a,b)\cdot(c,d)=(ac+2bd,ad+bc)$.
\item\sttr figures on the plane (addition is symmetric difference,
  multiplication is intersection).
\item\sttr mappings from a fixed set $C$ into a fixed group $G$ with
  an operation $(f \cdot g)(s) = f(s)g(s)$.
\end{enumerate}
\end{zadacha}

\begin{zadacha}
Which rings from the exercise~\ref{rings} are fields?
\end{zadacha}

\begin{zadacha}\label{poly} 
Consider a ring $R$. Consider a set of sequences
$$ 
a=(a_0,a_1,\dots,a_i,\dots,0,0,\dots)
$$
consisting of elements of $R$ with the finite number of non-zero
elements. Define the operations on this set as follows
$$
\begin{aligned}
(a+b)_i &= a_i+b_i,\\
(a \cdots b)_i &= \sum_{j=0}^i a_jb_{i-j}.
\end{aligned}
$$
Prove that this set is a ring (check in particular that multiplication
is associative).
\end{zadacha}

The ring defined in the exercise~\ref{poly} is called a {\bf ring of
  polynomials of single variable} on $R$, it is denoted by $R[x]$.
Elements of $R[x]$ are called ``polynomials''. They are usually
written down in the form $a_0 + a_1x + \dots + a_ix^i$ (for all $j>i$,
$a_j$ are zero).

In the algebra course we will suppose the notion of a real number
known, (for example, you can think about real numbers as of infinite
decimal fractions with usual operations defined on the fractions). The
rigorous definition is given in the course of geometry, topology and
analysis. All we need in the algebra course is

\medskip

\noindent
{\bf Important note:} Real numbers form a field.

\medskip

This ``important note'' is also proven in the course of geometry,
topology and analysis. Besides that, we need the following property :

\begin{zadacha}[*]
Prove that every equation of the form
$$
x^{2n+1} + a_{2n} x^{2n} + a_{2n-1} x^{2n-1} + \cdots + a_1 x + a_0 =0
$$
has a real solution.
\end{zadacha}

You should try solving this exercise when you are familiar with the
notion of a real number.

\begin{zadacha}[*]
Will the ring defined in the exercise~\ref{rings}~\ref{sqrt2} be a
field if we change ``rational numbers'' for ``real numbers'' in the
definition?
\end{zadacha}

\begin{zadacha} Consider a fixed natural number $n$. Natural numbers
  divided by $n$ have remainders $0$, $1$, $2$, \dots $n-1$.  Let us
  denote the operation of taking remainders by $\mod n$. Two numbers
  that have the same remainders $\mod n$ are called equal modulo $n$.
  Let us define addition and multiplication on a set of numbers $\mod
  n$ in such a way that
\begin{align*}
(x \mod n) + (y \mod n) &= ((x  + y) \mod n), \\
(x \mod n) \cdot (y \mod n) &= (xy \mod n)
\end{align*}
would hold for all pairs of integer numbers $x, y$.  Prove that this
definition is correct and the set of remainders form a ring.
\end{zadacha}

\begin{zadacha}[*]\label{modulo}
Prove that the set of remainders $\mod n$ with the addition and
multiplication defined as above form a field iff $n$ is a prime
number.
\end{zadacha}

\begin{zamechanie}
If you cannot solve this problem right away, put it away: the same
problem will be reintroduced without an asterisk after defining some
useful intermediate notions.
\end{zamechanie}

\begin{zadacha}
Build the field which consists of
\begin{enumerate}
\item 2 elements
\item 3 elements
\item\sttr 4 elements.
\end{enumerate}
\end{zadacha}

\begin{zadacha}[*]
Prove that there is no field that consists of 6 elements
\end{zadacha}

\begin{zadacha}
Prove that if $p$ is a prime number then a field that consists of $p$
elements is unique up to isomorphism.
\end{zadacha}

\begin{opredelenie} {\bf Characteristic} of a field $k$ is $0$ if $1
  \in k$ has infinite order with respect to addition, otherwise it is
  equal to the order $p$ of an element $1 \in k$ if it is finite.
\end{opredelenie}

\begin{zadacha}
Prove that if the characteristic $p$ of a field $k$ is not zero then
$p$ is a prime number.
\end{zadacha} 

\begin{zadacha}[*]
  Consider a field of characteristic $p$. Prove that Frobenius mapping
  $x \mapsto x^p$ preserves multiplication and addition (just like
  with the groups such a mapping is called a homomorphism).
\end{zadacha}

\begin{ukazanie} Use the binomial theorem.
\end{ukazanie}

\begin{zadacha}[*]
Deduce Fermat's little theorem from that: $x^p$ is equal to $x$
modulo $p$ for any integer number $x$.
\end{zadacha}

Let $P= x^n + a_{n-1} x^{n-1} + \cdots + a_1 x + a_0$ be a polynomial
with the coefficients in the field $k$. A {\bf root} $P$ is an
element $\alpha$ of a field $k$ such that $P(\alpha)=0$.

\begin{zadacha} Let $\alpha$ be the root of a polynomial $P$ over a
  field $k$. Prove that a polynomial $P$ can be divided by $z-\alpha$
  in the ring $k[z]$
\end{zadacha}

\begin{ukazanie} Use the long division of polynomials:
$$
\arraycolsep=0.05em
\begin{array}{rrr@{\,}r|r}
x^2&{}+2x&{}-12&&\,x+5\\
\cline{5-5}
x^2&{}+5x&&&\,x-3\\
\cline{1-2}
&{}-3x&{}-12\\
&{}-3x&{}-15\\
\cline{2-3}
&&3
\end{array}
$$
\end{ukazanie}

\begin{zadacha} Prove that nonzero polynomial of degree $n$ over a
  field cannot have more than $n$ different roots.
\end{zadacha}

\begin{ukazanie} Use the previous exercise.
\end{ukazanie}

Let $P$ be a nonzero polynomial over a field $k$. A polynomial $P$
 is called {\bf irreducible} if it cannot be represented as a product
 of polynomials of smaller degree.

Consider the set of remainders modulo $P$ in a ring $k[x]$.

\begin{zadacha} Prove that this is a ring (we denote it by $k[x] \mod
  P$).
\end{zadacha}

\subs{Complex numbers}

The set of integer numbers is denoted by $\Z$ and the set of real
numbers is denoted by $\R$. Let $\C$ be a set of pairs of real numbers
$(a,b)$ with addition defined by the formula $(a, b)+(c,d) = (a+c,
b+d)$ and with multiplication defined by formula
$$
(a, b) \cdot (c,d) = (ac - bd, ad + bc).
$$
Elements of $\C$ are called complex numbers.

\begin{zadacha} Check that $\C$ is a ring. Prove that an equation
  $x^2+1=0$ has a solution in $\C$. How many solutions does it have?
\end{zadacha}

\begin{zadacha} Let us take a solution of an equation $x^2+1$ in $\C$
  and denote it by $\1$. Prove that any complex number can be uniquely
  represented in the form $a + b \1$, $a,b \in \R$.
\end{zadacha}

\begin{zadacha}
Build an isomorphism $\C \cong (\R[x] \mod P)$ where $P$ is a polynomial
$P= x^2+1$.
\end{zadacha}

\begin{zadacha} Consider a complex number $z:=a+b\1$. A number conjugate
to $z$ is the number $\bar z := a-b\1$. Prove that complex conjugation
preserves multiplication and addition in $\C$ (such mappings are
called automorphisms of the field $\C$).
\end{zadacha}

\begin{zadacha} Consider a complex number $z:=a+b\1$. Prove that 
$z\bar z$ is real (it means that in the representation of a complex
  number $(x,y)$ the component $y$ is zero).
\end{zadacha}

\begin{zadacha} Consider a complex number $z:=a+b\1$. Prove that $z\bar z = a^2
+ b^2$. That means in particular the this number is always nonnegative and
equals zero only if $z = 0$. $z\bar z$ is often written down as $|z|^2$ since
the length of a vector $(a,b)$ on a plane equals $\sqrt{a^2 + b^2}$ (the
distance between $z$ and $0$, $|z|$ is called a modulus of $z$).
\end{zadacha}

\begin{zadacha} 
Deduce from the previous problem that complex numbers form a field.
\end{zadacha}

\begin{ukazanie} $z^{-1} = \bar z |z|^{-2}$
\end{ukazanie}

\begin{zadacha} Prove ``triangle inequality'': 
$|z_1| - |z_2| \leq |z_1 + z_2| \leq |z_1|+ |z_2|$
\end{zadacha}

\begin{zadacha} Prove that $|z_1 z_2| = |z_1||z_2|$.
\end{zadacha}

\begin{zadacha}[!]\label{povorot}
Let $z=a+b\1$ be a complex number with the modulus equal to 1:
$|z|=1$. Let us regard the multiplication by $z$ as a transformation
of a plane $\R^2$ associated naturally with $\C$. Prove that if $z
\neq 1$ then this transform is planar motion with a single fixed point
$0 \in \R^2$.
\end{zadacha}

\begin{zadacha}[!]
  It is known from geometry that a planar motion with the only fixed
  point $0 \in \R^2$ is a rotation by some angle $\phi$ around $0$.
  Given $\phi$, how $a$ and $b$ can be found in task~\ref{povorot}?
\end{zadacha}

\begin{zamechanie}
The angle $\phi$ is called an {\bf argument} of complex number $z$.
\end{zamechanie}

\begin{zadacha}[!]
Prove the formula $\cos (\phi +\psi)= \cos\phi \sin \psi +
\sin\phi \cos \psi$.
\end{zadacha}

\begin{ukazanie}
Use the previous problem.
\end{ukazanie}

\begin{zadacha}[!]
Prove that an equation $z^n =1$ has precisely $n$ complex solutions.
\end{zadacha}

\begin{ukazanie}
Use the trigonometric interpretation of complex numbers.
\end{ukazanie}

\begin{zadacha}[*]
Consider a polynomial $P$ of a degree less than $n$ and let  $\zeta_1, \dots,
\zeta_n$ be ``the roots of $n$-th degree from 1''  or, simply put,
let $\zeta_1, \dots, \zeta_n$ be all complex $z^n =1$. Prove that the
mean $\frac 1 n \sum P(\zeta_i)$ of values of $P$ in all the points
$\zeta_i$ equals $P(0)$.
\end{zadacha}

\begin{ukazanie}
Use the trigonometric interpretation of complex numbers.
\end{ukazanie}

\begin{zadacha}[*]
Consider a polynomial $P$ of a degree less than $n$. Let $\Xi$ be a regular
$n$-gon on a complex plane $\C = \R^2$. Prove that the value of $P$ in
the center of $\Xi$ equals to the mean of values of $P$ in the
vertices of $\Xi$.
\end{zadacha}

\begin{ukazanie} Use the previous problem.
\end{ukazanie}

\begin{zamechanie} Archimedes defined the perimeter of a circle
  as a limit of perimeters of polygons inscribed into it. If we follow
  Archimedes then we can define the mean of a function $f$ defined on a
  circle as a limit (by $n$) of means $\frac 1 n \sum f(\zeta_i)$
  where $z_i$ are vertices of regular $n$-gons inscribed into the
  circle. One can deduce from the previous problem that the mean of
  values of a polynomial function $P$ on a unity circle $|z|=1$ equals
  the value of $P$ in its center.
\end{zamechanie}

\begin{zadacha}
Calculate the group of automorphisms of $\C$
\begin{enumerate}
\item\sttr which translate $\R\subset \C$ into itself.
\item which translate the subfield $\R\subset\C$ into itself and do
  not move its elements.
\end{enumerate}
\end{zadacha}

\begin{zadacha}[!]
Let us change the definition of complex numbers. Instead of
$$
(a, b) \cdot (c,d) = (ac - bd, ad + bc).
$$
let us put
$$
(a, b) \cdot (c,d) = (ac + bd, ad + bc).
$$
Let us denote the obtained structure by $\R_2$. Is $\R_2$ a ring? Is
it a field?  Find all solutions of an equation $z^2=1$ in $\R_2$.
Find all the solutions of an equation $z^2=0$ in $\R_2$.
\end{zadacha}

\begin{zadacha}[!]
Let us change the definition of complex numbers. Instead of
$$
(a, b) \cdot (c,d) = (ac - bd, ad + bc).
$$
let us put
$$
(a, b) \cdot (c,d) = (ac, ad + bc).
$$
Let us denote the obtained structure by $\R_\epsilon$.  Is
$\R_\epsilon$ a ring? Is it a field? Is it isomorphic to $\R_2$ from
the previous problem?  Find all the solutions of an equation $z^2=1$.
\end{zadacha}

\begin{zadacha}[*]
Find all the solutions of an equation $z^2=z$ in two previous problems.
\end{zadacha}

\begin{zadacha}[*] 
  Let $P=a_n x^n + a_{n-1} x^{n-1} + \cdots + a_1 x + a_0$ be a
  polynomial of degree $n$ with $n$ roots lying outside the unit
  circle. Prove that $\frac {a_k}{a_0} < C_n^k$ where $C_n^k=
  \frac{n!}{k!(n-k)!}$ is a binomial coefficient.
\end{zadacha}

\end{document}
