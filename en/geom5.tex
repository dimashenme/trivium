\documentclass[12pt]{article}

\usepackage{theorem,amsmath,amssymb}
\usepackage{srcltx}



\addtolength{\topmargin}{-23mm}
\addtolength{\textheight}{60mm}
\addtolength{\oddsidemargin}{-20mm}
\addtolength{\textwidth}{40mm}

\def\eqref#1{(\ref{#1})}
\newcommand{\goth}{\mathfrak}
\newcommand{\arrow}{{\:\longrightarrow\:}}
\def\1{\sqrt{-1}\:}
\newcommand{\restrict}[1]{{\left|_{{\phantom{|}\!\!}_{#1}}\right.}}

\renewcommand{\bar}{\overline}
\renewcommand{\phi}{\varphi}
\renewcommand{\epsilon}{\varepsilon}
\renewcommand{\geq}{\geqslant}
\renewcommand{\leq}{\leqslant}

\def\rad{\operatorname{\sf rad}}
\def\tr{\operatorname{\sf tr}}
\def\rk{\operatorname{\sf rk}}
\def\Alt{\operatorname{\sf Alt}}
\def\Sym{\operatorname{\sf Sym}}
\def\Id{\operatorname{\sf Id}}
\def\Hom{\operatorname{Hom}}
\def\Map{\operatorname{Map}}
\def\Gal{\operatorname{Gal}}
\def\Aut{\operatorname{Aut}}
\newcommand{\End}{\operatorname{End}}
\newcommand{\Mat}{\operatorname{Mat}}

\newcommand{\coker}{\operatorname{Coker}}

\def\chpoly{\operatorname{\sf Chpoly}}
\def\minpoly{\operatorname{\sf Minpoly}}

\def\cchar{\operatorname{\sf char}}

\def\Z{{\mathbb Z}}
\def\R{{\mathbb R}}
\def\C{{\mathbb C}}
\def\Q{{\mathbb Q}}
\def\N{{\mathbb N}}
\def\F{{\mathbb F}}

\def\Re{\operatorname{Re}}
\def\Im{\operatorname{Im}}

\makeatletter
\theoremstyle{definition}

\newtheorem{zadacha}{������}[section]
\newtheorem{opredelenie}{�����������}[section]
\newtheorem*{ukazanie}{��������}%[section]
\newtheorem*{zamechanie}{���������}%[section]

%\renewcommand{\labelenumi}{\ralph{enumi}.}
\renewcommand{\labelenumi}{\alph{enumi}.}
\newcommand{\subs}[1]{{\bigskip\centerline{\bf\large #1}\bigskip}}
\newcommand{\sttr}{{\bf(*)}}
\newcommand{\shrk}{{\bf(!)}}
\newcommand{\doublesttr}{{\bf(**)}}

\newcommand{\listok}[2]{%
\setcounter{page}{1}
\renewcommand{\@oddhead}{\hfil #2 \hfil}
\renewcommand{\@evenhead}{\hfil #2 \hfil}
\section*{#2}
\refstepcounter{section}
\setcounter{section}{#1}
}

\@addtoreset{equation}{section}
\renewcommand{\theequation}{\thesection.\arabic{equation}}

\let\oldllim=\lim
\def\lim{\oldllim\limits}
\makeatother


\begin{document}

%%%%%%%%%%%%%%%%%%%%%%%%%%%%%%%%%%%%%%%%%%%%%%%%

\listok{5}{GEOMETRY 5: Point-set topology.}

%%%%%%%%%%%%%%%%%%%%%%%%%%%%%%%%%%%%%%%%%%%%%%%%

\begin{opredelenie}
  Consider a set $M$ and a collection of distinguished sets $S\subset
  \wp(M)$ called {\bf open subsets}. A pair $(M,S)$ (and, by abuse of
  notation, $M$ itself) is called a {\bf topological space}, if the
  following conditions are met:
\begin{enumerate}
\renewcommand{\labelenumi}{\arabic{enumi}.}
\item An empty set and $M$ are open;

\item The union of any number of open sets is open;

\item The intersection of a finite number of open sets is open.
\end{enumerate}
A mapping $\phi:\; M \arrow M'$ of topological spaces is called
 {\bf continuous}, if the preimage of every open set is
 open. Continuous mappings are also called {\bf morphisms} of
 topological spaces. An {\bf isomorphism} of topological spaces is a
 morphism $\phi:\; M \arrow M'$ such that there is an inverse morphism $\psi:\;
 M' \arrow M$ (i.e.\ $\phi\circ \psi $ and $\psi\circ \phi $ are
 identity morphisms). An isomorphism of topological spaces is called
 {\bf homeomorphism}. 

 A subset $Z\subset M$ is called {\bf closed}, if its complement is
 open. A {\bf neighborhood } of a point $x\in M$ is an open subset of
 $M$ which contains $x$. A {\bf neighborhood} of a subset $Z\subset
 M$ is an open subset of $M$ that contains $Z$.
\end{opredelenie}

\begin{zadacha}
Prove that a composition of continuous mappings is continuous.
\end{zadacha}

\begin{zadacha}[!]
  Consider a set $M$ and let $S$ be a set of all subsets of $M$. Prove
  that $S$ defines a topology on $M$. This topology is called {\bf
    discrete}. Describe a set of all continuous mappings from $M$ to a
  given topological space.
\end{zadacha}

\begin{zadacha}[!]
  Consider a set $M$ and let $S$ be the set containing an empty set and
  $M$ itself. Prove that $S$ defines a topology on $M$. This topology
  is called {\bf codiscrete}. Describe a set of all continuous mappings
  from $M$ to a space with discrete topology.
\end{zadacha}

\begin{zadacha}
  Give an example of a continuous 
bijection between topological spaces that is not
  a homeomorphism.
\end{zadacha}

\begin{zadacha} 
  Consider a subset $Z$ of a topological space $M$.  Open subsets of
  $Z$ are defined to be intersections of the form $Z\cap U$, where $U$  
  is open in $M$.
\begin{enumerate}
\item Prove that this defines a topology on $Z$. Prove that a natural
  embedding $Z\hookrightarrow M$ is continuous.

\item\sttr{} Can all the continuous embeddings be obtained in this
  way?
\end{enumerate}
\end{zadacha}

\begin{opredelenie}
  Such a topology on $Z\subset M$ is called the topology {\bf induced
    by $M$}.  We will consider any subset of any topological space as
  a topological space with induced topology.
\end{opredelenie}

\begin{opredelenie}
Consider a topological space $M$, and let $S_0$ be such a collection
of open sets such that any open set can be represented as a union of
sets from $S_0$. Then $S_0$ is called a {\bf base} of
$M$.
\end{opredelenie}

\begin{zadacha}
  Describe all bases of a space $M$ with discrete topology; of a space
  $M$ with codiscrete topology.
\end{zadacha}

\begin{opredelenie}
Consider a metric space $M$.  Recall that a subset
$U\subset M$ is called {\bf open}, if for every point $u\in
U$, $U$ contains a ball of radius $\epsilon >0$ with the center $u$.   
\end{opredelenie}

\begin{zadacha} 
Prove that this definition defines a topology on a metric space.
\end{zadacha}

\begin{opredelenie}
A topological space is called {\bf metrizable} if it can be
obtained from a metric space as described above.
\end{opredelenie}

\begin{zadacha} 
Prove that a discrete space is metrizable and a codiscrete space is
not.
\end{zadacha}

\begin{zadacha} 
Prove that open balls in a metric space $M$ are open.  Prove that
open balls define a base of topology on $M$.
\end{zadacha}

\begin{zadacha}[!]
  Consider a topological space $M$ and two topologies $S$, $S'$ on
  $M$.  Suppose that for every point $m\in M$ and every neighborhood
  $U'\ni m$ which is open in the topology $S'$ there is a neighborhood
  $U\ni m$, $U\subset U'$, which is open in the topology $S$. Prove
  that the identity mapping $(M, S) \overset{i}{\arrow} (M, S')$
  is continuous. Give an example where $i$ is not a homeomorphism.
\end{zadacha}

\begin{zamechanie}
 One says that the topology defined by $S'$ is {\bf  stronger} 
than the topology defined by $S$.
\end{zamechanie}

\begin{zadacha}  
Consider the space $\R^n$ with a norm $\nu$ (see GEOMETRY 3). This norm
defines a metric and hence a topology on $\R^n$. Denote this topology
by $S_\nu$. Let $\nu$, $\nu'$ be two norms satisfying
$C^{-1} \nu'(x) <\nu(x)< C\nu'(x)$ for a fixed $C\in \R$. 
Prove that the identity mapping on $\R^n$ defines a
homeomorphism $(\R^n, S_\nu)\arrow (\R^n, S_{\nu'})$. 
\end{zadacha}

\begin{ukazanie}
Use the previous problem.
\end{ukazanie}

\begin{zadacha}[*]
Consider two norms $\nu$, $\nu'$ on $\R^n$ such that the identity
mapping on $\R^n$ defines a homeomorphism $(\R^n, S_\nu)\arrow (\R^n,
S_{\nu'})$. Prove that there exists a constant $C$ such that $C^{-1}
\nu'(x) <\nu(x)< C \nu'(x)$. 
\end{zadacha}

\begin{zadacha}[*]
  Consider a finite-dimensional vector space $V$ endowed with a
  symmetric positive bilinear form $g$. We will consider $V$ as a metric space
  with the metric $d_g$, constructed in GEOMETRY 3. Denote by $S_g$ 
the topology
  defined by $d_g$. Prove that the corresponding topology on $V$ does
  not depend upon $g$, i.e.\ for any 
(symmetric positive bilinear)  $g, g'$, the identity map on
  $V$ is a homeomorphism $(V, S_g)\arrow (V, S_{g'})$.
\end{zadacha}

\begin{zadacha}[**]
  Consider a finite-dimensional vector space $V$ with norm $\nu$.
  Prove that the topology $S_\nu$ does not depend on norm $\nu$: the
  identity map on $\R^n$ is a homeomorphism $(\R^n, S_\nu)\arrow
  (\R^n, S_{\nu'})$. Is it true for a infinite-dimensional $V$?
\end{zadacha}

\begin{opredelenie}
Consider a metric $d$ on $\R^n$, defined by the norm
\[ 
|(\alpha_1, \dots, \alpha_n)|= \sqrt{\sum_i \alpha_i^2}.
\]
The topology on $\R^n$, defined by $d$ is called the {\bf natural}
topology. The {\bf natural topology} on subsets of $\R^n$ is 
the topology induced by the natural $\R^n$-topology.
\end{opredelenie}

\begin{zadacha} 
Consider $\R$ with the natural topology.  Consider a space $M$ with
discrete topology and a space $M'$ with a codiscrete topology. Find the
set of all continuous maps 
\begin{enumerate}
\item from $\R$ to $M$

\item from $M$ to $\R$

\item from $M'$ to $\R$

\item from $\R$ to $M'$.
\end{enumerate}
\end{zadacha}

\begin{zadacha} 
Consider a mapping $\phi:\; M \arrow M'$, where $M,M'$ are topological
spaces. Is it true that the continuity of $\phi$ implies that the preimage 
of any closed set is closed? Is it true that if a preimage of any
closed set is closed then $\phi$  is continuous?
\end{zadacha}

\begin{zadacha}
Give an example of a continuous mapping of topological spaces such
that the image of an open set is not open. Give an example of a
continuous mapping of topological spaces such that the image of a
closed set is closed.
\end{zadacha}

\begin{opredelenie}
  Consider a topological space $M$ and arbitrary $Z\subset
  M$.  The intersection of the closed sets of $M$ containing $Z$
   is denoted by $\overline{Z}$ and is called the {\bf closure}
  of $Z$.
\end{opredelenie}

\begin{zadacha}
Prove that $\overline Z$ is closed.
\end{zadacha}

\begin{opredelenie}
Consider a topological space $M$. The following conditions T0-T4
are called {\bf separation axioms.}
\begin{enumerate}
\renewcommand{\labelenumi}{{\bf T\arabic{enumi}.}}
\setcounter{enumi}{-1}
\item Let $x\neq y \in M$. Then at least one of
  the points $x$, $y$ has a neighborhood  containing the other point.

\item Every point in $M$ is closed.

\item For any $x\neq y\in M$ there are non-intersecting
  neighborhoods $U_x$, $U_y$.

\item For any point $y\in M$, every $M\supseteq U\ni y$ contains an
  open neighborhood $U'\ni y$ such that $U$ contains the closure of
  $U'$.

\item For any closed subset $Z\in M$, any neighborhood
$U\supset Z$ contains an open neighborhood $U'\supset Z$ such that
$U$ contains the closure of $U'$.
\end{enumerate}
The condition $T_2$ is widely known as the {\bf Hausdorff
  axiom}. A topological space that satisfies the $T_2$ condition is
called a {\bf Hausdorff}.
\end{opredelenie}

\begin{zadacha}
  Prove that the condition $T_1$ is equivalent to the following one:
  for any two distinct points $x, y\in M$, there exists a
  neighborhood of $y$, which does not contain $x$.
\end{zadacha}

\begin{zadacha}
Prove that the condition $T_4$ is equivalent to the following one: any
two distinct closed sets $X, Y\subset M$ have two non-intersecting
neighborhoods.
\end{zadacha}

\begin{zadacha} Let $M$ be a topological space. Consider an
  equivalence relation on $M$ defined the following way: $x$ is
  equivalent to $y$ iff $x \in \overline{\{y\}}$ and $y \in
  \overline{\{x\}}$. Denote the set of equivalence classes as $M'$.
\begin{enumerate}
\item Verify that this is indeed an equivalence relation .
Prove that $M$ satisfies the $T_0$ iff $M=M'$.

\item Define $U \subset M'$ to be open iff its preimage w.r.t. the 
  mapping $M \to M'$ is open. Prove that this defines a topology on
  $M'$. Does it satisfy the $T_0$ condition?

\item Prove that the open subsets of $M$ are exactly the preimages of the open
  subsets of $M'$.

\item Suppose that $M$ has the codiscrete topology. What is $M'$?
\end{enumerate}
\end{zadacha}

\begin{zadacha}
Are $T_0-T_4$ conditions satisfied by a space with the discrete topology?
With the codiscrete topology?
\end{zadacha}

\begin{zadacha}
Prove that $T_0-T_4$ are satisfied by $\R$.
\end{zadacha}

\begin{zadacha}
Prove that $T_1$ implies $T_0$ and that $T_2$ implies $T_1$.
\end{zadacha}

\begin{zadacha} 
Give an example of a space that does not satisfy the $T_1$
condition. Give an example of a non-Hausdorff space such that all
the singleton sets are closed in it.
\end{zadacha}

\begin{zadacha}[*]
Give an example of a space that satisfies the $T_1$ condition such
that any two non-empty open sets have an non-empty intersection.
\end{zadacha}

\begin{zadacha}[*]
Prove that $T_2$ follows from $T_1$ and $T_3$.
\end{zadacha}

\begin{zadacha}[*]
  Give an example of a space that satisfies $T_4$ but does not satisfy
  $T_1$.
\end{zadacha}

\begin{zadacha} 
Consider a metrizable topological space. Prove that it satisfies
conditions $T_1$, $T_2$, $T_3$.
\end{zadacha}

\begin{zadacha}[*]
Consider a metrizable topological space. Prove that it satisfies the
condition $T_4$. 
\end{zadacha}

\begin{zadacha}[*]
Let $M$ be a finite set.
\begin{enumerate}
\item Find all topologies on $M$ that satisfy the $T_1$ condition.
\item Are there any topologies on $M$ that do not satisfy $T_1$?
\item Are there any topologies on $M$ that do not satisfy $T_1$, but
  satisfy $T_0$?
\end{enumerate}
\end{zadacha}

\begin{opredelenie}
  A set $M$ is called a {\bf partially ordered set}, if there is a
  binary relation $x \le y$ (``$x$ less than or equal $y$'') defined
  on it such that:
\begin{enumerate}
\renewcommand{\labelenumi}{\arabic{enumi}.}
\item If $x \le y$ and $y \le z$, then $x \le z$.
\item If $x \le y$ and $y \le x$, then $x=y$.
\end{enumerate}
\end{opredelenie}

\begin{zadacha}[*]
\begin{enumerate}
\item Consider a partially ordered set $M$; say that $S \subset M$ 
is open if together with any  $x \in S$ it
  contains all $y \in M$ satisfying $y \le x$. Prove that this defines
  a topology on $M$. When does this topology satisfy the $T_0$
  condition? The $T_1$ condition?
\item Consider a finite set $M$ and a topology on $M$ 
that satisfies the $T_0$
  condition. Prove that it is induced by a partial order on $M$.
\end{enumerate}
\end{zadacha}

\begin{opredelenie}
  Let $Z\subset M$ be a subset of a topological space.  A subset $Z$
  is called {\bf dense}, if $Z$ has a non-empty intersection with
  every open subset of $M$.
\end{opredelenie}

\begin{zadacha}[!]
  Prove that $Z$ is dense iff the closure $\overline{Z}$ is the entire
  $M$.
\end{zadacha}

\begin{zadacha}
  Find all dense subsets in a topological space with the discrete
  topology; with the codiscrete topology.
\end{zadacha}

\begin{zadacha}
Prove that $\Q$ is dense in $\R$.
\end{zadacha}

\begin{zadacha}[!]
A subset $Z$ in a topological space $M$ is called {\bf nowhere dense},
if for every open $U\subset M$ the subset $Z\cap U$ is not
dense in $U$. Prove that $Z$ is nowhere dense iff $M\backslash
\overline{Z}$ is dense in $M$.
\end{zadacha}

\begin{zadacha}[*] 
  Construct a nowhere dense subset of the interval $[0, 1]$ (endowed
  with the natural topology) of the continuum cardinality.
\end{zadacha}

\begin{zadacha}
Find all nowhere dense subsets in a space with discrete topology; with
codiscrete topology.
\end{zadacha}

\begin{opredelenie}
Let $M$ be a topological space and $x\in M$ be an arbitrary
point. A neighborhood base of $x$ is a collection $B$ of
neighborhoods of $x$ such that any neighborhood $U\ni x$ contains some
neighborhood from $B$.
\end{opredelenie}

\begin{zadacha}
Consider a collection $B$ of open subsets of a topological space $M$
such that for any $x\in M$ the collection of all
$U\in B$ containing $x$ is a neighborhood base of $x$. Prove that
$B$ is a base of the topology of $M$.
\end{zadacha}

\begin{opredelenie}
  Consider a topological space $M$. One can impose two countability
  conditions on $M$. If every point of $M$ has a countable
  neighborhood base, then one says that $M$ satisfies {\bf the first
    countability axiom}. If $M$ has a countable base of open sets,
  then one says that $M$ satisfies {\bf the second countability
    axiom}, or that $M$ is a {\bf space with a countable
    base}. If there exists a countable dense subset of $M$  then one says that $M$ is {\bf separable}.
\end{opredelenie}

\begin{zadacha}
Consider a space $M$ with discrete topology. Prove that $M$ satisfies
the first countability axiom.
\end{zadacha}

\begin{zadacha} 
Consider a topological space $M$ with a countable base. Prove that it
is separable.
\end{zadacha}

\begin{zadacha}[*]
Consider a separable topological space $M$. Prove that $M$ has a
countable base. 
\end{zadacha}

\begin{zadacha}[!]
  Consider a metrizable topological space. Prove that it has a
  countable neighborhood base for every point.
\end{zadacha}

\begin{zadacha} 
Construct a non-separable metrizable topological space.
\end{zadacha}

\begin{zadacha}[**]
Give an example of a countable Hausdorff space without a countable
base.
\end{zadacha}

\subsection{Topology and convergence}

Topological space were invented as a language to speak about continuous
functions. In GEOMETRY 4 we defined a continuous function as a
function that preserves limits of convergent sequences. One can
consider topology from the axiomatic viewpoint as above, or from the
point of view of geometric intuition, by giving a class of convergent
sequences on a space to define its topology and considering a mapping
continuous if it preserves limits.

The second approach (despite all its obvious advantages) encounters
set-theoretical problems: if the space does not have a countable base,
then one has to use well-founded uncountable sequences. We are going
to work mostly with spaces which have a countable neighborhood base
and it is convenient to define topology and continuity via limits of
sequences.

\begin{opredelenie} 
  Let $M$ be a topological space and $Z\subset M$ be an infinite
  subset.  A point $x\in M$ is called an {\bf accumulation point} of
  $Z$, if every neighborhood of $x$ contains some point $z\in
  Z$. A {\bf limit } of a sequence $\{ x_i\}$ is defined to be a point
  $x$ such that every neighborhood of $x$ contains almost all $x_i$'s.
  A sequence is called {\bf convergent} if it has a limit.
\end{opredelenie}

\begin{zadacha}
Find all convergent subsequences in a space with discrete topology; in
a space with codiscrete topology.
\end{zadacha}

\begin{zadacha}
Consider a Hausdorff space $M$. Prove that every sequence has at most
one limit.
\end{zadacha}

\begin{zadacha}[*]
Is the converse true (i.e.\ does it follow from the uniqueness of a
limit that the space is Hausdorff)?  What if $M$ has a countable
neighborhood base of its point?  
\end{zadacha}

\begin{zadacha} 
  Consider a space $M$  where any sequence has at most one limit.
  Prove that $M$ satisfies the separation axiom $T_1$.
\end{zadacha}

\begin{zadacha}
  Consider a continuous mapping $f:\; M \arrow M'$ and a subset
  $Z\subset M$. Prove that $f$ maps accumulation points of $Z$ to
  accumulation point of $f(Z)$.  Prove that $f$ maps limits to
  limits. 
\end{zadacha}

\begin{zadacha}[!]
Consider a mapping that maps accumulation points to accumulation
points. Prove that it is continuous.
\end{zadacha}

\begin{zadacha} 
Consider a space $M$ with a countable neighborhood base for every
point, and an arbitrary  $Z\subset M$. Prove that the
closure of $Z$ is the set of limits of all sequences from $Z$.
\end{zadacha}

\begin{zadacha}[!]\label{lim.seq}
Consider topological spaces $M$, $M'$ with a countable neighborhood
base for every point and a mapping $f:\; M \arrow M'$ that preserves
limits of sequences. Prove that $f$ is continuous.
\end{zadacha}

\begin{ukazanie}
Use the previous problem.
\end{ukazanie}

\begin{zadacha}[*] 
What happens if we do not require in the previous problem that
neighborhood bases are countable in $M$? In $M'$?
\end{zadacha}

\begin{zadacha}[*]
Consider a set $M$ and let some sequences of elements of $M$ be
declared to {\bf converge} to points from $M$ (it is denoted like this:
$x\in \lim x_i$; note that there can be more than one limit of a
sequence\footnote{Thus we talk here 
about limits of sequences as {\em sets} of points. (DP)}). 
Let the following conditions hold for the notion of
convergence: 
\begin{enumerate}
\renewcommand{\labelenumi}{(\roman{enumi})}

\item The limit of a sequence $x, x, x,x,x, \dots $ contains $x$.

\item If $x\in \lim x_i$ then the limit of any subsequence
$\{x_{i_\ell}\}$ is nonempty and contains $x$.

\item Consider an infinite number of elements of a sequence $\{x_i\}$.
  Let us permute them and denote the result by $\{ y_i\}$. If
  $x\in \lim x_i$ then $x\in \lim y_i$.

\item If $x\in\lim x_i$ and $x\in\lim y_i$ then the sequence $x_1, y_1,
  x_2, y_2, ...$ converges to $x$.
\end{enumerate}
\begin{enumerate}
\item Define closed subsets of $M$ as these 
$Z\subset M$ that contain the limits of all sequences
$\{x_i\}\subseteq Z$.  Define open sets as complements of closed sets.
Prove that this defines a topology on $M$.

\item Consider a topology $S$ on $M$ with a countable
  neighborhood base for every point. Define the limits of sequences
  with respect to this topology. Prove that conditions (i)-(iv) hold
  for this notion of convergence. Let $S'$ be a topology obtained
  from limits with the help of construction in (a). Prove that the
  topologies $S'$ and $S$ coincide.

\item Take a uncountable set with the following topology: open sets
  are complements of finite sets (this topology is called cofinite).
  Consider a topology $S'$ defined by limits as above. Describe $S'$.
  Prove that $S'$ does not satisfy the first countability axiom.

\end{enumerate}
\end{zadacha}


\end{document}
