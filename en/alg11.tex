\documentclass[12pt]{article}

\usepackage{theorem,amsmath,amssymb}



\addtolength{\topmargin}{-23mm}
\addtolength{\textheight}{60mm}
\addtolength{\oddsidemargin}{-20mm}
\addtolength{\textwidth}{40mm}

\def\eqref#1{(\ref{#1})}
\newcommand{\goth}{\mathfrak}
\newcommand{\arrow}{{\:\longrightarrow\:}}
\def\1{\sqrt{-1}\:}
\newcommand{\restrict}[1]{{\left|_{{\phantom{|}\!\!}_{#1}}\right.}}

\renewcommand{\bar}{\overline}
\renewcommand{\phi}{\varphi}
\renewcommand{\epsilon}{\varepsilon}
\renewcommand{\geq}{\geqslant}
\renewcommand{\leq}{\leqslant}

\def\rad{\operatorname{\sf rad}}
\def\tr{\operatorname{\sf tr}}
\def\rk{\operatorname{\sf rk}}
\def\Alt{\operatorname{\sf Alt}}
\def\Sym{\operatorname{\sf Sym}}
\def\Id{\operatorname{\sf Id}}
\def\Hom{\operatorname{Hom}}
\def\Map{\operatorname{Map}}
\def\Gal{\operatorname{Gal}}
\def\Aut{\operatorname{Aut}}
\newcommand{\End}{\operatorname{End}}
\newcommand{\Mat}{\operatorname{Mat}}

\newcommand{\coker}{\operatorname{Coker}}

\def\chpoly{\operatorname{\sf Chpoly}}
\def\minpoly{\operatorname{\sf Minpoly}}

\def\cchar{\operatorname{\sf char}}

\def\Z{{\mathbb Z}}
\def\R{{\mathbb R}}
\def\C{{\mathbb C}}
\def\Q{{\mathbb Q}}
\def\N{{\mathbb N}}
\def\F{{\mathbb F}}

\def\Re{\operatorname{Re}}
\def\Im{\operatorname{Im}}

\makeatletter
\theoremstyle{definition}

\newtheorem{zadacha}{������}[section]
\newtheorem{opredelenie}{�����������}[section]
\newtheorem*{ukazanie}{��������}%[section]
\newtheorem*{zamechanie}{���������}%[section]

%\renewcommand{\labelenumi}{\ralph{enumi}.}
\renewcommand{\labelenumi}{\alph{enumi}.}
\newcommand{\subs}[1]{{\bigskip\centerline{\bf\large #1}\bigskip}}
\newcommand{\sttr}{{\bf(*)}}
\newcommand{\shrk}{{\bf(!)}}
\newcommand{\doublesttr}{{\bf(**)}}

\newcommand{\listok}[2]{%
\setcounter{page}{1}
\renewcommand{\@oddhead}{\hfil #2 \hfil}
\renewcommand{\@evenhead}{\hfil #2 \hfil}
\section*{#2}
\refstepcounter{section}
\setcounter{section}{#1}
}

\@addtoreset{equation}{section}
\renewcommand{\theequation}{\thesection.\arabic{equation}}

\let\oldllim=\lim
\def\lim{\oldllim\limits}
\makeatother


\begin{document}


\listok{11}{ALGEBRA 11: Galois theory}

%%%%%%%%%%%%%%%%%%%%%%%%%%%%%%%%%%%%%%%%%%%%%%%%
%%%%%%%%%%%%%%%%%%%%%%%%%%%%%%%%%%%%%%%%%%%%%%%%
\subs{Galois extensions}
%%%%%%%%%%%%%%%%%%%%%%%%%%%%%%%%%%%%%%%%%%%%%%%%

\begin{zadacha}[!]\label{_pryamaya_summa_polinom_Zadacha_}
Consider a polynomial $P(t)\in K[t]$ of degree $n$ with coefficients
in a field $K$ that has  $n$ distinct roots in $K$. Prove that the
ring  $K[t]/P$ of residues modulo  $P$  is isomorphic to the direct
sum of $n$ copies of $K$.
\end{zadacha}

\begin{ukazanie}
  There was a similar problem in ALGEBRA 9.
\end{ukazanie}

\begin{opredelenie}
  Let $K$ be an algebraic extension of a field $k$ (this fact is often
  denoted in writing by $[K:k]$). One says that $[K:k]$ is a {\bf
  Galois extension} if $K\otimes_k K$ is isomorphic (as an algebra)
  to a direct sum of several copies of $K$.
\end{opredelenie}

\begin{zadacha}
Let $P(t)\in k[t]$ be an irreducible polynomial of degree
$n$ that has  $n$ distinct roots in $K = k[t]/P$.
Prove that $[K:k]$ is a Galois extension.
\end{zadacha}

\begin{zadacha}
Prove that $[\Q[\1]:\Q]$ is a Galois extension.
\end{zadacha}

\begin{zadacha}
Let $[k:\Q]$ be an extension of degree 2 (i.e. $K$ is two dimensianal
as a vector space over $\Q$). 
Prove that it is a Galois extension.
\end{zadacha}

\begin{zadacha}[!]
Let $p$ be a prime.
Prove that for any root of unity  $\zeta$ of degree  $p$
$[\Q[\zeta]:\Q]$ is a Galois extension.
\end{zadacha}

\begin{zadacha}[*]
Is $[\Q[\sqrt[3]{2}]:\Q]$ a Galois extension?
\end{zadacha}

\begin{zadacha}[*]
Consider $F$, a field of characteristic $p$ and $k= F(z)$, the field
of rational functions over  $F$. Prove that the polynomial $P(t) = t^p -z$
is irreducible over $k$. Prove that  $[k[t]/P:k]$ is not a Galois
extension.
\end{zadacha}

\begin{zadacha}
Let  $K_1 \supset K_2 \supset K_3$ be a sequence of field
extensions. Prove that 
\[ K_2 \otimes_{K_3} K_1 \cong (K_2 \otimes_{K_3} K_2)\otimes_{K_2} K_1.\]
\end{zadacha}

\begin{zadacha}
Let $K_1 \supset K_2 \supset K_3$ be a sequence of field
extensions. Prove that 
\[  K_1 \otimes_{K_2}(K_2 \otimes_{K_3} K_2)\otimes_{K_2} K_1 
    \cong K_1 \otimes_{K_3} K_1.
\]
\end{zadacha}

\stepcounter{zadacha}

% \begin{zadacha}[!] 
% Let  $K_1 \supset K_2 \supset K_3$ be a sequence of field 
% extensions such that $[K_1: K_2]$ and $[K_2: K_3]$ are Galois
% extensions. Prove that $[K_1: K_3]$ is a Galois extension.
% \end{zadacha}

\begin{zadacha}
Prove that  $\Q[\sqrt[3]2, \frac{\sqrt{-3}-1}2]$ is a Galois 
extension.
\end{zadacha}

\begin{zadacha} Let $K_1 \supset K_2 \supset K_3$ be a sequence
of field extensions. Prove that the natural map
\[
K_1 \otimes_{K_3}K_1 \arrow K_1 \otimes_{K_2}K_1
\]
is a surjective homomorphism of algebras.
\end{zadacha}

\begin{zadacha}[!]
  Let $K_1 \supset K_2 \supset K_3$ be a sequence of field extensions
  such that $[K_1:K_3]$ is a Galois extension.  Prove that $[K_1:K_2]$
  is also a Galois extension.
\end{zadacha}

\begin{ukazanie}
Use the Problem~9.28 from ALGEBRA 9.
\end{ukazanie}

\begin{zadacha}
  Let $P\in k[t]$ be a polynomial of degree $n$ over the field
  $k$. Let $K_1= k$; consider the sequence of field extensions
  $K_l\supset K_{l-1} \supset \dots \supset K_1$ which is constructed
  as follows. Suppose $K_j$ is constructed. Decompose $P$ into
  irreducible factors $P= \prod P_i$ in $K_j$. If all $P_i$ are linear
  then the construction is over.  Otherwise, let $P_0$ be an
  irreducible factor of $P$ of degree $>1$. Consider
  $K_{j+1}=K_j[t]/P_0$. Prove that this process terminates in a finite
  number of steps and gives some field $K \supset k$.
\end{zadacha}

\begin{opredelenie}
  This field is called a  {\bf splitting field} of the polynomial  $P$.
\end{opredelenie}

\begin{zadacha}[!]
  Let $K$ be a splitting field of a polynomial $P(t)\in k[t]$.  Prove
  that $K$ is isomorphic to a subfield of the algebraic closure $\bar
  k$ that is generated by all roots of $P$.
\end{zadacha}

\begin{zadacha}
Let  $P(t)$ be a polynomial of degree $n$.
Prove that the degree of its splitting field is not greater than 
 $n!$.
\end{zadacha}

\begin{zadacha}
Let $P\in k[t]$ be a polynomial of degree $n$ that has  $n$ pairwise
disjoint roots in the algebraic closure $k$ and let $[K:k]$ be its
splitting field and
$K_l\supset K_{l-1} \supset \dots \supset K_1$
the corresponding sequence of field extensions. Prove that
$K\otimes_{K_{i-1}}K_i$ is isomorphic to a direct sum
of several copies of $K$.
\end{zadacha}

\begin{ukazanie} 
This follows immediately from
Problem~\ref{_pryamaya_summa_polinom_Zadacha_}. 
\end{ukazanie}

\begin{zadacha}[!]
  Let $P(t)\in k[t]$ be an irreducible polynomial of degree $n$ that
  has $n$ pairwise disjoint roots in the algebraic closure $k$ (one
  says that this polynomial has {\bf no multiple roots}) and let $K$
  be its splitting field.  Prove that $[K:k]$ is a Galois extension.
\end{zadacha}

\begin{ukazanie}
Use the previous problem.
\end{ukazanie}

\begin{zadacha}[*]
Let $P(t)\in k[t]$ be an irreducible polynomial over a field
$k$ of characteristic 0. Prove that  $P$ has no multiple roots. 
\end{zadacha}

\begin{ukazanie} 
Prove that $P(t)= t^n + a_{n-1} t^{n-1} + \dots$ doesn't have multiple
roots if and only if $P$ has no common factors with the polynomial 
\[ P'(t) = n t^{n-1} + (n-1) a_{n-1} t^{n-2} + \dots + 2 a_2
t + a_1.
\]
In order to show this, prove that $(PQ)'=PQ'+Q'P$ and compute
$P'(t)$ for $P=(t-b_1)\dots(t-b_n)$.
\end{ukazanie}

\begin{zamechanie} It follows from the previous problem
that over a field of characteristic 0 the splitting field of any
polynomial is a Galois extension.
\end{zamechanie}

\begin{zadacha}[*]
Give an example of a field $k$
(of non-zero characteristic) and an irreducible polynomial
 $P\in k[t]$ such that its splitting field is not a Galois extension. 
\end{zadacha}

%%%%%%%%%%%%%%%%%%%%%%%%%%%%%%%%%%%%%%%%%%%%%%%%
\subs{Galois groups}
%%%%%%%%%%%%%%%%%%%%%%%%%%%%%%%%%%%%%%%%%%%%%%%%

\begin{opredelenie}
Let  $[K:k]$ be a Galois extension. The {\bf Galois group $[K:k]$}
is the group of $k$-linear automorphisms of the field $K$. We denote
the Galois group by $\Gal([K:k])$ or $\Aut_k(K)$.
\end{opredelenie}

In what follows we consider
$K\otimes_k K$ as a $K$-algebra with the action of $K^*$ given by a
formula  $a(v_1\otimes v_2)= av_1 \otimes v_2$.
This action of $K^*$ is called the {\bf left} action.
It is different than the ``right action'' which is defined by the
formula  $a(v_1\otimes v_2)= v_1 \otimes av_2$.

\begin{zadacha} 
Let  $[K:k]$ be a Galois extension. Construct a bijection between the
set of $K$-linear homomorphisms $K\otimes_k K\arrow K$ and the set of
indecomposable idempotents in $K\otimes_k K$.
\end{zadacha}

\begin{zadacha}
Let $\mu:\; K\otimes_k K\arrow K$ be non-zero
$K$-linear homomorphism and $k\otimes_k K\subset
K\otimes_k K$ be a  $k$-subalgebra naturally isomorphic to $K$. Prove
that  $\mu\mid_{k\otimes_k K}$ defines a  $k$-linear automorphism
$K\arrow K$. 
\end{zadacha}

\begin{zadacha} 
  Prove that every $k$-linear automorphism $K$ can be obtained this
  way.
\end{zadacha}

\begin{ukazanie}
  Let $\nu\in \Gal([K:k])$. Define a homomorphism $K \otimes_k K \to
  K$ as follows: $v_1 \otimes v_2 \arrow v_1 \nu(v_2)$.
\end{ukazanie}

\begin{zadacha}[!] 
Let $[K:k]$ be a Galois extension.
Construct the natural bijection between $\Gal([K:k])$
and the set of indecomposable idempotents in $K\otimes_k K$.
Prove that the order of the Galois group is the $k$-vector space dimension
of $K$.
\end{zadacha}

\begin{zadacha} \label{_right_left_ide_Zadacha_} Let $[K:k]$ be a
  Galois extension, $\nu\in \Gal([K:k])$ be an element of the Galois
  group and $e_\nu$ be the corresponding idempotent in $K\otimes_k
  K$. Let $\mu_l$ denote the standard (left) action $K^*$ on
  $K\otimes_k K$, and let $\mu_r$ denote the standard right
  action. Prove that $\mu_l(a) e_\nu = \mu_r(\nu(a)) e_\nu$.
\end{zadacha}

\begin{zadacha}
Let  $[K:k]$ be a Galois extension and 
$a\in K$ be an element invariant under the action of
$\Gal([K:k])$. Prove that $a\otimes 1 = 1 \otimes a$
in $K\otimes_k K$.
\end{zadacha}

\begin{ukazanie} Use the Problem~\ref{_right_left_ide_Zadacha_}.
\end{ukazanie}

\begin{zadacha}[!]
Let  $[K:k]$ be a Galois extension and let
$a\in K$ be an element invariant under the action of
$\Gal([K:k])$. Prove that $a \in k$.
\end{zadacha}

\begin{zadacha}
Let  $[K:k]$ be a Galois extension and let $K'$ be an intermediate
extension,  $K\supset K' \supset k$. Prove that
$K' = K^{G'}$ where $G'\subset \Gal([K:k])$ is the group of
$K'$-linear automorphisms of $K$ and  $K^{G'}$ denotes the set of
elements of $K$ invariant under $G'$.
\end{zadacha}

\begin{ukazanie} Prove that $[K:K']$ is a Galois extension and use the
  previous problem. 
\end{ukazanie}

\begin{zadacha}[!]
  Prove the {\bf Fundamental Theorem of Galois theory}.  Let $[K:k]$
  be a Galois extension. Then $G' \arrow K^{G'}$ defines a bijective
  correspondence between the set of subgroups $G' \subset \Gal([K:k])$
  and the set of intermediate fields $K\supset K' \supset k$.
\end{zadacha}

\begin{zadacha}
  Let $[K:k]$ be a Galois extension and let $K'$ be an intermediate
  field, $K\supset K' \supset k$. Construct the natural correspondence
  between the set of $k$-linear homomorphisms $K' \to K$ and the
  collection $\Gal([K:k])/\Gal([K:K'])$ of cosets of $\Gal([K:K'])
  \subset \Gal([K:k])$ in the Galois group $\Gal([K:k])$ .
\end{zadacha}

\begin{zadacha}
Find the Galois group $[\Q[\sqrt a]:\Q]$.
\end{zadacha}

\begin{zadacha}[!] 
  Let $[K:k]$ be a Galois extension and let $a$ be an element of the
  field $K$ generates $K$ over $k$ (this element is called {\bf
    primitive}).  Prove that if $\nu_1, \nu_2, \dots, \nu_n$ are
  pairwise distinct elements of $\Gal([K:k])$ then $\nu_1(a),
  \nu_2(a), \dots \nu_n(a)$ are linearly independent over $k$.
\end{zadacha}

\begin{zadacha}[!]\label{_primitive_Zadacha_}
  Let $[K:k]$ be a Galois extension and let $V\subset K$ be the union
  of all intermediate fields $k\subset K'\subset K$ which are proper
  subfields of $K$. Suppose that $�$ is infinite. Prove that $V\neq
  K$.
\end{zadacha}

\begin{ukazanie} 
  $V$ is the union of a finitely many $k$-subsspaces of $K$ that have
  a dimension (over $k$) lower than the dimension of $K$ as a linear
  space over $k$.  Prove that in this case $V\neq K$.
\end{ukazanie}

\begin{zamechanie}
It follows that any Galois extension  $[K:k]$ of any infinite field
$k$ has a primitive element.
\end{zamechanie}

\begin{zadacha}[!]
Let  $[K:k]$ be a Galois extension. Prove that for any $a\in K$
the product $P(t)=\prod_{\nu_i \in \Gal([K:k])} (t- \nu_i(a))$ is a
polynomial with coefficients in $k$.
\end{zadacha}

\begin{zadacha}[*]
In the previous problem setting, let $a$ be primitive. Prove that
$P(t)$ is irreducible. 
\end{zadacha}

\begin{zadacha}[!]
Recall that the $n$-th root of unitiy is called
{\bf primitive} if it generates the group of 
$n$-th roots of unity. Let $\xi\in \C$ be a primitive $n$-th
root. Prove that the group $\Gal([\Q[\xi]:\Q])$ is isomorphic to the
group $\Aut(\Z/n\Z)$ of automorphisms of the group $\Z/n\Z$. 
Find its order. 
\end{zadacha}

\begin{zadacha}[*]
Consider an integer $n$.
Let $P(t)= \prod(t-\xi_i)$ where the product is taken over all
primitive $n$-th roots of unity $\xi_i$. Prove that $P(t)$ has
rational coefficients and is irreducible over $\Q$.
\end{zadacha}

\begin{zamechanie}
This polynomial is called  {\bf cyclotomic polynomial}.
\end{zamechanie}

\begin{zadacha}[*]
Find a decomposition of $x^n-1$ into factors irreducible over $\Q$. 
\end{zadacha}

\begin{zadacha} 
Let $a_1, \dots, a_n\in \Z$ be co-prime and non-square numbers. Prove
that  
$[\Q[\sqrt {a_1}, \sqrt {a_2}, \dots, \sqrt {a_n}]:\Q]$ is a Galois
extension. 
\end{zadacha}

\begin{zadacha} 
Find the Galois group of this extension.
\end{zadacha}

\begin{zadacha}[!]
Prove that $\sqrt {a_1}, \sqrt {a_2}, \dots, \sqrt {a_n}$
are linearly independent over $\Q$.
\end{zadacha}


%%%%%%%%%%%%%%%%%%%%%%%%%%%%%%%%%%%%%%%%%%%%%%%%
\subs{Finite fields}
%%%%%%%%%%%%%%%%%%%%%%%%%%%%%%%%%%%%%%%%%%%%%%%%

We know the following facts about finite fields from the previous
problem sheets. The order of a finite fied is  $p^n$ where $p$ is its
characteristic. For any field $k$ of characteristic  $p$
there exists the {\bf Frobenius endomorphism},
$Fr:\; k \arrow k$, $x \mapsto x^p$.
The finite field of $\F_p$ naturally embeds into any field of 
characteristic field $p$.

We denote the field of order $p^n$ by $\F_{p^n}$.

\begin{zadacha} 
Let $x \in \F_{p^n}$, $x\neq 0$. Prove that
$x^{p^n-1}=1$.
\end{zadacha}

\begin{ukazanie} 
Use Lagrange's theorem (the order of an element divides the number of
elements in the group).
\end{ukazanie}

\begin{zamechanie} 
  It follows that the polynomial $P(t) = t^{p^n-1}-1$ has exactly
  $p^n-1$ roots in $\F_{p^n}$.
\end{zamechanie}

\begin{zadacha}[!]
Prove that 
$\prod_{\xi\in \F_{p^n}\backslash 0} = t^{p^n-1}-1$.
\end{zadacha}

\begin{zadacha}[!]
  Prove that $[\F_{p^n}: \F_{p}]$ is a Galois extension.
\end{zadacha}

\begin{zadacha}[!]
Prove that  $Fr, Fr^2, \dots, Fr^{n-1}$ are pairwise distinct
automorphisms of $\F_{p^n}$. 
\end{zadacha}

\begin{zadacha}[!]
Prove that $\Gal([\F_{p^n}: \F_{p}])$ is a cyclic group of order $n$.
\end{zadacha}

\begin{zadacha}[*]
Prove that the splitting field of the polynomial $t^{p^n-1}-1$ over
$\F_p$ has order $p^n$.
\end{zadacha}

\begin{zadacha}[*]
Prove that the field of order $p^n$ is unique up to isomorphism. 
\end{zadacha}

\begin{zadacha}[!]
Find all subfields of  $\F_{p^n}$.
\end{zadacha}

\begin{zadacha}[!]
  Let $[K:k]$ be a Galois extension.  Prove that $K$ has a primitive
  element.
\end{zadacha}

\begin{zamechanie}
We have already proved this for infinite fields, see the remark after
the Problem \ref{_primitive_Zadacha_}.
\end{zamechanie}

%%%%%%%%%%%%%%%%%%%%%%%%%%%%%%%%%%%%%%%%%%%%%%%%
\subs{Abel's theorem}
%%%%%%%%%%%%%%%%%%%%%%%%%%%%%%%%%%%%%%%%%%%%%%%%

Abel's theorem states that a generic polynomial of degree 5 is not
solvable by radicals; in other words, the solution of a generic
equation of degree 5 cannot be expressed using algebraic operations
(multiplication, addition, division) and taking an $n$-th root. In
this section we will give an example of an equation that is not
solvable by radicals. 

\begin{zadacha}
Let $[K:k]$ be a Galois extension.
Prove that the subgroup $G'\subset \Gal([K:k])$
is normal if and only if
$[K^{G'}:k]$ is a Galois extension.
\end{zadacha}

\begin{zadacha}[!]
Let  $G'\subset \Gal([K:k])$ be a normal subgroup. Prove that the group $\Gal([K^{G'}:k])$
is isomorphic to the quotient $\Gal([K:k])/G'$.
\end{zadacha}

\begin{opredelenie} 
A Galois extension  $[K:k]$
is called {\bf cyclic},
if its Galois group is cyclic.
\end{opredelenie}

\begin{zadacha}[!]
Let Galois group of an extension $[K:k]$
be solvable. Prove that 
$[K:k]$ can be broken into a
sequence of Galois extensions
$k=K_0 \subset K_1 \subset ... \subset K_n =K$
so that for any  $i$, $\Gal([K_i:
K_{i-1}])$ is a cyclic group.
\end{zadacha}

\begin{zadacha}[*]
Let $k$ contain all $n$-th roots of
unity and $[K:k]$ be a splitting
field of the polynomial $t^n -a$
which does not have roots over $k$. 
Prove that this extension is
cyclic. 
\end{zadacha}

\begin{ukazanie}
Let $\alpha$ be some root of the polynomial $t^n -a$.
Then all roots of $t^n -a$ are of
the form
$\alpha, \alpha\xi, \alpha\xi^2, \dots, \alpha\xi^{p-1}$,
where $\xi$ is a root of unity. 
Prove that the automorphism that
maps $\alpha$ to $\alpha\xi^i$, also maps
$\alpha\xi^q$ to $\alpha\xi^{q+i}$.
\end{ukazanie}

\begin{zadacha}[*]
  Take $n\in \N$.  Let for any $k>1$ dividing $n$, $a\in \Q$ does not
  equal $k$-th power of any rational number, and $[K:\Q]$ be the
  splitting field of the polynomial $t^n -a$.  Prove that $K$ contains
  all $n$-th roots of unity and that $\Gal([K:\Q])$ is isomorphic to a
  semi-direct product $\Z/n\Z \rtimes \Aut(\Z/n\Z)$.
\end{zadacha}

\begin{zadacha}[*]
Let $k$ be a field of characteristic
 0, and let $[K:k]$ be a splitting
 field of the polynomial $t^n -a$.
Prove that the Galois group
$\Gal([K:k])$ is solvable.
\end{zadacha}

\begin{ukazanie}
If $k$ contains the  $n$-th roots of
unity then there is nothing to
prove. 
Suppose not, then prove that  $K$
contains the $n$-th roots. Consider
an intermediate extension $K'$
generated by these roots over $k$
and prove that $[K:K']$ and
$[K':k]$ are Galois extensios with
Abelian Galois groups.
\end{ukazanie}

\begin{zadacha} 
Let $[K:k]$ be a cyclic extension of
order $n$, and let $\nu$ be a
primitive element of the group
$\Gal[K:k]$, $\xi\in k$ be the
primitive roots of unity of degree $n$, 
and $\alpha\in K$ is a primitive
element of the extension.
Consider the {\bf Lagrange's resolvent}
\[
L = a + \xi^{-1} \nu(a) + \xi^{-2} \nu^2(a) + \dots + \xi^{-n+1} \nu^{n-1}(a)
\]
Prove that $\nu(L)= \xi L$. Prove that $L\neq 0$.
\end{zadacha}

\begin{zadacha}[*]
Prove that $\prod_{i=0}^{n-1}(t-\nu^i(L))= t^n-L^n$.
Prove that $L$ generates  $K$ over
$k$ and that $L^n\in k$.
\end{zadacha}

\begin{ukazanie}
To see that $L$ generates  $K$  over $k$,
use the fact that $\Gal[k[\sqrt[n]{L^n}],k]=\Z/n\Z$,
and therefore the dimension of $k[L]$ over $k$
is the same as dimension of $K$ over $k$.
\end{ukazanie}

\begin{zadacha}[*]
Let $[K:k]$ be a Galois extension of
order $n$, and let  $k$ contain all
the  $n$-th roots of unity.
Prove that $[K:k]$ is cyclic if and
only if it is generated by an
$n$-th root of $a\in k$.
\end{zadacha}

\begin{zadacha}[*]
(Galois theorem)
Deduce the following theorem.
A Galois extension  $[K:k]$ is
obtained by successive addition of
solutions of equations of the form
$t^n -a$ if and only if the group
$\Gal [K:k]$ is solvable.
\end{zadacha}

\begin{zamechanie}
Let $P(t)\in k[t]$ be a polynomial.
The {\bf Galois group} of $P$ is
defined to be the Galois group its
splitting field. Galois theorem
states that  $P(t)=0$ is solvable by
radicals if and only if the Galois
group of  $P(t)$ is solvable.
\end{zamechanie}

\begin{opredelenie}
Let group $G$ act on a set $\Sigma$.
The action is called  {\bf
  transitive} if any  $x\in \Sigma$
can be mapped to any 
$y\in \Sigma$ by an action of some $g\in G$.
\end{opredelenie}

\begin{zadacha} 
Let $G\subset S_n$ be a subgroup
that contains a transposition and
that acts transitively  on $\{1, 2, 3, \dots, n\}$.
Prove that $G= S_n$.
\end{zadacha}

\begin{zadacha} 
Let $P\in k[t]$ be an irreducible polynomial,
and let $\xi_1, \dots, \xi_n$ be its
roots and let all these roots be distinct. 
Prove that the Galois group of $P$ acts
on $\{\xi_1, \dots, \xi_n\}$
transitively. 
\end{zadacha}

\begin{ukazanie}
  Consider a decomposition of $\{\xi_1, \dots, \xi_n\}$ into
  equivalence classes under the action of $\Gal(P)$. Let $S$ be one of
  these equivalence classes.  Prove that the polynomial
  $\prod_{\xi_i\in S}(t-\xi_i)$ has coefficients in $k$ and divides
  $P$.
\end{ukazanie}

\begin{zadacha}[!]
  Let $P\in \Q[t]$ be an irreducible polynomial of degree $n$ that has
  exactly $n-2$ real roots.  Prove that its Galois group is $S_n$.
\end{zadacha}

\begin{ukazanie} 
Prove that $\Gal(P)$ acts
transitively on the roots of  $P$,
and that the complex conjugation
preserves the splitting field of 
$P$ and acts on the set of roots as
a transposition. 
\end{ukazanie}

\begin{zadacha}[!]
(Eisenstein theorem)
Let $Q= t^n + t^{n-1} a_{n-1} +
t^{n-2} a_{n-2} + \dots + a_0$ be a
polynomial with integer coefficients
such that all  $a_i$ divide a given
prime numeber $p$, and �
$a_0\not\vdots p^2$. Prove that $Q$
is irreducible over  $\Q$.
\end{zadacha}

\begin{zadacha}[*]
Prove that  $Q(t) = x^5 - 10 x +5$
is an irreducible (over $\Q$)
polynomial which has exactly 3 real roots.
Deduce that its Galois group is $S_5$.
\end{zadacha}

\begin{zadacha}[*]
Prove that the equation $x^5 - 10 x +5=0$ 
is not solvable by radical.
\end{zadacha}

\end{document}


% hey, Emacs, this is koi8-r file!
% Local Variables: 
% coding: koi8-r
% End: 
