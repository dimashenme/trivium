\documentclass[12pt]{article}

\usepackage{theorem,amsmath,amssymb}



\addtolength{\topmargin}{-23mm}
\addtolength{\textheight}{60mm}
\addtolength{\oddsidemargin}{-20mm}
\addtolength{\textwidth}{40mm}

\def\eqref#1{(\ref{#1})}
\newcommand{\goth}{\mathfrak}
\newcommand{\arrow}{{\:\longrightarrow\:}}
\def\1{\sqrt{-1}\:}
\newcommand{\restrict}[1]{{\left|_{{\phantom{|}\!\!}_{#1}}\right.}}

\renewcommand{\bar}{\overline}
\renewcommand{\phi}{\varphi}
\renewcommand{\epsilon}{\varepsilon}
\renewcommand{\geq}{\geqslant}
\renewcommand{\leq}{\leqslant}

\def\rad{\operatorname{\sf rad}}
\def\tr{\operatorname{\sf tr}}
\def\rk{\operatorname{\sf rk}}
\def\Alt{\operatorname{\sf Alt}}
\def\Sym{\operatorname{\sf Sym}}
\def\Id{\operatorname{\sf Id}}
\def\Hom{\operatorname{Hom}}
\def\Map{\operatorname{Map}}
\def\Gal{\operatorname{Gal}}
\def\Aut{\operatorname{Aut}}
\newcommand{\End}{\operatorname{End}}
\newcommand{\Mat}{\operatorname{Mat}}

\newcommand{\coker}{\operatorname{Coker}}

\def\chpoly{\operatorname{\sf Chpoly}}
\def\minpoly{\operatorname{\sf Minpoly}}

\def\cchar{\operatorname{\sf char}}

\def\Z{{\mathbb Z}}
\def\R{{\mathbb R}}
\def\C{{\mathbb C}}
\def\Q{{\mathbb Q}}
\def\N{{\mathbb N}}
\def\F{{\mathbb F}}

\def\Re{\operatorname{Re}}
\def\Im{\operatorname{Im}}

\makeatletter
\theoremstyle{definition}

\newtheorem{zadacha}{������}[section]
\newtheorem{opredelenie}{�����������}[section]
\newtheorem*{ukazanie}{��������}%[section]
\newtheorem*{zamechanie}{���������}%[section]

%\renewcommand{\labelenumi}{\ralph{enumi}.}
\renewcommand{\labelenumi}{\alph{enumi}.}
\newcommand{\subs}[1]{{\bigskip\centerline{\bf\large #1}\bigskip}}
\newcommand{\sttr}{{\bf(*)}}
\newcommand{\shrk}{{\bf(!)}}
\newcommand{\doublesttr}{{\bf(**)}}

\newcommand{\listok}[2]{%
\setcounter{page}{1}
\renewcommand{\@oddhead}{\hfil #2 \hfil}
\renewcommand{\@evenhead}{\hfil #2 \hfil}
\section*{#2}
\refstepcounter{section}
\setcounter{section}{#1}
}

\@addtoreset{equation}{section}
\renewcommand{\theequation}{\thesection.\arabic{equation}}

\let\oldllim=\lim
\def\lim{\oldllim\limits}
\makeatother


\begin{document}

\listok{3}{ALGEBRA 3: vector spaces and linear mappings}

\subs{Vector spaces}

Recall that abelian (or commutative) group is a group where group
operation is commutative:
\[
f \cdot g = g\cdot f
\]
Group operation in abelian groups is often denoted by $+$ and called
``addition''; unity is denoted by $0$ in this case and is called
``zero''.

\begin{opredelenie}
{\bf Linear} or {\bf vector} space $V$ over field $k$ is an abelian
group with an operation $k\times V \mapsto V$
(``multiplication of vector by element of a field'') which is in
accordance with the group operation in the following sense:
\begin{enumerate}
\item For any $\lambda\in k$, $u, v\in V$, $\lambda(u+v) =
\lambda u+ \lambda v$. For any $\lambda_1, \lambda_2\in k$, $u\in
V$,  $(\lambda_1+\lambda_2) u = \lambda_1 u +\lambda_2 u$
(distributivity of multiplication with respect to addition).
\item For any $\lambda_1, \lambda_2\in k$, $u\in V$, 
$\lambda_1(\lambda_2 u) = (\lambda_1 \lambda_2) u$ (associativity of multiplication).
\item For any $v\in V$, $1v=v$ where $1\in k$ is a unity.
\end{enumerate}
Elements of vector space are called vectors and group operation is
called the addition of vectors.
\end{opredelenie}

\begin{zadacha} Consider a field $k$. Prove that $k$ is a vector space
  over itself.
\end{zadacha}

\begin{zadacha} Prove that the group $k^*$ of invertible elements in
  $k$ with multiplication as group operation acts on any vector space
  over $k$.
\end{zadacha}

\begin{zamechanie} This group is called the multiplicative group of
  field $k$.
\end{zamechanie}

\begin{zadacha} Prove that a group of parallel transports of a plane
  has a structure of vector space over $\R$.
\end{zadacha}

\begin{zadacha} Consider a vector space $V$ over $k$. Prove that $0_k
  (v) =0_V$ for any $v\in V$. Here $0_k$ is the zero in $k$ and $0_V$
  is the unity in $V$.
\end{zadacha}

\begin{zadacha} Consider a field $�$ and its subfield $k$. Prove that 
$�$ is a vector space over $k$.
\end{zadacha}

\begin{zadacha} Consider a field $k$. 
\begin{enumerate}
\item Denote a set of $n$-tuples $(a_1, a_2, \dots , a_n)$ of elements
  of $k$ by $k^n$. Define a natural addition on $k^n$ and action of 
   $k^*$ on it and prove that you obtained a vector space.
\item Consider a set $S$. Denote by $k[S]$ a set of all collections of
  elements of $k$ 
\[
\langle a_{s_1},a_{s_2},\dots \rangle
\] 
each element of a collection corresponding to precisely one element of
$S$ such that $a_s$ are zero except a finite number of them. Introduce
a structure of vector space over $k$ on $k[S]$.
\end{enumerate}
\end{zadacha}

Vector space $k[S]$ is called a vector space, {\bf generated} by a set
$S$. Set $S$ can be naturally embedded into $k[S]$ -- every element of
$s \in S$ corresponds to a vector $[s] \in k[S]$ such that all
$a_{s'}$ are zeros except $a_s$ which is $1$.

\begin{opredelenie} Let $A$, $B$ be two sets and let $G$ act on them.
  One says that a mapping $\kappa:\; A \to B$ {\bf is compatible
    with action $G$} if $\kappa(g(a)) = g(\kappa(a))$.
\end{opredelenie}

Recall that a homomorphism of abelian groups is a mapping that
preserves the group operation
\begin{equation}\label{_homomo_Equation_}
  f:\; G_1 \to G_2, \ \  f(g+ g') = f(g) + f(g')
\end{equation}

\begin{opredelenie}
Homomorphism of vector spaces over $k$ is a mapping which preserves
addition of vectors (cf. \eqref{_homomo_Equation_}) and is compatible
with the action of $k^*$.
\end{opredelenie}

In other words a homomorphism of vector spaces is a mapping $f:\; V_1
\to V_2$ which satisfies conditions $f(v_1 + v_2) = f(v_1) + f(v_2)$,
$f(\lambda v) = \lambda f(v)$. Monomorphism, epimorphism, isomorphism
and automorphism of vector spaces are defined in the same fashion as
for groups (as well as rings, fields and all other algebraic
structures). When talking about vector spaces one says ``linear
operator'' or ``linear mapping'' instead of ``homomorphism''.

Recall that identity mapping of a vector space $V$ is denoted by
$\Id_V$. $\Id_V$ is obviously an automorphism.

\begin{zadacha} Prove that a linear mapping $\phi:\; V_1 \to
V_2$ is bijective if and only if it is {\bf invertible},
i.e. there exists a linear mapping $\psi:\; V_2 \to V_1$
such that
\begin{equation}
\psi \circ \phi = \Id_{V_1}, 
\phi \circ \psi = \Id_{V_2}.
\end{equation} 
Are these conditions sufficient each on its own ?
\end{zadacha}

\begin{zadacha}\label{iso.equiv} 
The fact that two vector spaces $V$, $V'$ are isomorphic is denoted by
$V\cong V'$. Prove that 
\begin{enumerate}
\item If $V \cong V'$ and $V' \cong V''$ then $V \cong V''$.
\item If $V \cong V'$ then $V' \cong V$.
\item It is always true that $V \cong V$.
\end{enumerate}
\end{zadacha}

\begin{zadacha} Prove that the set of homomorphisms from $V_1$ to
  $V_2$ form a vector space (it is often denoted by $\Hom(V_1, V_2)$).
\end{zadacha}

\begin{zadacha} Prove that the set of automorphisms $V$ forms a group
  (this group is often denoted by $GL(V)$). Is this a commutative group?
\end{zadacha}

\begin{opredelenie} Subgroup $V' \subset V$ of a vector space $V$ is
  called a {\bf vector subspace} or {\bf linear subspace of $V$} if it
  is preserved under the action of $k^*$ (in other words for any
  $\lambda\in k$, $v, v'\in V'$ it is true that $\lambda(v) \in V'$,
  $v+ v'\in V'$).
\end{opredelenie}

Vector subspace is itself a vector space over the same field.

\begin{zadacha} Consider a set $S$. Prove that a set of all mappings
  $\Map(S,k)$ from $S$ to $k$ is a vector space.
\end{zadacha}

\begin{zadacha}[*]\label{univ.gene}
  Consider a vector space $W$ and a set $S$. Prove that any mapping $S
  \to W$ can be extended in a unique way to a linear mapping $k[S] \to
  W$. Is it true if we allow for the infinite non-zero elements of a
  field in the definition of $k[S]$?
\end{zadacha}

\begin{zadacha} Let $k[t]$ be a set of polynomials 
$a_n t^n + a_{n-1} t^{n-1} + ... + a_0$ with coefficients belonging to
a field $k$. Prove that this is a linear space.
\end{zadacha}

\begin{zadacha} Consider a set $\Map(k,k)$ of all mappings from the
  field $k$ to $k$. For any polynomial $P=a_n t^n + a_{n-1} t^{n-1} +
... + a_0$ and every $\lambda \in k$ define $\Psi_P(\lambda) =
P(\lambda)$. We obtain the mapping $\Psi:\; k[t]\to \Map(k,k)$, $P
\mapsto \Phi_P$. Prove that it is a homomorphism.
\end{zadacha}

\begin{zadacha}[*] Let $k$ be a finite field. Prove that
$\Psi:\; k[t]\to \Map(k,k)$ is not a monomorphism.  Prove that it is
an epimorphism.
\end{zadacha}

\begin{zadacha}[*] Let $k$ be an infinite field. Prove that  $\Psi:\;
  k[t]\to \Map(k,k)$ is a monomorphism.
\end{zadacha}

\begin{zadacha}[*] Prove that it is not an epimorphism.
\end{zadacha}

Consider a set $A$ and a binary relation $\sim$ on
$A$ (that is, we state for certain pairs of elements $a,b \in A$ that
$a \sim b$). One says that  $\sim$ is an {\bf equivalence relation}
if the following holds:
\begin{enumerate}
\item For any element  $a \in A$ it is true that $a \sim a$.
\item If  $a \sim b$ and $b \sim c$ then $a \sim c$ (transitivity).
\item If $a \sim b$ then  $b \sim a$ (symmetry).
\end{enumerate}

An equivalence relation $\sim$ having been defined an {\bf equivalence
  class} of an element $a \in A$ is a set of all elements $a' \in A$
such that $a' \sim a$. It is easy to check that if $a \sim a'$ then
the equivalence class of $a$ is the same as the equivalence class of
$a'$. So we can talk just about an equivalence class without
mentioning a particular element $a$. A relation $a = b \mod n$ is an
example of an equivalence relation of the set of natural numbers. The
problem~\ref{iso.equiv} can be reformulated as follows: ``$V \cong V'$
is an equivalence relation''.

\begin{zamechanie}
We work with equivalence relation in Geometry 1 when considering Cauchy
sequences, although we do not introduce this notion explicitly.
\end{zamechanie}

\begin{zadacha}
  Let $V$ be a vector space and $V'\subset V$ be a subspace. Consider
  the following equivalence relation on $V$: $a \sim b$ iff $a+v = b$
  for some $v\in V'$. Prove that the set of equivalence classes forms
  linear space.
\end{zadacha}

\begin{opredelenie} 
This space is called a {\bf quotient space} and is denoted by $V/V'$. 
\end{opredelenie}

\begin{zadacha} Consider a natural mapping $V \to V/V'$ which maps an
  element to its equivalence class. Prove that this is a homomorphism
  and an epimorphism.
\end{zadacha}

\begin{zadacha} Let $\phi:\; V_1 \to V_2$ be a linear mapping and 
$V_0\subset V_1$ be a set of elements that are mapped to zero. 
\begin{enumerate}
\item Prove that $V_0$ is a linear subspace in $V_1$. 
\item Prove that an image of $\phi$ -- that is, a subset of all $v \in
  V_2$ of the form $\phi(v')$, $v' \in V_1$ -- is a linear subspace in
  $V_2$.
\end{enumerate}
\end{zadacha}

\begin{opredelenie}
$V_0$ is called a {\bf kernel of $\phi$}.
\end{opredelenie}

\begin{zadacha} Let $\phi:\; V_1 \to V_2$ be a linear operator.
Prove that an image of $\phi$ is isomorphic to a quotient space
$V_1/V_0$ where $V_0$ is a kernel of $\phi$.
\end{zadacha}

%%%%%%%%%%%%%%%%%%%%%%%%%%%%%%%%%%%%%%%%%%%%%%%%%%%%%%%%%%%%
\subs{Linear hull, basis, dimension}
%%%%%%%%%%%%%%%%%%%%%%%%%%%%%%%%%%%%%%%%%%%%%%%%%%%%%%%%%%%%

\begin{zadacha}
  Let $V$ be a vector space over the field $k$ and $x_1, \dots, x_n$
  be vectors from $V$. Any vector of the form $v = \lambda_1 x_1+
  \lambda_2 x_2 + \dots + \lambda_m x_n$ is called {\bf linear
    combination} of vectors $x_1, \dots, x_n$ where $\lambda_1$ are
  arbitrary elements from $k$. Prove that linear combinations of
  vectors $x_1, \dots, x_n$ form a linear subspace of $V$.
\end{zadacha}

\begin{opredelenie} This subspace is called a {\bf linear hull} of
  $x_1, \dots, x_n$ and is denoted $\langle x_1, x_2, \dots,
  x_n\rangle$. $\langle x_1, x_2, \dots, x_n\rangle$ is called a
  subspace {\bf generated} by vectors $x_1\dots,x_n$.
\end{opredelenie}

\begin{zadacha}\label{epi} Construct an epimorphism from $k^n$ into 
$\langle x_1, x_2, \dots, x_n\rangle$.
\end{zadacha}

\begin{ukazanie} Map an $n$-tuple
$(0, 0, 0, \dots, 1, \dots, 0)\in k^n$ (unity is in the $l$-th
position) to $x_l$.
\end{ukazanie}

\begin{opredelenie} Vectors $x_1, \dots, x_n$ are called {\bf linear
    independent} vectors, if for any linear combination $v = \lambda_1
  x_1+ \lambda_2 x_2 + \dots + \lambda_m x_n$ such that there is at
  least one $\lambda_i\neq 0$, it is true that $v\neq 0$
\end{opredelenie}

\begin{zadacha} Let $x_1, \dots, x_n$ be vectors from vector space
$V$ and $\phi$ be an epimorphism constructed in the
exercise~\ref{epi}. Prove that $\phi$ is injective iff  $x_1, \dots,
x_n$ are linearly independent vectors.
\end{zadacha}

\begin{opredelenie} 
  Let $x_1, \dots, x_n$ be linearly independent vectors from a vector
  space $V$ such that $V = \langle x_1, x_2, \dots, x_n\rangle$. $x_1,
  \dots, x_n$ is called a {\bf basis of} the vector space $V$.
\end{opredelenie}

\begin{zadacha} Construct a basis of $k^n$.
\end{zadacha}

\begin{zadacha} Prove that a vector space with a basis $x_1, \dots,
  x_n$ is isomorphic to $k^n$.
\end{zadacha}

\begin{zadacha} Let $v$ be a non-zero vector from $V \cong k^n$,
$\langle v\rangle$ be a subspace generated by it and $V/\langle
v\rangle$ is a quotient space. Prove that $V/\langle v\rangle$
is isomorphic $k^{n-1}$.
\end{zadacha}

\begin{ukazanie} Consider a subspace $V_l\subset V \cong k^n$,
generated by $n$-tuples of the form
\[
\langle \lambda_1, \lambda_2, \dots,
\lambda_{l-1}, 0, \lambda_{l+1}, \dots, \lambda_n\rangle
\]
(there is a $0$ on the $l$-th position). This space is isomorphic to
$k^{n-1}$. Prove that for some $l= 1, 2, \dots, n$ the natural
projection $V_l \to V/\langle v\rangle$ is an isomorphism.
\end{ukazanie}

\begin{zadacha} Let $x_1, \dots, x_l$ be linearly independent vectors 
  from $V \cong k^n$. Prove that $V /\langle x_1, x_2, \dots,
  x_l\rangle$ is isomorphic to $k^{n-l}$
\end{zadacha}

\begin{ukazanie} Use an inductive argument.
\end{ukazanie}

\begin{zadacha} Let $V_1\subset V_2$ be a subspace of $V_2 \cong k^n$.
  Suppose that $V_1 \cong k^m$. Prove that $m\leq n$.
\end{zadacha}

\begin{zadacha} Let $x_1, \dots, x_l$ be a basis of $V \cong k^n$.
Prove that $l=n$.
\end{zadacha}

\begin{zadacha}[!] 
Let vector spaces $k^l$ and $k^m$ be isomorphic. Prove that
$l=m$.
\end{zadacha}

\begin{opredelenie} Vector space $V$ over $k$ is called a 
{\bf finite-dimensional} if it is isomorphic to $k^n$. Number $n$ is
called a {\bf dimension} of $V$. This is denoted by 
$\dim V=n$. It follows from the previous exercise that $n$ is uniquely
defined. 
\end{opredelenie}

\begin{zadacha} Let $x_1, x_2, \dots, x_l$ be linearly independent
  vectors in a linear space $V$ such that $V':= V /\langle x_1, x_2,
\dots, x_l\rangle$ is a non-zero space. Let $x_{l+1}\in V$ be a
vector such that its natural projection on $V'$ is non-zero. Prove
that  $x_1, x_2, \dots, x_l, x_{l+1}$ are linearly independent vector. 
\end{zadacha}

\begin{zadacha}[!] Let $V$ be a linear space which is not
  finite-dimensional. Prove that there exists an infinite sequence of
  linearly independent vectors $x_1, x_2, \ldots, x_l, \ldots \in V$. 
\end{zadacha}

\begin{ukazanie} Use the previous exercise.
\end{ukazanie}

\begin{zadacha}[!] Let $V\subset V'$ be a subspace of a finite
  dimensional vector space. Prove that $V$ is finite-dimensional.
  Deduce that $\dim V \leq \dim V'$
\end{zadacha}

\begin{ukazanie} 
Use the previous exercise.
\end{ukazanie}

\begin{zadacha} Let $V =V'/V''$ be a quotient space of finite
  dimensional space. Prove that $V'$ is finite-dimensional. Prove that
  $\dim V \leq \dim V'$. Deduce that $\dim V' = \dim V + \dim V''$.
\end{zadacha}

\begin{ukazanie} Use a proof by contradiction:
consider an infinite sequence of linearly independent vectors from $V$
and lift them to $V'$.
\end{ukazanie}

\begin{zadacha}\label{mono=>iso} 
  Let $f:\; V \to V'$ be homomorphism of vector spaces of the same
  dimension $n$. Suppose that $f$ is an injection or surjection. Prove
  that $f$ is an isomorphism.
\end{zadacha}

%%%%%%%%%%%%%%%%%%%%%%%%%%%%%%%%%%%%%%%%%%%%%%%%%%%%%%%%%%%%
\subs{Linear forms, bilinear forms}
%%%%%%%%%%%%%%%%%%%%%%%%%%%%%%%%%%%%%%%%%%%%%%%%%%%%%%%%%%%%

\begin{opredelenie}
  Let $V$ be a linear space over $k$. A {\bf linear form} or {\bf
    bilinear form} over $V$ is a homomorphism of linear spaces $V$ and
  $k$. The space of linear forms over $V$ is denoted by $V^*$.
\end{opredelenie}

\begin{zadacha} 
Consider a finite-dimensional linear space $V$. Prove that $\dim
V\!=\dim V^*$.
\end{zadacha}

\begin{zadacha}
Let $k$ be a field and $S$ be an arbitrary set. Is it true that $(k[S])^*
\cong \Map(S,k)$?
\end{zadacha}

\begin{zadacha}[!] 
Consider a natural map  $V\overset{ev}{\to} V^{**}$, $v
\mapsto (\lambda \mapsto \lambda(v))$, the vector $v\in V$ maps a form
$\lambda \mapsto \lambda(v)$ to $V^*$. Let $V$ be finite
dimensional. Prove that  $V\overset{ev}{\to} V^{**}$ is an isomorphism.
\end{zadacha}

\begin{ukazanie} 
Prove that this is an injection and use the exercise~\ref{mono=>iso}.
\end{ukazanie}

\begin{zadacha}[**] Consider a infinite-dimensional linear space $V$.
Prove that $V\overset{ev}{\to} V^{**}$ is not an isomorphism.
\end{zadacha}

\begin{opredelenie}
  Let $U$, $V$, $W$ be linear spaces over a field $k$. A mapping
  $U\times V \overset{\mu}{\to} W$, $u, v \mapsto \mu(u, v)$ is called
  {\bf bilinear} if for every $u$ the mappings $\mu(u, \cdot):\; V \to
  W$ and $\mu(\cdot, u):\; U \to W$ are linear.
\end{opredelenie}

\begin{zadacha} Prove that a sum of bilinear mappings is bilinear.
  Prove that a structure of vector space can be defined on the set of
  bilinear mappings $U\times V \to W$.
\end{zadacha}

A space of bilinear mappings is denoted $\Hom(U\otimes V, W)$. The
reason for that is the following:

\begin{zadacha}[*]
  Consider vector spaces $U$ and $V$. Consider the set $U \times V$
  and the vector space generated by it $k[U \times V]$. Let us denote
  the element of $k[U \times V]$ corresponding to the pair $\langle
  u,v \rangle \in U \times $ by $u \otimes v$. Consider the subspace
  generated by the vectors of the form $au \otimes v - a(u \otimes
  v)$, $u \otimes av - a(u \otimes v)$, $(u_1+u_2) \otimes v - u_1
  \otimes v - u_2 \otimes v$, $u \otimes (v_1 + v_2) - u \otimes v_1 -
  u \otimes v_2$ and denote the quotient space by $U \otimes V$.  Prove
  that for any $W$ the subspace $\Hom(U \otimes V,W)$ is isomorphic to
  the set of bilinear mappings from $U \times V$ to $W$.
\end{zadacha}

The space $U \otimes V$ is called a {\bf tensor product}
of spaces $U$ and $V$.

\begin{zadacha}[*] The dimensions of $U$, $V$, $W$ are $a,b,c$.
Find the dimension of $\Hom(U\otimes V, W)$.
\end{zadacha}

\begin{opredelenie} Let $V$ be a vector space over $k$.  Bilinear form
  over $V$ is a bilinear mapping $V\times V \overset{\mu}{\to} k$. A
  bilinear symmetric form is a form that satisfies the equality
  $\mu(x,y) = \mu(y,x)$. Bilinear antisymmetric form is a form is
  a form that satisfies the equality $\mu(x,y) = -\mu(y,x)$. We will
  denote the space of bilinear symmetric forms by $S^2 V^*$, the space of
  bilinear 
  antisymmetric forms by $\Lambda^2 V^*$ and the space of all
  bilinear forms by $(V\otimes V)^*$.
\end{opredelenie}

\begin{opredelenie} One says that the characteristic of a field $k$
  is not 2 if the number $2=1+1$ is invertible in $k$.
\end{opredelenie}

\begin{zamechanie} 
Apparently, this is not true in a field of two elements.
\end{zamechanie}

\medskip

\noindent
Up to the end of this section we suppose that the characteristic of a
field we are talking about is not 2.

\medskip

\begin{opredelenie}
If $U$, $V$ are vector spaces then the product $U\times V$
of sets $U$ and $V$ is endowed with a natural structure of vector
space. This product considered as a vector space is called a {\bf direct
  sum of $U$ and $V$} and is denoted $U \oplus V$.
\end{opredelenie}

\begin{zadacha}
Consider two subspaces $U, V$ of a vector space $W$ such that
intersection of $U$ and $W$ contains only $0 \in V$ and the linear hull
over $U$ and $V$ is $W$. Prove that  $W$ is isomorphic to
$U \oplus V$. 
\end{zadacha}

\begin{zamechanie}
Notation that is used in that case: $W=U \oplus V$.
\end{zamechanie}

\begin{zadacha} Consider the symmetrization mapping that turns any
  bilinear form into symmetric form $\Sym(\mu)(x,y) =
\frac{1}{2}(\mu(x,y) +\mu(y,x))$ and alternation mapping 
$\Alt(\mu)(x,y) = \frac{1}{2}(\mu(x,y) -\mu(y,x))$. Prove that these
mappings are linear operators
\[ 
(V\otimes V)^*\overset{\Sym}{\to} S^2 V^*, \ \ (V\otimes
V)^*\overset{\Alt}{\to} \Lambda^2 V^*
\]
Prove that the sum
\[
\Sym \oplus \Alt:\; (V\otimes V)^*\to S^2 V^*\oplus \Lambda^2 V^*
\]
is an isomorphism.
\end{zadacha}

\begin{zadacha}[*] 
Let $\dim V = n$. Find dimension of $S^2 V^*$ and $\Lambda^2
V^*$.
\end{zadacha}

\begin{zadacha} Let $\mu$ be a bilinear symmetric form. Prove that
  $\mu(u+v, u+v) = \mu(v,v) + \mu(u,u) + 2 \mu(u, v)$.
\end{zadacha}

\begin{zadacha}[!] 
  Let $\mu$ be a non-zero bilinear symmetric form over $V$. Prove that
  $\mu(x,x)\neq 0$ for some $x \in V$.
\end{zadacha}

\begin{ukazanie} Use the previous exercise.
\end{ukazanie}

\begin{opredelenie}
Consider $V$ a linear space with a symmetric or antisymmetric bilinear
form  $\mu:\; V \times V \to k$ defined on it. For any  $v\in V$,
$\mu$ defines a linear form $\mu(v,
\cdot):\; V \to k$.  We say that $v$ belongs to the radical {\bf
  radical} of $\mu$ if this form equals zero.
\end{opredelenie}

\begin{zadacha}[*] 
Prove that a radical is a linear subspace of $V$.
\end{zadacha}

Radical is denoted by $\rad \mu$.

\begin{zadacha}[*] 
Prove that $\mu(v+r, v'+r')= \mu(v, v')$ where $r, r'\in \rad \mu$.
\end{zadacha}

\begin{zamechanie} It follows that  $\mu$ is naturally defined on 
the quotient space $V/\!\rad \mu$.
\end{zamechanie}

\begin{opredelenie} A symmetric (or antisymmetric) form $\mu$ is
  called {\bf non-degenerate} if its radical is zero.  Non-degenerate
  bilinear antisymmetric form is called {\bf symplectic}.
\end{opredelenie}

\begin{zadacha} Consider a non-degenerate symmetric (or antisymmetric)
  bilinear form $\mu$ defined on a finite-dimensional vector space
  $V$. Define the mapping  $V \to V^*$, $v \mapsto
\mu(v, \cdot)$ that maps  $v$ to the form  $t\mapsto \mu(v, t)$.
Prove that it is an isomorphism.
\end{zadacha}

\begin{ukazanie} Prove that it is a monomorphism of spaces of the same
  dimension.
\end{ukazanie}

\begin{zadacha} Let $\mu$ be non-degenerate symmetric (or
  antisymmetric) bilinear form on the finite-dimensional space
$V$ and $\lambda:\; V \to k$ be a linear functional. Prove that there exists
a vector $v\in V$ such that $\lambda(t) = \mu(v, t)$.
\end{zadacha}

\begin{ukazanie} Use the previous exercise.
\end{ukazanie}

\begin{opredelenie} Let $V$ be a space with a symmetric (or
  antisymmetric) bilinear form $\mu$ and be $V_1\subset V$ its linear
  subspace. Define an {\bf orthogonal complement} $V_1^\bot$ to be a
  set of all vectors $v\in V$ such that $\mu(v, v_1)=0$ for all
  $v_1\in V_1$.
\end{opredelenie}

\begin{zadacha}[!] 
  Let $\mu$ be non-degenerate on $V$ and on $V_1$. Suppose that $V_1$
  is finite-dimensional. Then $V = V_1 \oplus V_1^\bot$.
\end{zadacha}

\begin{ukazanie} That $V_1$ and $V_1^\bot$ do not intersect can be
  shown explicitly. It remains to prove that every vector $v\in V$ can
  be represented as the sum of vectors from $V_1$ and $V_1^\bot$.
  Consider $\mu(v, \cdot)$ as a functional over $V_1$. Use the
  previous exercise and find the vector $v_1\in V_1$ such that a form
  $\mu(v-v_1, \cdot)$ is zero on $V_1$. It follows that $v-v_1 \in
  V_1^\bot$.
\end{ukazanie}

\begin{zadacha}[!] Deduce the following statement from the previous
  exercise.  Let $\mu$ be a symmetric bilinear non-degenerate form on
  a vector space $V$. Then there exists a basis $x_1, \dots, x_n$ in
  $V$ such that $\mu(x_i, x_j)=0$ for all $i\neq j$ and $\mu(x_i,
  x_i)\neq 0$ for all $i$.
\end{zadacha}

\begin{ukazanie} Find a vector $x$ such that $\mu(x,x)\neq 0$. Use the
  decomposition $V = \langle x \rangle
\oplus \langle x \rangle^\bot$ from the previous exercise and apply an
inductive argument.
\end{ukazanie}

\begin{zadacha}[*] Let $\mu$ be a bilinear symmetric form on $V$ Then
  there exists a basis $x_1, \dots, x_n$ in $V$ such that $\mu(x_i,
  x_j)=0$ for all $i\neq j$. Such basis is called an {\bf orthogonal
    basis}.
\end{zadacha}

\begin{zadacha}[*] Let $\mu$ be a symplectic form 
on the space $V$. Prove that $V$ dimensions is even. Prove a basis
$x_1, ..., x_{2n}$ in $V$ such that 
\[
\mu(x_{2r-1}, x_{2r}) = - \mu(x_{2r}, x_{2r-1}) =1
\] 
when $r = 1, 2, ..., n$ and $\mu(x_i, x_j)=0$ for all other pairs
$(i,j)$.
\end{zadacha}

\begin{ukazanie} 
The proof is analogous to the symmetric case .
\end{ukazanie}

\begin{opredelenie} Let $V$ be a vector space over $\R$ and $\mu$ be a
  bilinear symmetric form on it. A form $\mu$ is called {\bf
    positive} if  $\mu(x,x) >0$  for all non-zero vector $x$.
\end{opredelenie}

\begin{zadacha} Let $\mu$ be a positive bilinear form on $V$. Then
  there is a basis $x_1, \dots, x_n$ in $V$ such that $\mu(x_i,
  x_j)=0$ for all $i\neq j$ and $\mu(x_i, x_i)=1$ for all $i$.
\end{zadacha}

\begin{opredelenie} Such basis is called {\bf orthonormal}.
\end{opredelenie}

\begin{zadacha}[*] 
Let $x,y$ be arbitrary vectors in a space $V$  and $\mu$ be a positive
bilinear form. Prove the inequality
\[
\frac{\mu(x,x) + \mu(y,y)}{2} \geq \mu(x,y).
\]
\end{zadacha}

\begin{zadacha}[*] 
Prove the {\bf Cauchy inequality}:
\[
\sqrt{\mu(x,x)\mu(y,y)} \geq \mu(x,y).
\]
\end{zadacha}

\begin{zadacha}[*] 
Prove the {\bf triangle inequality}
\[ 
\sqrt{\mu(x,x)} + \sqrt{\mu(y,y)} \geq \sqrt{\mu(x+y,x+y)}.
\] 
\end{zadacha}

\end{document}
