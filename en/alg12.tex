\documentclass[12pt]{article}

\usepackage{theorem,amsmath,amssymb}



\addtolength{\topmargin}{-23mm}
\addtolength{\textheight}{60mm}
\addtolength{\oddsidemargin}{-20mm}
\addtolength{\textwidth}{40mm}

\def\eqref#1{(\ref{#1})}
\newcommand{\goth}{\mathfrak}
\newcommand{\arrow}{{\:\longrightarrow\:}}
\def\1{\sqrt{-1}\:}
\newcommand{\restrict}[1]{{\left|_{{\phantom{|}\!\!}_{#1}}\right.}}

\renewcommand{\bar}{\overline}
\renewcommand{\phi}{\varphi}
\renewcommand{\epsilon}{\varepsilon}
\renewcommand{\geq}{\geqslant}
\renewcommand{\leq}{\leqslant}

\def\rad{\operatorname{\sf rad}}
\def\tr{\operatorname{\sf tr}}
\def\rk{\operatorname{\sf rk}}
\def\Alt{\operatorname{\sf Alt}}
\def\Sym{\operatorname{\sf Sym}}
\def\Id{\operatorname{\sf Id}}
\def\Hom{\operatorname{Hom}}
\def\Map{\operatorname{Map}}
\def\Gal{\operatorname{Gal}}
\def\Aut{\operatorname{Aut}}
\newcommand{\End}{\operatorname{End}}
\newcommand{\Mat}{\operatorname{Mat}}

\newcommand{\coker}{\operatorname{Coker}}

\def\chpoly{\operatorname{\sf Chpoly}}
\def\minpoly{\operatorname{\sf Minpoly}}

\def\cchar{\operatorname{\sf char}}

\def\Z{{\mathbb Z}}
\def\R{{\mathbb R}}
\def\C{{\mathbb C}}
\def\Q{{\mathbb Q}}
\def\N{{\mathbb N}}
\def\F{{\mathbb F}}

\def\Re{\operatorname{Re}}
\def\Im{\operatorname{Im}}

\makeatletter
\theoremstyle{definition}

\newtheorem{zadacha}{������}[section]
\newtheorem{opredelenie}{�����������}[section]
\newtheorem*{ukazanie}{��������}%[section]
\newtheorem*{zamechanie}{���������}%[section]

%\renewcommand{\labelenumi}{\ralph{enumi}.}
\renewcommand{\labelenumi}{\alph{enumi}.}
\newcommand{\subs}[1]{{\bigskip\centerline{\bf\large #1}\bigskip}}
\newcommand{\sttr}{{\bf(*)}}
\newcommand{\shrk}{{\bf(!)}}
\newcommand{\doublesttr}{{\bf(**)}}

\newcommand{\listok}[2]{%
\setcounter{page}{1}
\renewcommand{\@oddhead}{\hfil #2 \hfil}
\renewcommand{\@evenhead}{\hfil #2 \hfil}
\section*{#2}
\refstepcounter{section}
\setcounter{section}{#1}
}

\@addtoreset{equation}{section}
\renewcommand{\theequation}{\thesection.\arabic{equation}}

\let\oldllim=\lim
\def\lim{\oldllim\limits}
\makeatother


\begin{document}


%%%%%%%%%%%%%%%%%%%%%%%%%%%%%%%%%%%%%%%%%%%%%%%%%%%%%%%%%%%%

\listok{12}{ALGEBRA 12: semisimple and nilponent operators}

%%%%%%%%%%%%%%%%%%%%%%%%%%%%%%%%%%%%%%%%%%%%%%%%%%%%%%%%%%%%

%%%%%%%%%%%%%%%%%%%%%%%%%%%%%%%%%%%%%%%%%%%%%%%%%%%%%%%%%%%%
\subs{Artinian algebras over an algebraically closed field}
%%%%%%%%%%%%%%%%%%%%%%%%%%%%%%%%%%%%%%%%%%%%%%%%%%%%%%%%%%%%

Let $R$ be an Artinian ring over a field $k$. 
Recall that in the exercise sheet 9 we have constructed a canonical
decomposition $R\cong \oplus_i e_i R_i$ where $e_i$ are indecomposable
orthogonal idempotents and $e_i R$ is Artinian with no non-unit
idemotents; moreover, this decomposition is unique.

\begin{zadacha}[!]
Assume that $R$  does not have non-unit idempotents
and $k$ is algebraically closed. Prove that if $R$ is semisimple then
$R=k$. 
\end{zadacha}

\begin{zadacha}[!]
Assume that  $R$ does not have non-unit idempotents, and $k$ is
algebraically closed. Prove that $R= k \oplus {\goth n}$ where
${\goth n}$ is a nilradical.
\end{zadacha}

\begin{ukazanie}
Prove that  $R/{\goth n}$ is semisiple and apply the previous
exercise. 
\end{ukazanie}

\begin{zadacha}[!]\label{_razlozhe_kolca_Zadacha_}
Let $R$ be an Artinian ring over an algebraically closed field 
 $k$. Prove that $R= R_{ss} \oplus {\goth n}$ where  $R_{ss}$ is a
 semisimple Artinian subring in  $R$. Prove that $R_{ss}\subset R$ is
uniquely defined.
\end{zadacha}

\begin{zadacha}[*]
Is this true if $k$ is not algebraically closed?
\end{zadacha}

We will further need the following statement.

\begin{zadacha}[!]\label{_Syurie_polupro_Zadacha_}
Let $R$ be a semi-simple Artinian ring over a field $k$,
and $R\arrow R'$ be a surjective homomorphism of $k$-algebras.
Prove that $R'$ is a semisimple Artinian ring too.
\end{zadacha}

\begin{ukazanie}
There is a similar problem in ALGEBRA 9.
\end{ukazanie}

\begin{opredelenie}
Let $R$ be an algebra over a field $k$. A {\bf representation} of an
algebra $R$ is a homomorphim of algebras from $R$ to $\End(V)$, where
$V$ is a vector space over $k$.
\end{opredelenie}

\begin{zadacha}
Let $R$  be an algebra over a field  $k$.
Consider a mapping $R\arrow \End(R)$, defined by the formula
$r \mapsto (v\arrow rv)$. Prove that this is a representation.
\end{zadacha}

\begin{zadacha} 
Let $R$ be an algebra over $k$, isomorphic to a finite extension $k$,
and let $V$ be a finite dimensional representation of  $R$. Prove that
$V\cong R^n$, that is,
$V$ is isomorphic (as a representation of $R$) to a sum of several
copies of  $R$.
\end{zadacha}

\begin{zadacha}
  Let $V$ be a finite dimensional representation of the quaternion
  algebra ${\mathbb H}$ over $\R$.  Prove that $V$ is isomorphic to
  ${\mathbb H}^n$.
\end{zadacha}

\begin{zadacha} 
Let $G$ be a group, and $k$ be a field. 
A {\bf group algebra} $G$ over $k$ (denoted $k[G]$)
is the vector space of linear combinations of the form $\sum \lambda_i
g_i$ ($\lambda_i \in k$, $g_i\in G$) with multiplication defined by
the formula
\[ 
(\sum \lambda_i g_i)(\sum \lambda'_j g'_j) = 
 \sum_{i,j}\lambda_i\lambda'_jg_ig'_j.
\]
Prove that this is indeed an algebra. Prove that 
any representation of a group $G$ can be uniquely extended to a
representation of the group algebra.
\end{zadacha}

\begin{zadacha}[!]
Let $G_1, G_2$ be groups and $k[G_1\times G_2]$ be the group algebra
of their product. Prove that $k[G_1\times G_2]\cong k[G_1]\otimes
k[G_2]$. 
\end{zadacha}

\begin{zadacha}[!]
Let $G = (\Z/2\Z)^n$ be a product of 
$n$ copies of $\Z/2\Z$. Prove that $k[G]\cong k^{\oplus_{2^n}}$
(direct sum of $2^n$ copies of $k$).
\end{zadacha}

\begin{ukazanie}
  Prove that $k[\Z/2\Z]\cong k\oplus k$, and use the isomorphism
  $k[G_1\times G_2]\cong k[G_1]\otimes k[G_2]$.
\end{ukazanie}

\begin{zadacha}[*]
Consider the Klein group (the subgroup of order 8 in the quaternions
that consists of elements $\{\pm 1, \pm I, \pm J, \pm K\}$). 
Prove that its group algebra over $\R$ is isomorphic to
${\mathbb H}\oplus \R \oplus \R \oplus \R \oplus \R$.
\end{zadacha}

\begin{zadacha}[*]
Let  $G$ be a finite Abelian group, and let $k$ be an algebraically
closed field of characteristic 0. Prove that $k[G]$ is a semisimple
Artinian ring over $k$. Deduce from this that $k[G]$ is a direct sum
of $|G|$ copies of $k$.
\end{zadacha}

\begin{ukazanie}
Use the criterion mentioned in ALGEBRA 9: an Artinian ring $R$ over a
field of characteristic 0 is semisimple if and only if the trace
defines a nondegenerate form on $R$.
\end{ukazanie}

\begin{zadacha}[*]
Let $G$ be a finite Abelian group,
$k$ be an algebraically closed field characteristic 0, and
 $\rho:\; G \arrow \End(V)$ be a representation of $G$ over $k$.
Prove that $V$ decomposes into a direct sum of one-dimensional
$G$-invariant subspaces.
\end{zadacha}

\begin{ukazanie} Use the previous exercise and the exercise
 \ref{_Syurie_polupro_Zadacha_}.
\end{ukazanie}

\begin{zadacha}[*]
  Let $G$ be a finite Abelian group, and $\R[G]$ its group ring over
  $\R$.  Prove that $\R[G]$ is isomorphic to a direct sum of several
  copies of $\R$ and $\C$.
\end{zadacha}

\begin{zadacha}[*]
Let $G$ be a finite Abelian group,
and $\rho:\; G \arrow \End(V)$ be a representation of $G$ over
$\R$. Prove that $V$ can be decomposed into a direct sum of
one-dimensional and two-dimensional 
$G$-invariant subspaces.
\end{zadacha}

\begin{zadacha}[!]
Let $G$ be a finite Abelian group,
and let $\rho:\; G \arrow \End(V)$ be its three-dimensional
representation over $\R$. Prove that there is a $G$-invariant line in
$V$. 
\end{zadacha}

%%%%%%%%%%%%%%%%%%%%%%%%%%%%%%%%%%%%%%%%%%%%%%%%
\subs{Semi-simple operators}
%%%%%%%%%%%%%%%%%%%%%%%%%%%%%%%%%%%%%%%%%%%%%%%%

Let $A \in \End(V)$ be a linear operator over a finite-dimensional
vector space. It is easy to see that the subalgebra 
$\langle 1, A, A^2, A^3, \dots \langle\subset \End(V)$
generated by $A$ is commutative.

\begin{opredelenie}
The operator $A\in \End(V)$ is called  {\bf semi-simple} if the
algebra generated by it in $\End(V)$ is semi-simple.
\end{opredelenie}

\begin{zadacha}
Prove that a linear operator over an algebraically closed field is
semi-simple if and only if it is diagonalizable.
\end{zadacha}

\begin{zadacha}[!]
Let $k\subset \bar k$  be two fields, moreover $\bar k$
is algebraically closed, and let $V$ be a finite dimensional vector
space over $k$. Consider $V\otimes_k \bar k$ as a vector space over
$\bar k$.  Prove that $\End(V)\otimes_k \bar k$ is naturally
isomorphic to $\End_{\bar k}(V\otimes_k \bar k)$.
This defines a natural inclusion
$\End(V)\arrow \End_{\bar k}(V\otimes_k \bar k)$.
Prove that the linear operator 
$A\in \End(V)$ is semi-simple if and only if 
the corresponding linear operator in $V\otimes_k \bar k$ is
diagonalizable. 
\end{zadacha}

\begin{zadacha}
Let $V$  be a two-dimensional vector space over $\R$ endowed with a
positive definite bilinear symmetric form, and let 
$A\in \End(V)$ be an orthogonal operator. Prove that it is
semi-simple. 
\end{zadacha}

\begin{zadacha}[*]
Let $V$ be a vector space over $\R$ of arbitrary finite dimension
endowed with a positive definite bilinear symmetric form, and let
$A\in \End(V)$ be an orthogonal operator. Prove that it is
semi-simple. 
\end{zadacha}

\begin{zadacha}[*]
Let $V$ be a vector space over $\R$ endowed with a non-degenerate
bilinear symmetric form, not necessarily positive definite, and let
$A\in \End(V)$ be an orthogonal operator. Is it always semi-simple? 
\end{zadacha}

\begin{opredelenie}
An element of an Artinian ring over $k$ is called {\bf semi-simple} if
it generates a semi-simple subalgebra in $R$.
\end{opredelenie}

\begin{zadacha}
  Let $R$ be an Artinian ring over $k$ and let $r\in R$ be a
  semi-simple element. Prove that in any representation of
  $R \arrow \End(V)$, $r$ is mapped to a semi-simple endomorphism of
  $V$.
\end{zadacha}

\begin{ukazanie}
Use the exercise~\ref{_Syurie_polupro_Zadacha_}.
\end{ukazanie}

\begin{zadacha}[!]
  Let $V$ be a finite dimensional vector space over an algebraically
  closed field, and let $A\in \End(V)$ be a linear operator. Prove
  that $A$ decomposes into a direct sum of a semi-simple and a
  nilpotent operator, $A= A_{ss}+A_n$, which commute. Prove that this
  decomposition is unique and $A_{ss}$, $A_n$ can be expressed as
  polynomials of $A$.
\end{zadacha}

\begin{ukazanie}
Use exercise~\ref{_razlozhe_kolca_Zadacha_}.
\end{ukazanie}

\begin{zadacha}[*]
Is this true if the base field $k$ is not algebraically closed? 
\end{zadacha}

\begin{zadacha}[!]
Let $A$ be a upper-triangular matrix,
$A_\delta$ be its diagonal part. Prove that  $A$ and $A_\delta$
commute. 
\end{zadacha}

\begin{zadacha}[**]
Let $(V,g)$ be a vector space endowed with a  bilinear skew-symmetric
form, and let $A$ be anti-symmetric operator and $A=A_{ss}+A_n$ be its
decomposition into a semi-simple and nilpotent part. Prove that
$A_{ss}$, $A_n$ are anti-symmetric.
\end{zadacha}

\begin{zadacha}[*]
  Is it possible that an antisymmetric operator over $\C$ be nilpotent?
\end{zadacha}


%%%%%%%%%%%%%%%%%%%%%%%%%%%%%%%%%%%%%%%%%%%%%%%%
\subs{Hamilton-Cayley theorem}
%%%%%%%%%%%%%%%%%%%%%%%%%%%%%%%%%%%%%%%%%%%%%%%%


Let $k$ be any field and let $k(t)$ be the field of rational functions
over $k$, and $V$ be an $n$-dimensional vector space over $k$, and
$B(t)\in \End(V)[t]$ a polynomial with coefficients in
$\End(V)$. Recall that in this situation $\det(B(t))$ is a polynomial
of $t$ (see ALGEBRA 8).  Let us consider $B(t)$ as a $k(t)$-linear
endomorphism of $V\otimes k(t)$. Consider the endomorphism
$\Lambda^{n-1}(V\otimes k(t))$ induced by $B(t)$ and let $\check B(t)$
be the adjoint endoorphim of $V \otimes k(t)$ with respect to the
natural pairing
\[ 
\Lambda^{n-1}(V\otimes k(t))\otimes V\otimes k(t)
\arrow \det V\otimes  k(t)
\]
It is shown in ALGEBRA 7 that $B(t)\check B(t)= \check
B(t)B(t)=\det(B(t))\Id_V$. 

\begin{zadacha}
In this situation show that $\check B(t)$ is
$\End(V)$-valued polynomial:  $\check B(t)\in \End(V)[t]$.
\end{zadacha}

\begin{ukazanie}
Express $\check B(t)$ via the minors of  $B(t)$.
\end{ukazanie}

\begin{zadacha}
Let $A\in \End(V)$. Applying the argument from the Remark to $B=t-A$
prove that $(t-A)\check{(t-A)}= \chpoly_A(t)$.
Prove the the coefficients of the polynomial
$\check{(t-A)}\in \End(V)[t]$ commute with $A$.
\end{zadacha}

\begin{zadacha}
Let $R\subset \End(V)$ be a subset.
Denote by $Z(R)$ the set of all operators
$A'\in \End(V)$ that commute with all operators
 $r\in R$ (this set is called the {\bf centralizer} of $R$). Prove
 that $Z(R)$ is a subalgebra of $\End(V)$. 
\end{zadacha}

\begin{zadacha} 
Let $R\in \End(V)$ be a subalgebra and let
$A_1\in Z(A)$ be an element of the centralizer $R$, 
$R[t]$ be the algebra of $R$-valued polynomials,
and $R[t]\stackrel \phi \arrow R'$ be a homomorphism of algebras.
Denote by  $R[A_1]$ the subalgebra $\End(V)$, generated by 
$R$ and $A_1$. Prove that there exists a homomorphism
 $\phi_0:\; R[A_1]\arrow R'$ such that
 $\phi_0\restrict R = \phi\restrict R$
and $\phi_0(A_1) = \phi(t)$. Prove that these conditions determine
$\phi_0$ uniquely. 
\end{zadacha}

\begin{zadacha} 
  Let $A\in \End(V)$ be a linear operator, apply the previous exercise
  and construct a homomorphism $Z(A)[t]\stackrel \Psi\arrow Z(A)$ that
  maps $t$ to $A$, and which is identity on $Z(A)$.
\end{zadacha}

\begin{zadacha}[!]
  (Cayley-Hamilton theorem) Consider the equality
  $(t-A)\check{(t-A)}= \chpoly_A(t)$ in $Z(A)[t]$. Apply the
  homomorphism $\Psi$ constructed above to both left and right parts
  of the equality. Prove that this results in the following equality
  in the algebra $\End(V)$:
  \[ \chpoly_A(A)=0.
  \]
\end{zadacha}

\begin{zadacha}[*]
Let $A,B\in \End V$ be linear operators. 
Consider a function of two variables $Q(t_1, t_2) = \det(t_1A + t_2 B)$,
where $t_1A + t_2 B$ is considered as a linear operator
on $V\otimes_k k(t_1,t_2)$, and $k(t_1,t_2)=k(t_1)(t_2)$ is the field
of rational functions over $k(t_1)$. Prove that 
$Q(t_1, t_2)$ is a polynomial with coefficients in $k$. 
Prove that in the ring $\End V$ the equality $Q(-B,A)=0$ holds. 
\end{zadacha}

\begin{zadacha}[!]
Let $A\in \End(V)$ be a linear operator that acts on a finite
dimensional vector space over an algebraically closed field $k$. Let
$\{\lambda_1,\dots,\lambda_n\}$ be the roots of the characteristic
polynomial of the operator $A$. Consider the space  $V_{\lambda_i}$ of
all $v\in V$ such that $(A-\lambda_i)^{m_i}(v)=0$ where 
$m_i$ is the multiplicity of the root $\lambda_i$ of the polynomial
$\chpoly_A(t)$.  Prove that  $V= \oplus V_{\lambda_i}$, where
summation is over all roots $\lambda_i$  of the characteristic
polynomial  $A$.
\end{zadacha}

\begin{ukazanie}
Use the Cayley-Hamilton theorem.
\end{ukazanie}

\begin{zamechanie}
The space $V_{\lambda_i}$ is called an
{\bf generalized eigenspace} of operator $A$.
\end{zamechanie}


%%%%%%%%%%%%%%%%%%%%%%%%%%%%%%%%%%%%%%%%%%%%%%%%
\subs{Minimal polynomial and characteristic polynomial}
%%%%%%%%%%%%%%%%%%%%%%%%%%%%%%%%%%%%%%%%%%%%%%%%


\begin{opredelenie}
Let $A\in \End(V)$ be a linear operator that acts on a vector space of
finite dimension over  $k$. ����������
a sequence of endomorphisms $1, A, A^2,\dots \in \End(V)$.
Since the space $\langle 1, A, A^2, \dots \rangle$
is finite dimensional, then starting from some -�� $i$
all $A^i$ can be expressed as a sum of the form:
$A^N = \sum_{i=0}^{l-1} \lambda_i A^i$, ��� $\lambda_i\in k$,
$l=\dim \langle 1, A, A^2,\dots \rangle$. 
Let us write down such an equation for $A^l$:
$A^l+ \sum_{i=0}^{l-1} \lambda_i A^i=0$.
Recall that the polynomial $P(t)=t^l +  \lambda_{l-1} t^{l-1} + \dots
+\lambda_0$ is called the  {\bf minimal polynomial } of $A$ and is
denoted $\minpoly_A(t)$.
\end{opredelenie}

\begin{zadacha}[!]
Prove that the following identity holds in the algebra $\End(V)$: 
$$
\minpoly_A(A)=0.
$$
Prove that any polynomial 
$Q(t)=t^{m} +  \mu_{m-1} t^{l-1} + \dots +\mu_0$,
such that $Q(A)=0$, is divided by $\minpoly_A(t)$.
\end{zadacha}

\begin{zadacha} 
Prove that the characteristic polynomial of the operator is divided by
its minimal polynomial.
\end{zadacha}


\begin{zadacha}
Let $A\in \End(V)$ be a linear operator.
\begin{enumerate}
\item Prove that $A$  is nilpotent if and only if  $\minpoly_A(t) =
  t^n$. 

\item Prove that $A$ is a non-identity idempotent if and only if
  $\minpoly_A(t)= t^2-t$. 
\end{enumerate}
\end{zadacha}

\begin{zadacha} 
  Let $A\in \End(V)$ be a linear operator that acts on a finite
  dimensional vector space over an algebraically closed field $k$ and
  let $V=\oplus V_{\lambda_i}$ be a decomposition of $V$ into a direct
  sum of generalized eigenspaces.  Let $P(t)$ be a minimal polynomial
  of $A$ and let $P_i(t)$ be minimal polynomials of restrictions of
  $A$ to $V_{\lambda_i}$. Prove that $P(t)=P_1(t)P_2(t)\dots$.  Prove
  that $P_i(t)= (t-\lambda_i)^k$ where $k\leq \dim V_{\lambda_i}$.
\end{zadacha}

\begin{ukazanie}
  It is clear that $P_i(t)= (t-\lambda_i)^k$, since the operator
  $A-\lambda_i$ on $V_{\lambda_i}$ is nilpotent. That
  $P(t)=P_1(t)P_2(t)\dots$ follows easily from the fact that all
  $P_j(A)$ ($j\neq i$) are invertible on $V_{\lambda_i}$.
\end{ukazanie}

\begin{zamechanie} 
  The characteristic polynomial also has this muliplicativity
  property, as can be easily observed.
\end{zamechanie}


\begin{zadacha} \label{_minpoly_chpoly_nilpo_Zadacha_}
Let $A\in \End(V)$ be a linear operator in a 
 $n$-dimensional vector space. Prove that
$\minpoly_A(t)= (t-\lambda)^n$ if and only if 
in some basis $A$ has the form
\begin{equation}\label{_Zhordanova_Cletka_Equation_}
\begin{pmatrix}
\lambda & 1 & 0 &\hdotsfor{3} &0\\
0 & \lambda & 1 & 0 &\hdotsfor{2} &0\\
0 & 0 & \lambda & 1 & 0 &\dots &0\\
\vdots &\vdots &\vdots
&\ddots 
&\vdots &\vdots &\vdots\\
0 & 0 &\hdotsfor{2} &\lambda & 1 &0\\
0 & 0 &\hdotsfor{3} &\lambda & 1\\
0 & 0 &\hdotsfor{4} &\lambda
\end{pmatrix}
\end{equation}
\end{zadacha}

\begin{ukazanie} 
Replacing $A$ by $A-\lambda \Id_V$ one may assume that
 $\minpoly_A(t)=t^n$. Take a vector $v\in V$ such that
 $(A-\lambda)^{n-1}(v)\neq 0$. Prove that 
$v, A(v), A^2(v), \dots, A^{n-1}(v)$ constitute a basis in $V$,
and in this basis $A$ hase the form
\eqref{_Zhordanova_Cletka_Equation_}.
\end{ukazanie}

\begin{zamechanie}
Such a matrix is called a {\bf Jordan block}.
We will denote it by $J(n,\lambda)$.
\end{zamechanie}

\begin{opredelenie}
Let $e_1,\dots,e_n$ be a basis in a vector space and let
$A_i^j$ be a matrix of a linear operator $A$ in this basis.
Assume that $e_1,\dots,e_n$ are divided into groups (blocks)
$[e_1,\dots, e_{k_1}] [e_{k_1+1},\dots, e_{k_2}] \dots$,
in such a way that $A$ maps every $e_i$ into a linear combination of
vectors that belong to the same block. In this case
$A$ consists of square pieces of size $k_i-k_{i-1}$, and is zero
outside of these square pieces:
$$
\begin{pmatrix}
* & \dots & * &0 &\dots &0 &0 &\dots &0\\
\vdots &\ddots &\vdots
&\vdots &\ddots &\vdots
&\vdots &\ddots &\vdots\\
* & \dots & * &0 &\dots &0 &0 &\dots &0\\
0 & \dots & 0 &* &\dots &* &0 &\dots &0\\
\vdots &\ddots &\vdots 
&\vdots &\ddots &\vdots
&\vdots &\ddots &\vdots\\
0 & \dots & 0 &* &\dots &* &0 &\dots &0\\
0 &\dots &0 &0 &\dots &0 &* & \dots & * \\
\vdots &\ddots &\vdots
&\vdots &\ddots &\vdots
&\vdots &\ddots &\vdots\\
0 &\dots &0 &0 &\dots &0 &* & \dots & *
\end{pmatrix}
$$
A matrix of this form is called \textbf{ block diagonal}.
\end{opredelenie}

\begin{zadacha} \label{_cikli_zhorda_Zadacha_}
Let $A\in \End(V)$ be a linear operator that acts on a finite
dimensional vector space over an algebraically closed field 
$k$. Assume that the minimal polynomial 
$\minpoly_A(t)$ equals characteristic polynomial
$\chpoly_A(t)$. Prove that in some basis $A$ 
can be represented as a block diagonal matrix that consists of Jordan blocks
$J(n_i, \lambda_i)$ where all $\lambda_i$ are distinct.
\end{zadacha}

\begin{ukazanie}
Use the multiplicativity of 
$\minpoly$ and $\chpoly$ with respect to decomposition of 
$V$ into a direct sum of generalized eigenspaces, and reduce the
problem to the case $V=V_{\lambda_i}$.
Now apply exercise \ref{_minpoly_chpoly_nilpo_Zadacha_}.
\end{ukazanie}

\begin{opredelenie}
Assume that operator $A$ can be represented in some basis as a block
diagonal matrix that consists of Jordan blocks.
The operator $A$ is said then to be in a {\bf Jordan normal form}.
\end{opredelenie}

We will now show the unicity of the Jordan normal form, and then we
will show its existence. We work assuming the base field to be
algebraically closed.

\begin{zadacha} 
Let $A\in\End(V)$ be a nilpotent operator with the Jordan normal form
that consists of Jordan blocks $J(0, n_1),\dots,J(0,n_k)$. Prove that
the number of blocks in the Jordan normal form of $A$ equals the
dimension of the space $V/AV$. Prove that 
$A^jV/A^{j+1}V$ is the number of blocks $J(0,n_i)$
with $n_j\geq j$. Deduce that Jordan normal form of a nilpotent
operator is determined uniquely, up to permutation of the blocks.
\end{zadacha}

\begin{zadacha}[!]
Prove that Jordan normal form of any operator is unique, up to
permutation of the blocks.
\end{zadacha}

\begin{ukazanie} 
Decompose $V$ into a direct sum of generalized eigenspaces, and reduce
the problem to the case $V=V_{\lambda_i}$. 
Replacing $A$ by $A-\lambda_i$, one can content oneself with nilpotent
operators. Now the statement follows from the previous exercise.
\end{ukazanie}

\begin{opredelenie}
  Let $A\in \End(V)$ be a linear operator.  We say that $A$ {\bf acts
    cyclically} on $V$ if there exists an element $v$ such that
  $v, Av, A^2v, A^3 v, \dots$ generates $V$.
\end{opredelenie}

\begin{zadacha} 
  Let $A\in \End(V)$ be a linear operator that acts cyclically on $V$.
  Prove that $\minpoly_A(t)=\chpoly_A(t)$.
\end{zadacha}

\begin{ukazanie}
If $A$ act cyclically then the degree of
$\minpoly_A(t)$ equals $\dim V$ equals the degree of 
$\chpoly_A(t)$.
\end{ukazanie}

\begin{zadacha} 
Let $A\in \End(V)$ be a linear operator such that $V$ decomposes as a
sum of  $A$-invariant subspaces on which $A$ acts cyclically.
Prove that $A$ can be represented in some basis in a Jordan normal
form. 
\end{zadacha}

\begin{ukazanie}
Use the exercise \ref{_cikli_zhorda_Zadacha_}.
\end{ukazanie}


%%%%%%%%%%%%%%%%%%%%%%%%%%%%%%%%%%%%%%%%%%%%%%%%%%%%%%%%%%%%
\subs{Modules over a ring and Jordan normal form}
%%%%%%%%%%%%%%%%%%%%%%%%%%%%%%%%%%%%%%%%%%%%%%%%%%%%%%%%%%%%

\begin{opredelenie}
Let $R$ be a ring. A {\bf module} over $R$ is an Abelian group $�$
endowed with an operation $R\times M\arrow M$ that is compatible with
the addition in the following sense
\begin{enumerate}
\renewcommand{\labelenumi}{(\roman{enumi})}
\item For any $\lambda\in R$, $u, v\in M$ we have  $\lambda(u+v) =
\lambda u+ \lambda v$. For any $\lambda_1, \lambda_2\in R$, $u\in
M$ we have $(\lambda_1+\lambda_2) u = \lambda_1 u +\lambda_2 u$
(distributivity of multiplication over addition).

\item For any $\lambda_1, \lambda_2\in R$, $u\in M$ we have
$\lambda_1(\lambda_2 u) = (\lambda_1 \lambda_2) u$ (associativity of
multiplication ).

\item For any $v\in M$ we have $1v=v$ where 1 denotes the identity in
  $R$. 
\end{enumerate}
\end{opredelenie}

\begin{zamechanie} 
This definition repeats almost verbatim the definition of a vector
space over a field. Many notions that were defined for vector spaces 
(for example, homomorphism, monomorphism, epimorphism, kernel, image,
quotient space)  can be redefined without modification for modules
over a ring.
\end{zamechanie}

\begin{zadacha} 
Let $R$ be an algebra over a field $k$. For any module 
 $M$ over $R$ consider $M$ as a vector space over  $k\subset
 R$. Consider each the operation of multiplication by an elements of $R$
as an endomorphism of $M$. Prove that this defines a homomorphism 
$R\arrow \End_k(M)$. Prove that all representations can be obtained in
this way.
\end{zadacha}

\begin{zadacha}
Prove that any Abelian group has a unique structure of a module over
 $\Z$.
\end{zadacha}

\begin{opredelenie}
Consider the group $R^n$ as a module over $R$, with the action given
by  $r\cdot(x_1,\dots, x_n) = (rx_1,\dots, rx_n)$. This module is
called {\bf free}. The quotient of $R^n$ by a submodule is called
{\bf finitely generated}. If  $M$ can be represented as a quotient of
a free module by a finitely generated submodule then
$M$ is called {\bf finitely presented}.
\end{opredelenie}

\begin{opredelenie}
Let $\phi:M \to M'$ be a homomorphism of modules over an algebra $R$.
The {\bf cokernel} $\phi$ (denoted by $\coker \phi$)
is the quotient of $M'$ by the image of $\phi$.
\end{opredelenie}

\begin{zadacha} 
  Let $M$ be a module over $R$. Prove that $M$ is finitely generated
  if and only if it has a collection of elements $m_1,\dots,m_N$ such
  that any element of $M$ can be represented as a linear combination ,
  $m = r_1m_1 + \dots + r_Nm_N$, $r_1,\dots,r_N \in R$.
\end{zadacha}

\begin{zadacha}
Let $M$ be a module over $R$. Prove that $M$ is finitely presented
if and only if it is isomorphic to a cokernel of a homomorphism 
$\phi:R^N \to R^M$ of free $R$-modules.
\end{zadacha}

\begin{zadacha}[!]\label{_kone_poro_Zadacha_}
Let $k$ be a field and $M$ be a module over $k[t]$ that has
a finite dimension over $k$. Prove that
$M$ is finitely generated and finitely presented over $k[t]$.
\end{zadacha}

\begin{ukazanie}
Consider  $M$ as a vector space over $k$, pick  a basis
 $m_1,\dots,m_M \in M$, and consider these elements as
 generators. Then prove that the kernel of the map
$\phi: M \otimes_k k[t] \to M$ is generated by elements of the
form  $m_i \otimes t - tm_i \otimes 1$. 
\end{ukazanie}

\begin{zadacha}[!]
Let $M$ be a finite Abelian group. Prove that $M$ is finitely
generated and finitely presented as a module over  $\Z$.
\end{zadacha}

\begin{ukazanie}
Take all elements of $M$ as the set of generators $m_1,\dots,m_N \in
M$. 
\end{ukazanie}

\begin{zadacha}
  Let $R$ be a ring and $V$ a module over $R$ represented as the
  cokernel of the homomorphism $(R)^n\stackrel\phi\arrow (R)^m$.
  Write down $\phi$ as a matrix $A^i_j$ with coefficients in $R$. Let
  $B^i_j$ be a matrix obtained from $R$ using elementary (Gaussian)
  row and column transformations (see ALGEBRA 7).  Prove that $V$ is
  isomorphic to a cokernel of a homomorphism that corresponds to
  $B^i_j$.
\end{zadacha}

\begin{opredelenie}
Let $R$ be a ring and $a\in R$ be an element. Consider  $aR$ as a
module over $R$.
A {\bf cyclic module} over $R$ is a quotient module $R/aR$.
\end{opredelenie}

\begin{zadacha}
Let $M$ be  a $Z$-module.
Prove that $M$  is cyclic if and only if
the corresponding Abelian group is cyclic.
\end{zadacha}

\begin{zadacha}
Let $M$ be a  $k[t]$-module.
Prove that $M$ is cyclic if and only if
for some  $v\in M$, 
$v, tv, t^2v, t^3v , \dots$
generated $M$.
\end{zadacha}

\begin{zadacha}[!]
Let $R$ be a ring such that any 
 $n \times m$ matrix with coefficients in 
$R$ can be brought into a diagonal form using elementary row and
column operations. Prove that any finitely generated and finitely
presented module over $R$ is isomorphic to a direct sum of cyclic
ones. 
\end{zadacha}

\begin{ukazanie}
If  $(R)^n\stackrel\phi\arrow (R)^n$ is represented by a diagonal matrix
with $a_i^i$ on the diagonal then the cokernel of this homomorphism
has the form $\oplus_i R/a_i^i R$.
\end{ukazanie}

\begin{zadacha}[!]
  Let $R$ be a Euclidean ring (see ALGEBRA 2).  Prove that any matrix
  of size $n \times m$ with coefficients in $R$ can be brought to a
  diagonal form using elementary row and column operations. Deduce
  that any finitely generated and finitely presented module over $R$
  is a direct sum of cyclic ones.
\end{zadacha}

\begin{ukazanie}
A similar problem is in ALGEBRA 7.
\end{ukazanie}

\begin{zadacha}[!]
Let $G$ be a finite Abelian group.
Prove that $G$ is a finite sum of cyclic groups.
\end{zadacha}

\begin{zadacha}[!]
Let  $V$ be a module over $k[t]$ which is finite dimensional over
$k$. Prove that $V$ is a direct sum of cyclic modules.
\end{zadacha}

\begin{ukazanie} The ring $k[t]$ is Euclidean and 
$V$ is finitely generated and finitely presented as fallows from
exercise \ref{_kone_poro_Zadacha_}.
\end{ukazanie}

\begin{zadacha}[!]
Let  $A\in \End(V)$ be a linear operator. Prove that $V$ can be
decomposed into a direct sum of  $A$-invariant subspaces so that on
each of them $A$ acts cyclically. Deduce that if the field $k$ is
algebraically closed then $A$ can be brought into Jordan normal form. 
\end{zadacha}

\begin{ukazanie}
Consider the action of $k[t]$ on $V$ given by
$P(t)(v) = P(A)v$. Prove that $V$ is a
$k[t]$-module. Decompose $V$ into a direct sum of cyclic submodules:
$V=\oplus V_i$.  Prove that all $V_i$  are $A$-invariant,
and $A$  acts on them cyclically.
\end{ukazanie}

\begin{zadacha}[*]
Find a commutative ring and a module over it that does not decompose
into a direct sum of cyclic ones.
\end{zadacha}

\end{document}


%%% Local Variables: 
%%% mode: latex
%%% coding: koi8-r
%%% ispell-local-dictionary: "british"
%%% TeX-master: t
%%% End: 
