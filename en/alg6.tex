\documentclass[12pt]{article}

\usepackage{theorem,amsmath,amssymb}



\addtolength{\topmargin}{-23mm}
\addtolength{\textheight}{60mm}
\addtolength{\oddsidemargin}{-20mm}
\addtolength{\textwidth}{40mm}

\def\eqref#1{(\ref{#1})}
\newcommand{\goth}{\mathfrak}
\newcommand{\arrow}{{\:\longrightarrow\:}}
\def\1{\sqrt{-1}\:}
\newcommand{\restrict}[1]{{\left|_{{\phantom{|}\!\!}_{#1}}\right.}}

\renewcommand{\bar}{\overline}
\renewcommand{\phi}{\varphi}
\renewcommand{\epsilon}{\varepsilon}
\renewcommand{\geq}{\geqslant}
\renewcommand{\leq}{\leqslant}

\def\rad{\operatorname{\sf rad}}
\def\tr{\operatorname{\sf tr}}
\def\rk{\operatorname{\sf rk}}
\def\Alt{\operatorname{\sf Alt}}
\def\Sym{\operatorname{\sf Sym}}
\def\Id{\operatorname{\sf Id}}
\def\Hom{\operatorname{Hom}}
\def\Map{\operatorname{Map}}
\def\Gal{\operatorname{Gal}}
\def\Aut{\operatorname{Aut}}
\newcommand{\End}{\operatorname{End}}
\newcommand{\Mat}{\operatorname{Mat}}

\newcommand{\coker}{\operatorname{Coker}}

\def\chpoly{\operatorname{\sf Chpoly}}
\def\minpoly{\operatorname{\sf Minpoly}}

\def\cchar{\operatorname{\sf char}}

\def\Z{{\mathbb Z}}
\def\R{{\mathbb R}}
\def\C{{\mathbb C}}
\def\Q{{\mathbb Q}}
\def\N{{\mathbb N}}
\def\F{{\mathbb F}}

\def\Re{\operatorname{Re}}
\def\Im{\operatorname{Im}}

\makeatletter
\theoremstyle{definition}

\newtheorem{zadacha}{������}[section]
\newtheorem{opredelenie}{�����������}[section]
\newtheorem*{ukazanie}{��������}%[section]
\newtheorem*{zamechanie}{���������}%[section]

%\renewcommand{\labelenumi}{\ralph{enumi}.}
\renewcommand{\labelenumi}{\alph{enumi}.}
\newcommand{\subs}[1]{{\bigskip\centerline{\bf\large #1}\bigskip}}
\newcommand{\sttr}{{\bf(*)}}
\newcommand{\shrk}{{\bf(!)}}
\newcommand{\doublesttr}{{\bf(**)}}

\newcommand{\listok}[2]{%
\setcounter{page}{1}
\renewcommand{\@oddhead}{\hfil #2 \hfil}
\renewcommand{\@evenhead}{\hfil #2 \hfil}
\section*{#2}
\refstepcounter{section}
\setcounter{section}{#1}
}

\@addtoreset{equation}{section}
\renewcommand{\theequation}{\thesection.\arabic{equation}}

\let\oldllim=\lim
\def\lim{\oldllim\limits}
\makeatother


\newcommand{\dett}{\operatorname{det}}
\newcommand{\hdot}{{\:\raisebox{3pt}{\text{\circle*{1.5}}}}}
 
\begin{document}

%%%%%%%%%%%%%%%%%%%%%%%%%%%%%%%%%%%%%%%%%%%%%%%%

\listok{6}{ALGEBRA 6: Grassmann algebra and determinant}

%%%%%%%%%%%%%%%%%%%%%%%%%%%%%%%%%%%%%%%%%%%%%%%%%%%%%%%%%%%%

%%%%%%%%%%%%%%%%%%%%%%%%%%%%%%%%%%%%%%%%%%%%%%%%
\subs{Grassmann algebra}
%%%%%%%%%%%%%%%%%%%%%%%%%%%%%%%%%%%%%%%%%%%%%%%%

\begin{opredelenie} 
  An algebra $A$ is called {\bf graded}, if $A$ can be represented in
  the form $A = \oplus_{i=1}^{\Z} A^i$ and the multiplication satisfies
  the following condition: $A^i \cdot A^j \subset A^{i+j}$.  $\oplus_i
  A^i$ is often written as $A^\hdot$, which means a direct sum over
  all possible indices. Some $A^i$ subspaces can be empty. Algebra
  unit (if it exists) always belongs to $A^0$.
\end{opredelenie}

\begin{zadacha}
What is the natural grading of $T(V)$?
\end{zadacha}

\begin{opredelenie}
  A subspace $W\subset A^\hdot$ of a graded algebra is called {\bf
    graded} or {\bf homogeneous}, if $W$ is a direct sum of components
  of the form $W^i\subset A^i$.
\end{opredelenie}

\begin{zadacha}[!]
Consider a graded subspace $W\subset T(V)$. Prove that algebra defined
by the relations space $W$ is graded. 
\end{zadacha}

\begin{zadacha} 
  Consider a vector space $V$ and its basis $\langle x_i\rangle
  $. Consider a subspace $W\subset V\otimes V$ generated by vectors
  of the form $x\otimes y - y\otimes x$.  Prove that an algebra of
  polynomials $k[x_1, \dots, x_n]$ is defined by generators $V$ and
  relations $W$. Describe a natural grading inherited from $T(V)$.
\end{zadacha}

\begin{opredelenie}
  The algebra obtained is called {\bf symmetric algebra of space $V$},
  � and is denoted as $\Sym^\hdot(V)$.
\end{opredelenie}

\begin{zadacha}
  Let $\dim V>1$. Are there an injective algebra homomorphism
  $\Sym^\hdot(V)\arrow T(V)$.
\end{zadacha}

\begin{opredelenie}
  Consider a vector space $V$ and a graded subspace $W\subset V\otimes
  V$ generated by vectors of the form $x\otimes y + y\otimes x$ and
  vectors of the form $x\otimes x$.  The graded algebra defined by the
  generators space $V$ and relations space $W$ is called a {\bf
    Grassmann algebra} and is denoted as $\Lambda^\hdot(V)$.  The
  space $\Lambda^i(V)$ is called an {\bf $i$-th exterior power} of the
  space $V$ and the operation of multiplication in Grassmann algebra
  is called {\bf exterior multiplication}. Exterior multiplication is
  usually denoted as $\wedge$.
\end{opredelenie}

\begin{zamechanie}
Elements of Grassmmann algebra can be thought of as
``anticommutative polinomials'' on $V$.
\end{zamechanie}

\begin{zamechanie}
Grassmann algebra is a particular case of Clifford algebra defined in
Algebra 5.
\end{zamechanie}

\begin{zadacha} 
Prove that $\Lambda^1 V$ is isomorphic to $V$.
\end{zadacha}

\begin{zadacha} 
  Consider a finite-dimensional space $V$. Prove that $\Lambda^2(V)^*$
  is isomorphic to a space of bilinear antisymmetric forms on $V$.
\end{zadacha}

\begin{zadacha}
  Consider a subalgebra $\Lambda^{2\hdot}(V)\subset \Lambda^\hdot(V)$
  that consists of linear combinations of vectors of even
  grading. Prove that this subalgebra is commutative.
\end{zadacha}

\begin{opredelenie} 
  Vector $\Lambda^\hdot(V)$ is called {\bf even}, if it belongs to an
  even grading component and {\bf odd} if it belongs to an odd
  component.  A {\bf parity} $\tilde x$ of a vector $x$ is defined to
  be zero for an even $x$ and 1 for an odd $x$.  ������������.
\end{opredelenie}

\begin{zadacha}[!]
  Prove that $xy = (-1)^{\tilde x \tilde y}yx$.
\end{zadacha}

\begin{zadacha}[*]
  Find all elements $\eta \in \Lambda^{2}(V)$ such that $\eta^2 =0$.
\end{zadacha}

\begin{zadacha}[!]
  Let $x_1, x_2, \dots$ be a basis in $V\cong \Lambda^1 V$. Denote the
  product of vectors that belong to the basis in $\Lambda^\hdot(V)$ as
  $x_{i_1}\wedge x_{i_2} \wedge x_{i_3} \wedge \cdots$ . Prove that
  vectors of the form $x_{i_1}\wedge x_{i_2} \wedge x_{i_3} \wedge
  \cdots$ where $i_1 < i_2 < i_3 < \dots$, define a basis in
  $\Lambda^\hdot(V)$.
\end{zadacha}

\begin{zadacha}[!]
Let $V$ be a $d$-dimensional vector space. Find $\dim
\Lambda^i(V)$. Prove that $\Lambda^d V$ is one-dimensional.
\end{zadacha}

\begin{opredelenie}
  The space $\Lambda^d V$ is called a {\bf space of determinant
    vectors in $V$}.
\end{opredelenie}

\begin{zadacha}[!]
  Let $V$ be a $d$-dimensional vector space, $x_1, x_2, \dots, x_d$ be
  its basis and $\dett:= x_{1}\wedge x_{2} \wedge x_{3} \dots \wedge
  x_d$ be a determinant vector in $\Lambda^d V$. Consider a
  permutation $I = (i_1, i_2, \dots, i_d)$ and consider a vector
  $I(\dett):= x_{i_1}\wedge x_{i_2} \wedge x_{i_3} \dots \wedge
  x_{i_d}$. Prove that $I(\dett)=\pm\dett$. Prove that this
  correspondence defines a homomorphism from a permutation group $S_n$
  into $\{\pm 1\}$. Prove that this homomorphism maps a product of an
  odd number of transpositions to $-1$ and a product of even number of
  transpositions to 1.
\end{zadacha}

\begin{opredelenie}
  A homomorphism constructed above $S_n\overset{\sigma}{\arrow}
  \Z/2\Z$ is called a {\bf sign} of a permutation. The additive
  notation is used here for historical reasons. One says that a
  permutation is {\bf even} if its sign is 0 and is {\bf odd} if its
  sign is 1.
\end{opredelenie}

\begin{zadacha} 
Consider a permutation decomposed into cycles as follows:
\[ 
I = (i_{1,1}, i_{2,1} \dots i_{k_1, 1})(i_{1,2}, i_{2,2} \dots
i_{k_2, 2}) \dots
\]
where cycles are of length $k_1$, $k_2$ etc. Prove that $I$ is even
iff there is an even number of even $k_i$-s.
\end{zadacha}

\bigskip

From now till the end of the section we suppose that the field $k$ we
are using is of characteristic 0.

\begin{opredelenie} 
Let $\eta\in V^{\otimes i}$ be a vector of a $i$-th tensor power of the
space $V$. Consider a natural action of $S_i$ on $V^{\otimes
  i}$. Define $\Alt(\eta)$ as 
\[
\Alt(\eta) := \frac 1 {i!}\sum_{I\in S_i} (-1)^{\sigma(I)} I(\eta).
\]
This operation is called {\bf alternation}. One says that a vector
$\eta\in V^{\otimes i}$ is {\bf totally antisymmetric} if $\eta =
\Alt(\eta)$.
\end{opredelenie}

\begin{zadacha} 
Let $\eta = \frac 1 {i!}\sum_{I\in S_i} I(\eta)$. Prove that 
$I(\eta) = \eta$ for any permutation $I\in S_i$.
\end{zadacha}

\begin{zadacha}[!]
  Consider a totally antisymmetric vector $\eta\in V^{\otimes
    i}$. Prove that $I(\eta) = (-1)^{\sigma(I)} \eta$ for any
  permutation $I\in S_i$.
\end{zadacha}

\begin{zadacha}[!]
Prove that $\Alt(\Alt(\eta))=\Alt(\eta)$ for any $\eta$.
\end{zadacha}

\begin{zadacha}
Consider a tensor $x_{i_1} x_{i_2} \cdots x_{i_k}\in V^{\otimes_i}$.
Prove that 
\[ 
\Alt(x_{i_1} x_{i_2} \dots x_{i_k})= 
-\Alt(x_{i_1} x_{i_2}\dots x_{i_l} x_{i_l-1} \dots x_{i_k})
\]
($x_{i_l}$ is permuted with $x_{i_l-1}$ in the second expression).
\end{zadacha}

\begin{zadacha} 
  Prove that the map $x_{i_1} x_{i_2} \dots x_{i_k}\arrow \Alt(x_{i_1}
  x_{i_2} \dots x_{i_k})$ vanishes on all tensors of the form $a w b$,
  where $w$ belongs to the relations space of
  $\Lambda^\hdot(V)$. Deduce that there exists a natural map
  $\Lambda^i(V)\arrow R^i$ from $\Lambda^i(V)$ to the space of totally
  antisymmetric tensors.
\end{zadacha}

\begin{zadacha}[!]
  Prove that the natural map constructed above $\Lambda^i(V)\arrow
  R^i$ is a bijection.
\end{zadacha}

\begin{zadacha}[!]
  We put $\Lambda^i(V)$ into one-to-one correspondence with the space
  of totally antisymmetric tensors. It defines a multiplicative
  structure on antisymmetric tensors. Prove that this multiplicative
  structure can be defined like this: take two totally antisymmetric
  tensors $\alpha, \beta\in T(V)$, multiply them in $T(V)$ and apply
  $\Alt$ to the result.
\end{zadacha}

\begin{zadacha}
  Consider two algebras $A$ and $B$ over a field $k$. Define a
  multiplicative structure on $A\otimes B$ like this: $a\otimes b
  \cdot a'\otimes b'= aa' \otimes bb'$.  Prove that this
  multiplication indeed defines an algebra structure on $A\otimes B$.
\end{zadacha}

\begin{opredelenie}
  A tensor product of algebras $A$ and $B$ is a space $A\otimes B$
  with multiplicative structure defined above. If the algebras are
  graded, then the grading on the tensor product is defined by the
  formula $(A\otimes B)^n = \oplus_{i+j=n} A^i\otimes B^j$.
\end{opredelenie}

\begin{zadacha}[!]
  Let $V_1$, $V_2$ be vector spaces. Prove that $\Sym^\hdot(V)$ is
  isomorphic (as an algebra) to
  $\Sym^\hdot(V_1)\otimes\Sym^\hdot(V_1)$. Prove that
  $\Lambda^\hdot(V_1\oplus V_2)$ and $\Lambda^\hdot(V_1)\otimes
  \Lambda^\hdot(V_2)$ are isomorphic as vector spaces.  Is this
  isomorphism an isomorphism of algebras?
\end{zadacha}

\begin{zadacha}
Prove that $\dim \Lambda^\hdot(V)= 2^{\dim V}$.
\end{zadacha}

\begin{ukazanie}
Use the previous problem.
\end{ukazanie}

\begin{zadacha}[*]
Consider a mapping
\[ 
V\otimes \Lambda^i(V)\overset{\wedge}{\arrow} \Lambda^{i+1}(V),
\]
defined by the formula $x\otimes \eta \mapsto x \wedge\eta$. For some
fixed $\eta$ we get a linear operator $L_\eta:\; V\arrow
\Lambda^{i+1}(V)$. Prove that for $\eta\neq 0$ an inequality $\dim
\ker L_\eta \leq i$ holds.
\end{zadacha}

\begin{zadacha}[*]
Suppose in the previous problem setting an equality $\dim \ker L_\eta =
i$ holds. Prove that in this case $\eta$ can be represented as $\eta
= x_1\wedge x_2\wedge \dots \wedge x_i$ for some vectors $x_1,
\dots, x_i \in V$.
\end{zadacha}

\begin{zadacha}[*]
  Let $P\in \Sym^i(V^*)$ be a symmetric $i$-form on $V$. Suppose that
  $P(v, v, v, \dots) =0$ for all $v\in V$.  Is it possible that $P$ is
  non-zero?
\end{zadacha}

\newpage

%%%%%%%%%%%%%%%%%%%%%%%%%%%%%%%%%%%%%%%%%%%%%%%%
\subs{Determinant}
%%%%%%%%%%%%%%%%%%%%%%%%%%%%%%%%%%%%%%%%%%%%%%%%

\begin{zadacha}
Consider a one-dimensional vector space $V$ over $k$. Prove that $\End
V$ is naturally isomorphic to $k$.
\end{zadacha}

\begin{zadacha}[!]
Consider a linear space $V$ and a linear operator $A\in
\End(V)$. Prove that $A$ on $V\cong \Lambda^1 V$ can be uniquely
extended to a grading preserving homomorphism from $\Lambda^\hdot V$
to itself.
\end{zadacha}

\begin{opredelenie}
  Consider a $d$-dimensional vector space $V$ over $k$ and a linear
  operator $A\in \End(V)$. Consider a endomorphism induced by $A$
  defined on a space of determinant vectors:
\[
\dett A\in \End(\Lambda^d(V))
\]
Since $\Lambda^d(V)$ is one-dimensional,  $\End
(\Lambda^d(V))$ is naturally isomorphic to $k$. This allows to treat
$\dett A$ as a number, i.e.\ an element of $k$. This number is called
a {\bf determinant} of a linear operator $A$.
\end{opredelenie}

\begin{zadacha}[!]
  Consider a set of $d$ vectors $x_1, \dots, x_d$ in a vector space
  $V$. Prove that their product $x_1 \wedge x_2 \wedge \dots$ in
  $\Lambda^\hdot(V)$ is zero iff these vectors are linearly
  dependent.
\end{zadacha}

\begin{zadacha} 
  Consider an operator $A\in \End(V)$ which has a non-zero kernel
  (such an operator is called {\bf singular} of {\bf
    degenerate}). Prove that $\dett A=0$.
\end{zadacha}

\begin{zadacha} 
  Let an operator $A\in \End(V)$ be invertible (such an operator is
  called {\bf nonsingular} or {\bf nondegenerate}). Prove that $\dett
  A\neq 0$.
\end{zadacha}

\begin{zadacha}[!] 
Prove that $\dett$ defines a homomorphism from a group $GL(V)$ of
invertible matrices to $k^*$, a multiplicative group of all nonzero
elements of $k$.
\end{zadacha}

\begin{zadacha}[!]
  Consider vector spaces $V$ and $V'$, and endomorphisms $A, A'$. Then
  $A\oplus A'$ defines an endomorphism $V\oplus V'$. Prove that
  $\dett( A\oplus A') = \dett A \dett A'$.
\end{zadacha}

\begin{zadacha}
  Consider a finite-dimensional vector space $V$, endowed with a
  positive bilinear symmetric form $g$. Recall that an endomorphism
  $A\in \End V$ is called {\bf orthogonal} if it preserves $g$, i.e.\
  for any $x, y\in V$ it is true that $g(Ax, Ay) = g(x,y)$. Prove that
  an orthogonal operator is always invertible. Consider an orthogonal
  operator in $\R^2$. What values can  $\dett A$ can take?
\end{zadacha}

\begin{zadacha}[*]
Consider a vector space $V$ endowed with
\begin{enumerate}
\item nondegenerate bilinear symmetric form $g$

\item nondegenerate bilinear antisymmetric form $g$
 
\item\doublesttr{} nondegenerate bilinear form (i.e.\ an isomorphism
  $g:\; V\arrow V^*$).
\end{enumerate}
Consider a linear operator $A\in \End(V)$ that preserves $g$. Prove
that $A$ is invertible in any of the aforementioned cases and find all
the values that $\dett A$ can take.
\end{zadacha}

\end{document}
