\documentclass[12pt]{article}

\usepackage{theorem,amsmath,amssymb}



\addtolength{\topmargin}{-23mm}
\addtolength{\textheight}{60mm}
\addtolength{\oddsidemargin}{-20mm}
\addtolength{\textwidth}{40mm}

\def\eqref#1{(\ref{#1})}
\newcommand{\goth}{\mathfrak}
\newcommand{\arrow}{{\:\longrightarrow\:}}
\def\1{\sqrt{-1}\:}
\newcommand{\restrict}[1]{{\left|_{{\phantom{|}\!\!}_{#1}}\right.}}

\renewcommand{\bar}{\overline}
\renewcommand{\phi}{\varphi}
\renewcommand{\epsilon}{\varepsilon}
\renewcommand{\geq}{\geqslant}
\renewcommand{\leq}{\leqslant}

\def\rad{\operatorname{\sf rad}}
\def\tr{\operatorname{\sf tr}}
\def\rk{\operatorname{\sf rk}}
\def\Alt{\operatorname{\sf Alt}}
\def\Sym{\operatorname{\sf Sym}}
\def\Id{\operatorname{\sf Id}}
\def\Hom{\operatorname{Hom}}
\def\Map{\operatorname{Map}}
\def\Gal{\operatorname{Gal}}
\def\Aut{\operatorname{Aut}}
\newcommand{\End}{\operatorname{End}}
\newcommand{\Mat}{\operatorname{Mat}}

\newcommand{\coker}{\operatorname{Coker}}

\def\chpoly{\operatorname{\sf Chpoly}}
\def\minpoly{\operatorname{\sf Minpoly}}

\def\cchar{\operatorname{\sf char}}

\def\Z{{\mathbb Z}}
\def\R{{\mathbb R}}
\def\C{{\mathbb C}}
\def\Q{{\mathbb Q}}
\def\N{{\mathbb N}}
\def\F{{\mathbb F}}

\def\Re{\operatorname{Re}}
\def\Im{\operatorname{Im}}

\makeatletter
\theoremstyle{definition}

\newtheorem{zadacha}{������}[section]
\newtheorem{opredelenie}{�����������}[section]
\newtheorem*{ukazanie}{��������}%[section]
\newtheorem*{zamechanie}{���������}%[section]

%\renewcommand{\labelenumi}{\ralph{enumi}.}
\renewcommand{\labelenumi}{\alph{enumi}.}
\newcommand{\subs}[1]{{\bigskip\centerline{\bf\large #1}\bigskip}}
\newcommand{\sttr}{{\bf(*)}}
\newcommand{\shrk}{{\bf(!)}}
\newcommand{\doublesttr}{{\bf(**)}}

\newcommand{\listok}[2]{%
\setcounter{page}{1}
\renewcommand{\@oddhead}{\hfil #2 \hfil}
\renewcommand{\@evenhead}{\hfil #2 \hfil}
\section*{#2}
\refstepcounter{section}
\setcounter{section}{#1}
}

\@addtoreset{equation}{section}
\renewcommand{\theequation}{\thesection.\arabic{equation}}

\let\oldllim=\lim
\def\lim{\oldllim\limits}
\makeatother


\begin{document}

%%%%%%%%%%%%%%%%%%%%%%%%%%%%%%%%%%%%%%%%%%%%%%%%%%%%%%%%%%%%

\listok{2}{GEOMETRY 2: real numbers, part 2.}

%%%%%%%%%%%%%%%%%%%%%%%%%%%%%%%%%%%%%%%%%%%%%%%%%%%%%%%%%%%%

\subs{Roots of polynomials of an odd degree.}

\begin{zadacha}[!] Consider a polynomial over $\Q$ of an odd degree,
$P= t^{2n+1} + a_{2n} t^{2n} + a_{2n-1} t^{2n-1} + \cdots + a_0$.  Let
$R_P$ be the set of all $x\in \Q$ such that $P(t)<0$ on an interval 
$]-\infty, x]$. Prove that $R_P$ is not empty.
\end{zadacha}

\begin{ukazanie} Prove that $R_P$ contains
$-\max(1, \sum |a_i|)$.
\end{ukazanie}

\begin{zadacha}[!] 
Prove that $R_P$ is not the set of all real numbers.
\end{zadacha}

\begin{ukazanie} Prove that the complement $\Q \backslash R_P$ 
contains $\max(1, \sum |a_i|)$.
\end{ukazanie}

\begin{zadacha}[!] Prove that $R_P$ is a Dedekind section. 
\end{zadacha}

\begin{zadacha}[!]\label{lips} 
  Prove that $P$ satisfies {\bf Lipschitz property}: for any interval
  $I$ there exists a constant $C > 0$ such that $|P(a)-P(b)| < C|a-b|$
  for any $a,b \in I$.
\end{zadacha}

\begin{zadacha}[!] Consider a Dedekind section
$R_P$ as a real number. Prove that $P(R_P)=0$. It follows that any
polynomial over $\R$ of an odd degree has a root.
\end{zadacha}

\begin{ukazanie}
First prove that $P(R_P) \leq 0$. Then prove that $P(R_P) <
0$ contradicts the problem~\ref{lips}.
\end{ukazanie}

%%%%%%%%%%%%%%%%%%%%%%%%%%%%%%%%%%%%%%%%%%%%%%%%
\subs{Limits.}
%%%%%%%%%%%%%%%%%%%%%%%%%%%%%%%%%%%%%%%%%%%%%%%%

\begin{opredelenie} Let $A\subset \R$ be a set of real numbers and $c$
  be a real number. Then $c$ is called {\bf accumulation point
    (limit point)} of a set $A$ if every open interval $I = ]x,
  y[$ containing $c$ contains infinitely many elements of $A$.
\end{opredelenie}

\begin{opredelenie} Let $\{a_i\}$ be a sequence of real numbers and
  $c$ be a real number. Let any open interval $I = ]x, y[$
  containing $c$ contain all elements of $\{a_i\}$ except a
  finite number of them. Then $c$ is called the {\bf limit of the sequence
    $\{a_i\}$} (denoted by $c = \lim_{i \to \infty} a_i $). One says
  that the sequence $a_i$ {\bf converges to $c$}.
\end{opredelenie}

\begin{zadacha} Let $c$ be an accumulation point of a sequence
  $\{a_i\}$. Prove that there exists a subsequence of $\{a_i\}$ that
  converges to $c$.
\end{zadacha}

\begin{zadacha}[*] Consider a sequence $\{a_i\}$ of points from an
  interval $[x,y]$. Prove the existence of accumulation points of that
  sequence.
\end{zadacha}

\begin{opredelenie} A set $A\subset \R$ is called {\bf
discrete} if it has no accumulation points.
\end{opredelenie}

\begin{zadacha}[*] Let $\{a_i\}$ be a sequence. Denote a set of all
  $a_i$ by $A$.  Prove that $\{a_i\}$ converges if and only if $A$ has
  no infinite discrete subsets and has a unique accumulation point.
\end{zadacha}

\begin{zadacha} Consider a sequence $0, 1, 2, 3, 4,
\ldots$. Prove that this sequence has no limit.
\end{zadacha}

\begin{zadacha} Consider a sequence  $0, 1, 1/2, 1/3, 1/4,
\ldots$. Prove that this sequence converges to 0.
\end{zadacha}

\begin{zadacha} Consider an increasing sequence
$a_1\leq a_2 \leq a_3 \leq \ldots$, $a_i\in \R$. Let all 
$a_i$ be bounded above by $C$: $a_i \leq
C$. Prove that this sequence has a limit. Use the definition of real
numbers as Dedekind sections.
\end{zadacha}

\begin{ukazanie} Prove that  $\lim_{i \to \infty} a_i  = \sup\{a_i\}$,
and use the fact that the supremum exists.
\end{ukazanie}

\begin{opredelenie} Let $\{a_i\}= a_0, a_1, a_2, \ldots$ be a sequence
  of real numbers. $\{a_i\}$ is called a {\bf Cauchy sequence} if for
  any $\epsilon >0$ there exists an interval $[x, y]$ of length
  $\epsilon$ which contains all members $\{a_i\}$ except a finite number
  of them.
\end{opredelenie}

\begin{zamechanie} This is the same definition as the definition of
  Cauchy sequences of rational numbers.
\end{zamechanie}

\begin{zadacha} Let a sequence $\{a_i\}$ converge to a real number
  $c$. Prove that this is a Cauchy sequence.
\end{zadacha}

\begin{zadacha} Let a Cauchy sequence  $\{a_i\}$ 
have a subsequence that converges to $x\in \R$. Prove that  $\{a_i\}$
converges to $x$.
\end{zadacha}

\begin{zadacha} Let $\{a_i\}$ be a Cauchy sequence.
Consider the sequence  $\{b_i\}$, $b_i = \inf_{i\geq k}
a_i$. Prove that this infimum is correctly defined and that the
sequence $b_i$ increases.
\end{zadacha}

\begin{zadacha} Consider the previous problem and prove that if the
  sequence $\{b_i\}$ has a limit then $\lim_{i \to \infty} a_i
  =\lim_{i \to \infty} b_i$.
\end{zadacha}

\begin{zadacha}[!] Let $\{a_i\}$ be a Cauchy sequence. Prove that
  $\{a_i\}$ converges. Use the definition of real numbers as Dedekind
  sections.
\end{zadacha}

\begin{ukazanie} Use the previous problem.
\end{ukazanie}

\begin{zadacha}[!] Let $\{a_i\}$ be a Cauchy sequence. Prove that
  $\{a_i\}$ converges. Use the definition of real numbers as Cauchy
  sequences.
\end{zadacha}

\begin{ukazanie} 
Let a real number $\{a_i\}$ be represented by a Cauchy sequence of
rational numbers $a_i(0)$, $a_i(1)$, $a_i(2),\ldots$. Passing to a
suitable subsequence one can suppose that all $a_i$ ($i> n$)
are contained in an interval on length $2^{-n}$ and that all $a_i(j)$ ($j>
m$) are contained in an interval of length $2^{-m}$. Prove that the
sequence $\{a_i(i)\}$ is a Cauchy sequence and that the sequence
$\{a_i\}$ converges to the real number represented by it.
\end{ukazanie}

\begin{zadacha}[!] 
Let $\{ a_i\}$, $\{b_i\}$,
$\{c_i\}$ be converging sequences of real numbers and  $a_i \leq b_i
\leq c_i$ for all $i$.  Suppose that 
$\lim_{i \to \infty} a_i = \lim_{i \to \infty} c_i =x$.  Prove that 
 $\lim_{i \to \infty} b_i =x$.
\end{zadacha}

\begin{zadacha}[*] Let a sequence $\{a_i\}$ converge
to $x$. Prove that $b_j = \frac 1 j
\sum_{i=0}^j a_i$ converges to $x$. Give an example, when $\{b_j\}$
converges but $\{a_i\}$ does not.
\end{zadacha}

%%%%%%%%%%%%%%%%%%%%%%%%%%%%%%%%%%%%%%%%%%%%%%%%
\subs{Series.}
%%%%%%%%%%%%%%%%%%%%%%%%%%%%%%%%%%%%%%%%%%%%%%%%

\begin{opredelenie} Let $\{a_i\}$ be a sequence of real
  numbers. Consider a sequence of partial sums 
$\sum_{i=0}^n a_i$.  If this sequence converges one says that
 {\bf series $\sum_{i=0}^\infty a_i$ converge}. It is denoted by
$\sum_{i=0}^\infty a_i =x$ where
$$ 
   x = \lim_{i \to \infty} \sum_{i=0}^n a_i.
$$
It is often denoted by $\sum a_i =x$.
\end{opredelenie}

\begin{opredelenie} A series $\sum a_i$ {\bf  converges absolutely},
if series $\sum |a_i|$ converges.
\end{opredelenie}

\begin{zadacha}[!] Consider a series $\sum a_i$ which converges
  absolutely. Prove that these series converges.
\end{zadacha}

\begin{zadacha} Consider a series $\sum a_i$ which converges
  absolutely. Let $b_i$ be a sequence of nonnegative numbers such that
  $a_i \geq b_i$. Prove that the series $\sum b_i$ converges absolutely.
\end{zadacha}

\begin{zadacha}[**] Let $a_i$, $b_i$ be sequences of integer numbers
  such that the series $\sum a_i^2$, $\sum b_i^2$ converge. Prove that
  the series $\sum a_ib_i$ converge.
\end{zadacha}

\begin{zadacha}[*] Let $a_i$ be a sequence of positive real
  numbers. Limit of the sequence of products
$$
\lim_{n\to \infty} \prod^n_{i=0} (1+a_i)
$$
is denoted by $\prod^\infty_{i=0} (1+a_i)$. If this limit exists one
says that an infinite product $\prod^\infty_{i=0} (1+a_i)$ converges.
Let the product $\prod^\infty_{i=0} (1+a_i)$ converge. Prove that the series
$\sum^\infty_{i=0} a_i$ converges.
\end{zadacha}

\begin{zadacha}[*] Prove that the infinite product $\prod^\infty_{i=0}
  (1+\frac 1 {3^n})$ converges.
\end{zadacha}

\begin{zadacha}[**] Let a series $\sum a_i$ converge.
Prove that $\prod^\infty_{i=0} (1+a_i)$ converges, as well.
\end{zadacha}

\begin{zadacha}[!] Let $a_0\geq a_1 \geq a_2 \geq \ldots$ be a
  decreasing sequence of positive real numbers converging to
  0. Consider the series $\sum^\infty_{i=0}
(-1)^ia_i$. Prove that these series converges. Such series are called
{\bf sign-alternating}.
\end{zadacha}

\begin{zadacha} Prove that the series $\sum \frac 1{n (n+1)}$
converges.
\end{zadacha}

\begin{ukazanie} Consider the value $\frac{1}{n} - \frac 1{(n+1)}$.
\end{ukazanie}

\begin{zadacha} Prove that the series
$\sum \frac{1}{n^2}$ converges.
\end{zadacha}

\begin{zadacha} Prove that the series $\sum \frac 1{n!}$
converges.
\end{zadacha}

\begin{zadacha}[!] Prove that the series $\sum \frac 1{2^n}$
converges. Calculate the value it converges to.
\end{zadacha}

\begin{zadacha}[*] Prove that the series $\sum_{n=0}^\infty \frac
{x^n}{n!}$ converges for all $x\in \R$.
\end{zadacha}

\begin{zadacha}[**] Consider the series $\sum_{n=0}^\infty \frac
  {x^n}{n!}$ in a complete ordered field $A$. Does this series
  converge for all $x\in A$?
\end{zadacha}
 
\end{document}
