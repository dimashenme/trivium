\documentclass[12pt]{article}

\usepackage{theorem,amsmath,amssymb}



\addtolength{\topmargin}{-23mm}
\addtolength{\textheight}{60mm}
\addtolength{\oddsidemargin}{-20mm}
\addtolength{\textwidth}{40mm}

\def\eqref#1{(\ref{#1})}
\newcommand{\goth}{\mathfrak}
\newcommand{\arrow}{{\:\longrightarrow\:}}
\def\1{\sqrt{-1}\:}
\newcommand{\restrict}[1]{{\left|_{{\phantom{|}\!\!}_{#1}}\right.}}

\renewcommand{\bar}{\overline}
\renewcommand{\phi}{\varphi}
\renewcommand{\epsilon}{\varepsilon}
\renewcommand{\geq}{\geqslant}
\renewcommand{\leq}{\leqslant}

\def\rad{\operatorname{\sf rad}}
\def\tr{\operatorname{\sf tr}}
\def\rk{\operatorname{\sf rk}}
\def\Alt{\operatorname{\sf Alt}}
\def\Sym{\operatorname{\sf Sym}}
\def\Id{\operatorname{\sf Id}}
\def\Hom{\operatorname{Hom}}
\def\Map{\operatorname{Map}}
\def\Gal{\operatorname{Gal}}
\def\Aut{\operatorname{Aut}}
\newcommand{\End}{\operatorname{End}}
\newcommand{\Mat}{\operatorname{Mat}}

\newcommand{\coker}{\operatorname{Coker}}

\def\chpoly{\operatorname{\sf Chpoly}}
\def\minpoly{\operatorname{\sf Minpoly}}

\def\cchar{\operatorname{\sf char}}

\def\Z{{\mathbb Z}}
\def\R{{\mathbb R}}
\def\C{{\mathbb C}}
\def\Q{{\mathbb Q}}
\def\N{{\mathbb N}}
\def\F{{\mathbb F}}

\def\Re{\operatorname{Re}}
\def\Im{\operatorname{Im}}

\makeatletter
\theoremstyle{definition}

\newtheorem{zadacha}{������}[section]
\newtheorem{opredelenie}{�����������}[section]
\newtheorem*{ukazanie}{��������}%[section]
\newtheorem*{zamechanie}{���������}%[section]

%\renewcommand{\labelenumi}{\ralph{enumi}.}
\renewcommand{\labelenumi}{\alph{enumi}.}
\newcommand{\subs}[1]{{\bigskip\centerline{\bf\large #1}\bigskip}}
\newcommand{\sttr}{{\bf(*)}}
\newcommand{\shrk}{{\bf(!)}}
\newcommand{\doublesttr}{{\bf(**)}}

\newcommand{\listok}[2]{%
\setcounter{page}{1}
\renewcommand{\@oddhead}{\hfil #2 \hfil}
\renewcommand{\@evenhead}{\hfil #2 \hfil}
\section*{#2}
\refstepcounter{section}
\setcounter{section}{#1}
}

\@addtoreset{equation}{section}
\renewcommand{\theequation}{\thesection.\arabic{equation}}

\let\oldllim=\lim
\def\lim{\oldllim\limits}
\makeatother


\begin{document}

%%%%%%%%%%%%%%%%%%%%%%%%%%%%%%%%%%%%%%%%%%%%%%%%

\listok{4}{ALGEBRA 4: algebraic numbers}

\subs{Algebraic numbers}

%%%%%%%%%%%%%%%%%%%%%%%%%%%%%%%%%%%%%%%%%%%%%%%%%%%%%%%%%%%%

\begin{opredelenie} Let $k \subset K$ be a field contained in the field $K$
(one says that $k$ is a {\bf subfield} of $K$ and $K$ is an
{\bf extension} of $k$). Element $x\in K$ is {\bf algebraic over
 $k$} if $x$ is a root of a non-zero polynomial with coefficients from
$k$. 
\end{opredelenie}

One often means complex numbers which are algebraic over $\Q$ (that
is, roots of polynomials with rational coefficients)when saying simply
``algebraic numbers'' .

\begin{zadacha} Let $k$ be a subfield in $K$ and $x$ be an element in $K$.
Consider $K$ as a linear space over $k$. Let  $K_x\subset
K$ be a linear subspace of  $K$ generated by the powers of $x$. Prove
that $K_x$ is finite dimensional iff $x$ is algebraic.
\end{zadacha}

\begin{zadacha} Let $k$ be a subfield in $K$, $x$ be an algebraic
  element of $K$ �nd $K_x\subset K$ be a linear subspace generated by
  powers of $x$. Consider an operation $m_v$ of multiplication by a
  non-zero vector $v\in K_x$ defined on $K$. Prove that $m_v$ is a
  $k$-linear mapping that preserves a subspace $K_x\subset K$.
\end{zadacha}

\begin{zadacha} Consider the previous problem, prove that the
  restriction of  $m_v$ on $K_x\subset K$ is invertible.
\end{zadacha}

\begin{zadacha}[!] 
Conclude that $K_x$ is a subfield of $K$.
\end{zadacha}

\begin{opredelenie} {\bf Finite extension}
of a field $k$ is a field $K\supset k$ which is finite dimensional
vector subspace over $k$.
\end{opredelenie}

\begin{zadacha} Let $K_1 \supset K_2 \supset K_3$ be fields such that
  $K_1$ is finite dimensional over $K_2$ which is finite dimensional
  over $K_3$. Prove that  $K_1$ is a finite extension of $K_3$.
\end{zadacha}

\begin{zadacha}[!] 
Conclude that the sum, the product and the factor of elements which
are algebraic over  $k$ are also algebraic over $k$.
\end{zadacha}

\begin{zadacha}
Prove that any finite field is a finite extension of a field of
remainders modulo $p$ for some prime $p$. Conclude that a finite field
has $p^n$ elements (for some $p$, $n$, $p$ is prime).
\end{zadacha}

\begin{zadacha}[*] 
Prove that there exists a non-algebraic complex number.
\end{zadacha}

\begin{zadacha}[**] 
Prove that the number $0,0100100001000000001...$
(there are $2^i$ zeros after the $i$th one) is non-algebraic.
\end{zadacha}

\begin{zadacha}[*] 
Let the complex number $x$ be algebraic. Prove that its conjugate
$\bar x$ is also algebraic.
\end{zadacha}

\begin{ukazanie} Use the fact that complex conjugation is an
  automorphism of $\C$ that preserves $\Q$.
\end{ukazanie}

\begin{zadacha}[*] 
Let the complex number $x=a+b\1$ be algebraic. Prove that 
real numbers  $a$ and $b$ are algebraic.
\end{zadacha}

%%%%%%%%%%%%%%%%%%%%%%%%%%%%%%%%%%%%%%%%%%%%%%%%%%%%%%%%%%%%
\subs{Algebraic closure}
%%%%%%%%%%%%%%%%%%%%%%%%%%%%%%%%%%%%%%%%%%%%%%%%%%%%%%%%%%%%

\begin{zadacha} Let $P(t), Q(t)\in k[t]$ be polynomials of a positive
  degree over a field $k$ which are co-prime. Prove that $1$ can
  be represented as a linear combination of $P$ and $Q$ over $k[t]$:
$$
1 = Q(t) A(t) + P(t) B(t).
$$
\end{zadacha}

\begin{ukazanie} Use the algorithm of Euclid for polynomials.
\end{ukazanie}

\begin{zadacha} Let $P(t)$ be an irreducible polynomial (it cannot be
  represented as a product of polynomials of a positive degree with
  coefficients from $k$) and a product  $Q(t) Q_1(t)$ is divisible by
  $P(t)$ where $Q(t)$, $Q_1(t)$ are non-zero polynomials. Prove that
  either $Q(t)$ or $Q_1(t)$ is divisible by $P(t)$.
\end{zadacha}

\begin{ukazanie} Suppose $Q(t)$ is not divisible by $P(t)$.
Use the previous exercise to represent $1$ as a linear combination of
$Q(t)$ and $P(t)$: 
$$
1 = Q(t) A(t) + P(t) B(t).
$$
Then $1\cdot Q_1(t) = Q(t)Q_1(t) A(t) + P(t) B(t)Q_1(t)$ is divisible
by $P(t)$.
\end{ukazanie}

\begin{zadacha} Let $P(t)$ be a polynomial over $k$.
Consider a ring $k[t]$ of polynomials of $t$ and a quotient space 
$k[t]/Pk[t]$ of all polynomials factored by polynomials that are
divisible by $P$. Prove that $k[t]/Pk[t]$ is a ring (with respect to
naturally defined multiplication and addition).
\end{zadacha}

\begin{zadacha} Prove that multiplication by a polynomial
$Q(t)$ acts on $k[t]/Pk[t]$ as an endomorphism (an endomorphism is a
homomorphism from a space to itself).
\end{zadacha}

\begin{zadacha} Suppose that multiplication by $Q(t)$ maps all elements
$k[t]/Pk[t]$ to zero. Prove that $Q$ is divisible by $P$ in the ring
$k[t]$.
\end{zadacha}

\begin{zadacha} Suppose that $P(t)$ is irreducible. Suppose that
  $Q(t)$ is a polynomial that is not divisible by $P(t)$. Prove that
  the operator $m_Q$ of multiplication by $Q(t)$ on the space
  $k[t]/Pk[t]$ is a monomorphism.
\end{zadacha}

\begin{ukazanie} Suppose $v$ belongs to the kernel of $m_Q$ and
  $Q_1(t)$ is a polynomial representing $v$. Then $Q Q_1$ is divisible
  by $P$ by the previous exercise statement. Use the algorithm of
  Euclid for polynomials to deduce that either $Q$ is divisible by
  $P$ or $Q_1$ is divisible by $P$.
\end{ukazanie}

\begin{zadacha}[*] Let $A:\; V \arrow V$ be a linear operator.  Prove
  that there exists a polynomial $P(t)=t^n + a_n t^{n-1}+...$ such
  that $P(A)=0$. Is it possible in general to find an irreducible
  polynomial $P(t)$ such that $P(A)=0$?
\end{zadacha}

\begin{zadacha}[!] Let $P(t)$ be irreducible. Prove that 
$k[t]/Pk[t]$ is a field.
\end{zadacha}

\begin{ukazanie} Use the previous exercise to prove that if $Q$ is not
  divisible by $P$ then multiplication by $Q(t)$ defines an invertible
  linear operator on $k[t]/Pk[t]$.
\end{ukazanie}

\begin{opredelenie} Let  $P(t)$ be an irreducible polynomial. 
We say that the field $k[t]/Pk[t]$ is an  {\bf extension obtained by
  adding the root $P(t)$}.
\end{opredelenie}

\begin{opredelenie} {\bf Algebraic extension} of a field $k$ is a
  field $K\supset k$ such that all elements of $K$ are algebraic over
  $k$.
\end{opredelenie}

\begin{zadacha} Prove that any finite extension is algebraic.
\end{zadacha}

\begin{zadacha}[*] Prove that not every algebraic extension is finite.
\end{zadacha}

\begin{opredelenie} Let $k$ be a field. The field $k$ is called {\bf
    algebraically complete} if any polynomial of a positive degree $P\in k[t]$
has a root in $k$.
\end{opredelenie}

\begin{opredelenie} {\bf Algebraic closure of a field $k$} is an
  algebraic extension $\bar k \supset k$ which is algebraically complete.
\end{opredelenie}

\begin{zadacha}[*]\label{rt} 
Let $K$ be an extension of the field $k$ and $z\in K$ is a root of a
non-zero polynomial $P(t)$ with coefficients which are algebraic over
$k$. Prove that $z$ is algebraic over $k$.
\end{zadacha}

\begin{zadacha}[*] Suppose $K$ is an algebraic extension of the field
  $k$ such that any polynomial $P(t)\in k[t]$ has a root in $K$. Prove
  that any polynomial  $P(t)\in k[t]$ can be represented as a product
  of linear polynomials from $K[t]$.
\end{zadacha}

\begin{zadacha}[*] Take the statement of the previous exercise and
  prove that $K$ is algebraically complete.
\end{zadacha}

\begin{ukazanie} Let $P\in K[t]$ be an irreducible polynomial with
  coefficients  $K$. Add its root  $\alpha$ to $K$.
Using the exercise~\ref{rt} we obtain that  $\alpha$
is algebraic over $k$. Then $\alpha$ is a root of a polynomial from
$k[t]$. Every such polynomial can be represented as a product $\prod
(t-\alpha_i)$, $\alpha_i\in K$ as follows from the previous exercise.
Deduce that $\alpha \in K$.
\end{ukazanie}

\begin{zadacha}[*]
Prove that any field $k$ has an algebraic closure.
\end{zadacha}

\begin{ukazanie} 
Take any algebraic extension of the field $k$. If it is algebraically
complete then the proof is over. Otherwise there exists a polynomial $P(t)\in
k[t]$ which has no roots in $K$. Add its root to $K$ and obtain a
field $K_1$. Now consider $K_1$ instead of $K$ and prove the statement
for it.  After having applied this argument as many times as it would
be necessary consider the union of all algebraic extensions of $k$. We
have obtained a field that contains a root of any polynomial from
$k[t]$. Use the previous exercise to ensure that this field is
algebraically closed.
\end{ukazanie}

\begin{zadacha}[**] In the proof sketch for the previous exercise we
  have used implicitly the Zorn's lemma. Find a proof for a countable
  field  $k$ that does not use Zorn's lemma and therefore does not
  depend on the axiom of choice.
\end{zadacha}

\begin{zadacha}[**] Can you prove existence of an algebraic closure
  for an arbitrary field without using the axiom of choice?
\end{zadacha}

\begin{zadacha}[**] Prove that algebraic closure of a field is unique
  up to isomorphism.
\end{zadacha}

\end{document}
