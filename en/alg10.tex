\documentclass[12pt]{article}

\usepackage{theorem,amsmath,amssymb}



\addtolength{\topmargin}{-23mm}
\addtolength{\textheight}{60mm}
\addtolength{\oddsidemargin}{-20mm}
\addtolength{\textwidth}{40mm}

\def\eqref#1{(\ref{#1})}
\newcommand{\goth}{\mathfrak}
\newcommand{\arrow}{{\:\longrightarrow\:}}
\def\1{\sqrt{-1}\:}
\newcommand{\restrict}[1]{{\left|_{{\phantom{|}\!\!}_{#1}}\right.}}

\renewcommand{\bar}{\overline}
\renewcommand{\phi}{\varphi}
\renewcommand{\epsilon}{\varepsilon}
\renewcommand{\geq}{\geqslant}
\renewcommand{\leq}{\leqslant}

\def\rad{\operatorname{\sf rad}}
\def\tr{\operatorname{\sf tr}}
\def\rk{\operatorname{\sf rk}}
\def\Alt{\operatorname{\sf Alt}}
\def\Sym{\operatorname{\sf Sym}}
\def\Id{\operatorname{\sf Id}}
\def\Hom{\operatorname{Hom}}
\def\Map{\operatorname{Map}}
\def\Gal{\operatorname{Gal}}
\def\Aut{\operatorname{Aut}}
\newcommand{\End}{\operatorname{End}}
\newcommand{\Mat}{\operatorname{Mat}}

\newcommand{\coker}{\operatorname{Coker}}

\def\chpoly{\operatorname{\sf Chpoly}}
\def\minpoly{\operatorname{\sf Minpoly}}

\def\cchar{\operatorname{\sf char}}

\def\Z{{\mathbb Z}}
\def\R{{\mathbb R}}
\def\C{{\mathbb C}}
\def\Q{{\mathbb Q}}
\def\N{{\mathbb N}}
\def\F{{\mathbb F}}

\def\Re{\operatorname{Re}}
\def\Im{\operatorname{Im}}

\makeatletter
\theoremstyle{definition}

\newtheorem{zadacha}{������}[section]
\newtheorem{opredelenie}{�����������}[section]
\newtheorem*{ukazanie}{��������}%[section]
\newtheorem*{zamechanie}{���������}%[section]

%\renewcommand{\labelenumi}{\ralph{enumi}.}
\renewcommand{\labelenumi}{\alph{enumi}.}
\newcommand{\subs}[1]{{\bigskip\centerline{\bf\large #1}\bigskip}}
\newcommand{\sttr}{{\bf(*)}}
\newcommand{\shrk}{{\bf(!)}}
\newcommand{\doublesttr}{{\bf(**)}}

\newcommand{\listok}[2]{%
\setcounter{page}{1}
\renewcommand{\@oddhead}{\hfil #2 \hfil}
\renewcommand{\@evenhead}{\hfil #2 \hfil}
\section*{#2}
\refstepcounter{section}
\setcounter{section}{#1}
}

\@addtoreset{equation}{section}
\renewcommand{\theequation}{\thesection.\arabic{equation}}

\let\oldllim=\lim
\def\lim{\oldllim\limits}
\makeatother


\begin{document}

%%%%%%%%%%%%%%%%%%%%%%%%%%%%%%%%%%%%%%%%%%%%%%%%

\listok{10}{ALGEBRA 10: normal subgroups and representations}

%%%%%%%%%%%%%%%%%%%%%%%%%%%%%%%%%%%%%%%%%%%%%%%%

%%%%%%%%%%%%%%%%%%%%%%%%%%%%%%%%%%%%%%%%%%%%%%%%
\subsection{Normal subgroups}
%%%%%%%%%%%%%%%%%%%%%%%%%%%%%%%%%%%%%%%%%%%%%%%%

\begin{opredelenie}
  Let $G$ be a group and let $x$, $y$ be its elements. Denote by $x^y$
  the element of the form $y x y^{-1}$. A subgroup $G_1 \subset G$ is
  called {\bf normal}, if for any $x\in G_1$, $y\in G$ it holds that
  $x^y \in G_1$.
\end{opredelenie}

\begin{zadacha} The {\bf centre} of the group $G$ (denoted $Z(G)$) is
  the set of all elements $x\in G$ that commute with all elements of
  $G$. Prove that $Z(G) \subset G$ is a normal subgroup.
\end{zadacha}

\begin{zadacha}
  Let� $G_1 \subset G$ be a subgroup.  {\bf Left cosets} of the
  subgroup $G_1$ are subsets of $G$ the form $G_1 \cdot x\subset G$,
  where $x$ takes all values in $G_1$.  {\bf Right cosets} are subsets
  of $G$ of the form $x\cdot G_1\subset G$. Prove that right (left)
  cosets either intersect or coincide. Prove that right cosets are
  right (and vice versa) if and only if $G_1$ is a normal subgroup.
\end{zadacha}

\begin{zadacha} 
  Let $G_1 \subset G$ be a normal subgroup and let $S_1$, $S_2$ be its
  cosets. Take $x\in S_1$, $y\in S_2$. Prove that the coset of the
  product $xy$ does not depend on the choice of $x, y$ in $S_1,
  S_2$. Prove that the product thus defined makes the set $G_2$ of
  cosets of $G_1$ into a group.
\end{zadacha}

\begin{opredelenie}
In this case one says that
$G_2$ is the  {\bf quotient group of $G$ by $G_1$} (denoted
$G_2 = G/G_1$), and $G$ is an {\bf extension of $G_2$ by $G_1$}.
A group extension is denoted as follows:: $1 \arrow G_1 \arrow G
\arrow G_2\arrow 1$.
\end{opredelenie}

\begin{zadacha}
Let  $G\stackrel \phi \arrow G'$ be a homomorphism of groups.
Prove that the kernel $\phi$ (i.e. the set of elements that are mapped
to  $1_{G'}$) is a normal group in $G$.
\end{zadacha}

\begin{zadacha} 
Let  $G\stackrel \phi \arrow G'$ be a surjective homomorphism of
groups. Prove that $G' \cong G/\ker \phi$ where $\ker \phi$ is the
kernel of $\phi$.
\end{zadacha}

\begin{zadacha} 
Consider the set  $\Aut(G)$ of automorphisms of a group
$G$ with the composition operation. Prove that it is a group. Prove
that the correspondence $\phi_y(x) \mapsto x^y$ defines a homomorphism
 $G \arrow \Aut(G)$.
\end{zadacha}

\begin{opredelenie}
Let $G, G'$ be groups and 
 \[ G\arrow \Aut(G')\] be a homomorphism.
In this case one says that $G$
{\bf acts on  $G'$ by automorphisms}.
Automorphisms of the form  $x \stackrel {\phi_y}\arrow x^y$
are called {\bf inner}. 
\end{opredelenie}

\begin{zadacha}
Find the group $\Aut(G)$ for $G = \Z/p\Z$
($p$ prime).
\end{zadacha}

\begin{zadacha}[*]
Find the group $\Aut(G)$ for $G = \Z/n\Z$
($n$ arbitrary).
\end{zadacha}

\begin{zadacha}
Consider a homomorphism $G_2\stackrel \phi \arrow
\Aut(G_1)$. Define the following operation on the set of pairs $(g_1,
g_2)$:  
$(g_1, g_2)\cdot (h_1, h_2) = (g_1 \phi(g_2, h_1), g_2 h_2)$.
Prove that this defines a group.
\end{zadacha}

\begin{opredelenie}
This group is called a  {\bf semi-direct product of 
$G_1$  and $G_2$} and is denoted $G_1 \rtimes G_2$.
\end{opredelenie}

\begin{zadacha}  
In the previous problem setting 
prove that $(G_1, 1)$ defines a normal subgroup
in $G$ and that the quotient by this subgroup
is isomorphic to $G_2$.
\end{zadacha}

\begin{zadacha} 
Describe the group $S_3$ as a semi-direct product
of two non-trivial Abelian groups.
\end{zadacha}

\begin{zadacha}[!]
Describe the dihedral group
as a semi-direct product of two non-trivial Abelian groups.
\end{zadacha}

\begin{zadacha}[*]
The Klein group is the group of quaternions 
of the form $\pm 1, \pm I, \pm J, \pm K$,
with the natural product.
Is it possible to get the Klein group
as a semi-direct product of two Abelian groups?
\end{zadacha}

\begin{zadacha}[*]
Consider a group extension $1 \arrow G_1 \arrow G
\stackrel \phi \arrow G_2\arrow 1$.
Suppose that  
$G\stackrel \psi \arrow G_1$ is a homomorphism
such that $\psi \circ \phi$ is the identity 
automorphism of  $G_2$ (in this case one says that 
 $\phi$ {\bf admits a section} or \textbf{splits}).
Prove that $G$ is not a semi-direct product
 $G_1 \rtimes G_2$.
\end{zadacha}

\begin{zadacha}[!]
Consider a group $G$.
Consider a subgroup  $[G, G]\subset G$ generated by the 
elements of the form $xy x^{-1}y^{-1}$. Prove that
this is a normal subgroup and the quotient by this subgroup is 
commutative.
\end{zadacha}

\begin{opredelenie}
$[G, G]$ is called the  {\bf commutant} of the group $G$.
\end{opredelenie}

\begin{zadacha}[*]
Find the commutant of the symmetric group.
\end{zadacha}

\begin{zadacha}[!]\label{A5=[A5,A5]}.
Consider the group of even substitutions $A_n$, $n \geq 5$.
Prove that it coincides with its commutant.
\end{zadacha}

\begin{ukazanie}
Compute $xy x^{-1}y^{-1}$ where
$x$, $y$ are cyclic permutations of order  3.
\end{ukazanie}

%%%%%%%%%%%%%%%%%%%%%%%%%%%%%%%%%%%%%%%%%%%%%%%%
\subs{Solvable groups}
%%%%%%%%%%%%%%%%%%%%%%%%%%%%%%%%%%%%%%%%%%%%%%%%

\begin{opredelenie}
A group  $G$ is called {\bf solvable} if
there exists a sequence 
$1= G_n \subset G_{n-1} \subset \dots \subset G_0 = G$
of normal subgroups such that all $G_i/G_{i-1}$ are Abelian.
\end{opredelenie}

\begin{zadacha}
Prove that a subgroup of a solvable group is solvable.
\end{zadacha}

\begin{zadacha} Prove that the symmetric group $S_3$
is solvable.
\end{zadacha}

\begin{zadacha} 
Prove that the symmetric group $S_4$ is solvable.
\end{zadacha}

\begin{zadacha} 
Prove that the Klein group
$\{\pm 1, \pm I, \pm J, \pm K\}$
is solvable.
\end{zadacha}

\begin{zadacha}[!]
Consider a group $G_0$ and its commutant $G_1$, then
$G_2= [G_1,G_1]$ -- the commutant of the commutant and so on, $G_i=
[G_{i-1},G_{i-1}]$. Prove that  $G_0$ is solvable if and only if
at some stage we get $G_n =1$.
\end{zadacha}

\begin{zadacha}[!]
Prove that the group of even permutations  $A_n$, $n \geq 5$
is not solvable.
\end{zadacha}

\begin{zadacha}[*]
Prove that the group of motions of  $\R^3$ is not 
solvable.
\end{zadacha}

\begin{ukazanie}
Construct an isomorphism between $A_5$ and the group of motions
of an icosahedron and use the Problem~\ref{A5=[A5,A5]}.
\end{ukazanie}

\begin{zadacha} 
Consider a group $G$ of order $p^n$.
Prove that the centre of $G$ contains more that one element.
\end{zadacha}

\begin{ukazanie} 
Consider the action of $G$ on itself by automorphisms.
The order of  $G$ equals the sum of cardinalities of classes of the form
 $x^G$ where $x^G$ is the set of all elements of the form
 $x^y$, $y\in G$. First prove that if $x$ is not in the centre then 
the order of $x^G$ is divisible by $p$.
We thus obtain that $|G| = 1 + \sum |y_i^G|$,
and if  $G$ has no centre then all $|y_i^G|$
are divisible by  $p$.
\end{ukazanie}

\begin{zadacha}[!]
Let  $G$ be a group of order  $p^n$. Prove that
$G$ is solvable.
\end{zadacha}

\begin{zadacha}[*] 
Let $G$ be a group of order $p^2$, where $p$ is prime. Prove that
 $G$ is Abelian.
\end{zadacha}

\begin{zadacha}[*]
Give an example of a non-Abelian group of order $p^3$ where $p$ is any
prime number.
\end{zadacha}

\begin{zadacha}[*]
Consider the set  $S$ of all upper-triangular 
matrices  $n\times n$ with unity on the diagonal over the field $k$.
Prove that these matrices form a subgroup in $GL(n,k)$.
Prove that this group is solvable.
Find its order for $k= \Z/p\Z$.
\end{zadacha}


%%%%%%%%%%%%%%%%%%%%%%%%%%%%%%%%%%%%%%%%%%%%%%%%
\subsection{Representations and invariants}
%%%%%%%%%%%%%%%%%%%%%%%%%%%%%%%%%%%%%%%%%%%%%%%%

\begin{opredelenie}
A {\bf representation of a group $G$ on a vector spac}
$V$ is a homomorphism $G \arrow GL(V)$ from $G$ into the group
$GL(V)$ of invertible endomorphisms of $V$. If there is a
representation of $G$ on $V$ one says that $G$ {\bf acts on $V$}. 
A {\bf subrepresentation} $V$ is a subspace
that is preserved under the action of $G$.
\end{opredelenie}

\begin{zadacha} 
Let  $G$  act on vector spaces $V$, $V'$.
Define the action  $G$ on $V\otimes V'$ by the formula
$g(v\otimes v') = g(v)\otimes g(v')$.
Prove that this definition is correct and defines a representation of
$G$ on $V\otimes V'$. 
\end{zadacha}

\begin{opredelenie}
Let  $G$ be a group acting on a vector space
 $V$. A vector $v\in V$ is called 
 {\bf invariant under the action of  $G$} or an {\bf invariant of $G$}
 if   $g(v)=v$  for any  $g\in V$. The space of all  $G$-invariant
 vectors is denoted $V^G$. 
\end{opredelenie}

\begin{zadacha} 
Consider the action of the symmetric group  $S_n$ on  $V=R^n$ defined
by the permutations of coordinates. Find the space of invariants.
\end{zadacha}

\begin{zadacha}[*]
In the previous problem setting find the space of invariants of the
action of  $S_n$ on $V\otimes V$.
\end{zadacha}

\begin{zadacha} 
Consider the action of the cyclic group
$\Z/n\Z$ on $V=R^n$ by the cyclic permutations of coordinates. Find
the space of invariants.
\end{zadacha}

\begin{zadacha}[*]
In the previous problem setting find the space of invariants
$(V\otimes V)^{\Z/n\Z}$ under the action of $\Z/n\Z$ on $V\otimes V$. 
\end{zadacha}

\end{document}
