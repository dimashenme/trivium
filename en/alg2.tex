\documentclass[12pt]{article}

\usepackage{theorem,amsmath,amssymb}



\addtolength{\topmargin}{-23mm}
\addtolength{\textheight}{60mm}
\addtolength{\oddsidemargin}{-20mm}
\addtolength{\textwidth}{40mm}

\def\eqref#1{(\ref{#1})}
\newcommand{\goth}{\mathfrak}
\newcommand{\arrow}{{\:\longrightarrow\:}}
\def\1{\sqrt{-1}\:}
\newcommand{\restrict}[1]{{\left|_{{\phantom{|}\!\!}_{#1}}\right.}}

\renewcommand{\bar}{\overline}
\renewcommand{\phi}{\varphi}
\renewcommand{\epsilon}{\varepsilon}
\renewcommand{\geq}{\geqslant}
\renewcommand{\leq}{\leqslant}

\def\rad{\operatorname{\sf rad}}
\def\tr{\operatorname{\sf tr}}
\def\rk{\operatorname{\sf rk}}
\def\Alt{\operatorname{\sf Alt}}
\def\Sym{\operatorname{\sf Sym}}
\def\Id{\operatorname{\sf Id}}
\def\Hom{\operatorname{Hom}}
\def\Map{\operatorname{Map}}
\def\Gal{\operatorname{Gal}}
\def\Aut{\operatorname{Aut}}
\newcommand{\End}{\operatorname{End}}
\newcommand{\Mat}{\operatorname{Mat}}

\newcommand{\coker}{\operatorname{Coker}}

\def\chpoly{\operatorname{\sf Chpoly}}
\def\minpoly{\operatorname{\sf Minpoly}}

\def\cchar{\operatorname{\sf char}}

\def\Z{{\mathbb Z}}
\def\R{{\mathbb R}}
\def\C{{\mathbb C}}
\def\Q{{\mathbb Q}}
\def\N{{\mathbb N}}
\def\F{{\mathbb F}}

\def\Re{\operatorname{Re}}
\def\Im{\operatorname{Im}}

\makeatletter
\theoremstyle{definition}

\newtheorem{zadacha}{������}[section]
\newtheorem{opredelenie}{�����������}[section]
\newtheorem*{ukazanie}{��������}%[section]
\newtheorem*{zamechanie}{���������}%[section]

%\renewcommand{\labelenumi}{\ralph{enumi}.}
\renewcommand{\labelenumi}{\alph{enumi}.}
\newcommand{\subs}[1]{{\bigskip\centerline{\bf\large #1}\bigskip}}
\newcommand{\sttr}{{\bf(*)}}
\newcommand{\shrk}{{\bf(!)}}
\newcommand{\doublesttr}{{\bf(**)}}

\newcommand{\listok}[2]{%
\setcounter{page}{1}
\renewcommand{\@oddhead}{\hfil #2 \hfil}
\renewcommand{\@evenhead}{\hfil #2 \hfil}
\section*{#2}
\refstepcounter{section}
\setcounter{section}{#1}
}

\@addtoreset{equation}{section}
\renewcommand{\theequation}{\thesection.\arabic{equation}}

\let\oldllim=\lim
\def\lim{\oldllim\limits}
\makeatother


\newcommand{\delit}{\operatorname{\vdots}}

\begin{document}

\listok{2}{ALGEBRA 2: divisibility in rings and Euclid's algorithm}

%%%%%%%%%%%%%%%%%%%%%%%%%%%%%%%%%%%%%%%%%%%%%%%%%%%%%%%%%%%%%%%%%%%%%%%%

\subs{Greatest common divisor}

Let $R$ be a ring.

\begin{opredelenie} {\bf Divisors of zero} in ring $R$ are the
  elements $x$, $y$ such that $xy=0$.  $R$ is called an {\bf integral
    domain} if there are no divisors of zero in $R$.
\end{opredelenie}

Throughout this section all rings are supposed to be integral domains.

\begin{opredelenie}
  An invertible element in $R$ is called a {\bf unit} of ring $R$.
\end{opredelenie}

\begin{zadacha} 
Gauss integers are complex numbers of a form $x + y \1$ where $x$, $y$
are integers. Prove that they form a ring. It is denoted by $\Z[\1]$.
\end{zadacha}

\begin{zadacha} Describe all the unities in the ring of Gauss integers.
\end{zadacha}

\begin{ukazanie} If a complex number $z$ is invertible in $\Z[\1]$ then
  $z \bar z$ is also invertible in $\Z[\1]$.
\end{ukazanie}

\begin{zadacha} Let us fix a positive integer $n$.
Consider a set of all complex numbers of the form $x + y \sqrt{-n}$
where $x$, $y$ are integers. Prove that this is a ring.
\end{zadacha}

\begin{zadacha}[*]
Fix a positive integer $n$.  Consider a set of all complex numbers of
the form $\frac{x + y \sqrt{-3}}2$ where $x$, $y$ are either both even
or both odd.  Prove that this is a ring and describe all unities.  We
will denote this ring by $\widetilde{\Z[\sqrt{-3}]}$.
\end{zadacha}

\begin{opredelenie}
Let $R$ be a ring and $x, y\in R$ be elements of $R$.  If $x= yz$ in
$R$ then one says that $x$ is {\bf divisible} by $y$ in $R$ and $y$
{\bf divides} $x$.  The relation of divisibility is denoted by
$x\delit y$.
\end{opredelenie}

\begin{opredelenie} 
Let $R$ be a ring and $x, y\in R$ be the elements of $R$. {\bf
Greatest common divisor} (GCD) of $x, y$ is an element $z\in R$ such
that $z$ divides $x$ and $y$ and for all $z'$ which divides $x$, $z'$
divides $z$. $x$ and $y$ are called {\bf coprime} if $1$ is the
greatest common divisor of $x,y$.
\end{opredelenie}

Strictly speaking, if one considers an arbitrary ring GCD may not
exist for every pair of elements.

\begin{zadacha}
Prove that if GCD exists then it is unique up to a unit: if $z$ and
$z'$ are greatest common divisors $x$ and $y$ in a ring $R$, then $z =
e z'$, where $e$ is a unit of ring $R$.
\end{zadacha}

\begin{zadacha} Let $\Q(2)$ be a set of all rational numbers,
represented as fractions of the form $\frac p q$ with odd denominator
$q$. Prove that this set is closed under multiplication and addition
and forms a subring in the ring of rational numbers.
\end{zadacha}

\begin{zadacha} Give an example of a non-invertible element in $\Q(2)$.
\end{zadacha}

\begin{zadacha} Describe all unities of the ring $\Q(2)$.
\end{zadacha}

\begin{zadacha}[!] Prove that in $\Q(2)$ for any two elements there
 exists a greatest common divisor of them.\end{zadacha}

\begin{ukazanie} Prove that any element of $\Q(2)$ can be represented
  in the form $e 2^n$, where $e$ is a unit.
\end{ukazanie}

\begin{opredelenie} Let $p$ be an element of a ring $R$. It is called 
{\bf prime}, if for any $q,r$ with $p=qr$ either $q$, or $r$ is a
unit of the ring $R$.
\end{opredelenie}

\begin{zadacha} What are prime elements of $\Q(2)$?
\end{zadacha}

%%%%%%%%%%%%%%%%%%%%%%%%%%%%%%%%%%%%%%%%%%%%%%%%%%%%%%%%%%%%
\subs{Divisibility in the ring of integer numbers}
%%%%%%%%%%%%%%%%%%%%%%%%%%%%%%%%%%%%%%%%%%%%%%%%%%%%%%%%%%%%

\begin{zadacha} Let $x$, $y$ be positive integer numbers and $z= (x -
  ky)$ be the remainder when $x$ is divided by $y$. Prove that if 
GCD($y,z$) exists then GCD($x,y$) exists as well and GCD($x,y$)=
GCD($y,z$).
\end{zadacha}

\begin{opredelenie} 
The Euclid's algorithm takes two positive integer numbers $x$, $y$,
$x>y$ and return a positive integer number $z$.
\begin{enumerate}
\item If $x$ is divisible by $y$ then algorithm stops and returns $y$.
\item If $x$ is not divisible by $y$ then algorithm loops, taking
 numbers $x_1 = y$, $y_1= x - ky$ where $x - ky$ is the remainder when
 $x$ is divided by $y$.
\end{enumerate}
\end{opredelenie}

\begin{zadacha} Prove that the Euclid's algorithm terminates after
 finite number of iterations.
\end{zadacha}

\begin{zadacha} Prove that the number returned by the Euclid's
  algorithm applied to integer numbers $x,y$ is GCD($y,z$) 
\end{zadacha}

\begin{zadacha}
Solve the problem 1.26 from ALGEBRA 1 (unless you have already solved
it).
\end{zadacha}

\begin{zadacha} Prove that the Euclid's algorithm applied to numbers
  $x$, $y$ can be represented as a linear combination of  $x$ with $y$
  integer coefficients: $z = ax + by$.
\end{zadacha}

\begin{zadacha} Let $x$, $y$ be coprime integer numbers and $p$
  be a prime number. Suppose that $xy$ is divisible by $p^\alpha$ for
  some natural number $\alpha$. Prove that either $x$ is divisible by
  $p^\alpha$, or $y$ is divisible by $p^\alpha$.
\end{zadacha}

\begin{zadacha}[!] \label{int} Deduce that prime multipliers decomposition is
  unique: if a positive integer number  $x$ can be represented in two
  ways as a product of prime numbers then these two ways only differ
  by an order of multipliers.
\end{zadacha}

\begin{ukazanie} Present $x$ as a product
$p_i^{\alpha_i}$ where $p_i$ are different prime numbers and use the
  previous problem to prove that $\alpha_i$ can be defined in unique
  fashion.
\end{ukazanie}

%%%%%%%%%%%%%%%%%%%%%%%%%%%%%%%%%%%%%%%%%%%%%%%%%%%%%%%%%%%%
\subs{Unique factorization ring}
%%%%%%%%%%%%%%%%%%%%%%%%%%%%%%%%%%%%%%%%%%%%%%%%%%%%%%%%%%%%

\begin{opredelenie}
  Let $R$ be a ring. Two decompositions of $r\in R$ into prime
  multipliers $r = p_1 p_2 \ldots p_k$, $r = q_1 q_2 \ldots q_k$ are
  called equivalent if $r = q_1 q_2 \ldots q_k$ can be obtained after
  by permuting $p_i$ and by multiplying $p_i$ by ring unit.  It is
  said that $R$ is a {\bf unique factorization ring}, if for any $r\in
  R$ there exists decomposition of $r$ into the product of prime
  elements which is unique up to equivalence.
\end{opredelenie}

\begin{zadacha}[!] Let a ring $R$ admits decomposition into prime
multipliers and for each pair of elements $x, y$ there exists a GCD in
this ring. Let $z$ be represented in $R$ as a linear combination of
$x, y$: $z = ax + by$ where $a,b \in R$. Prove that $R$ is a unique
factoriazation ring.
\end{zadacha}

\begin{ukazanie} Use the hint to the problem~\ref{int}.
\end{ukazanie}

\begin{zadacha} Consider a positive number $n$. Consider a ring
$\Z[\sqrt{-n}] \subset \C$ of complex numbers of the form $z= x + y
\sqrt{-n}$ where $x$ and $y$ are integer. Prove that $|z|^2$ is
integer for all $z\in \Z[\sqrt{-n}]$.
\end{zadacha}

\begin{zadacha} Prove that $z$ is a unit in $\Z[\sqrt{-n}]$
iff $|z|^2=1$.
\end{zadacha}

\begin{ukazanie} $|z^{-1}|^2 = (|z|^2)^{-1}$.
\end{ukazanie}

\begin{zadacha}\label{prime.gauss} 
Let $z$ be an element of $\Z[\sqrt{-n}]$ such that $|z|^2$ is prime in
$\Z$. Prove that $z$ is prime in $\Z[\sqrt{-n}]$.
\end{zadacha}

\begin{ukazanie} $|z z'|^2 = |z|^2 |z'|^2$.
\end{ukazanie}

\begin{zadacha}[!] Consider the ring $\Z[\sqrt{-3}]$.
Prove that $2$ and $1\pm \sqrt{-3}$ are primes. Deduce that
$\Z[\sqrt{-3}]$ is not a unique factorization ring.
\end{zadacha}

\begin{ukazanie} Use the equality $2^2 =4$.
\end{ukazanie}

%%%%%%%%%%%%%%%%%%%%%%%%%%%%%%%%%%%%%%%%%%%%%%%%%%%%%%%%%%%%
\subs{Division with remainder in rings}
%%%%%%%%%%%%%%%%%%%%%%%%%%%%%%%%%%%%%%%%%%%%%%%%%%%%%%%%%%%%

\begin{opredelenie} Let $R$ be a ring.
One says that a ring $R$   {\bf allows division with remainder}  if 
for every pair $x, y$, $y\neq 0$ in $R$ there are elements  $z, k\in
R$ defined such that $z = x - ky$. In this case $z$ is called 
{\bf remainder} and $k$ is called {\bf factor}.
\end{opredelenie}

\noindent {\bf Examples.} The ring integers allows division with
remainder.  The ring of polynomials $k[t]$ over a field $k$ allows
division with remainder as well:
$$
\arraycolsep=0.05em
\begin{array}{rrr@{\,}r|r}
x^2&{}+2x&{}-12&&\,x+5\\
\cline{5-5}
x^2&{}+5x&&&\,x-3\\
\cline{1-2}
&{}-3x&{}-12\\
&{}-3x&{}-15\\
\cline{2-3}
&&3
\end{array}
$$

\begin{opredelenie} Let $R$ be a ring that allows division with
  remainder. {\bf Euclid's algorithm in $R$} is applied to a pair $x,
  y$ of non-zero elements in $R$ and is defined recursively. If $x$ is
  divisible by $y$ Euclid's algorithm stops and returns $y$.  If $x$
  is not divisible by $y$ then Euclid's algorithm is applied to $y,
  z$, where $z$ is a remainder when $x$ is divided by $y$. This
  process can be infinite, a priori.
\end{opredelenie}

\begin{zadacha}[!] Let $R$ be a ring that allows division with
  remainder. Suppose that Euclid's algorithm applied to a pair 
$x, y\in R$ stopped in some finite number of steps and returned $z\in 
  R$. Prove that 
\begin{enumerate}
\item $z = a x + by$ for some $a, b \in R$.

\item $z$ is the greatest common divisor of $x$ and $y$.
\end{enumerate}
\end{zadacha}

\begin{ukazanie} Proof for the arbitrary ring is the same as in the
  case of ring of natural numbers.
\end{ukazanie}

\begin{opredelenie} Let $R$ be a ring. One says that {\bf there
exists a Euclid's algorithm in $R$} or that {\bf $R$ is Euclidean} if
   $R$ allows division with remainder and for all $x, y \in R$
  Euclid's algorithm stops in finite number of steps.
\end{opredelenie}

\begin{zadacha}[!] Let there exists a prime multipliers decomposition
and an Euclid's algorithm in a ring $R$. Prove that $R$ is a unique
factorization ring.
\end{zadacha}

\begin{ukazanie} Use the previous problems.
\end{ukazanie}

\begin{zadacha} Prove that the ring $k[t]$ of polynomials over a field $k$.
\end{zadacha}

\begin{zadacha} Prove that an equation $x \cdot y=0$
has a solution (for $x,y \neq 0$) in $k[t] \mod P$ if and only if a
polynomial $P$ is irreducible.
\end{zadacha}

The integer part $[z]$ of a complex number $z= x + y \1$ is defined as
$[x+0.5] + [y+0.5]\1$ where $[\ ]$ denotes an operation of taking an
integer part of a real number (if one interprets complex numbers as
points on a plane $\R^2$ then $[z]$ is a point with integer
coordinates closest to $z$). Division with remainder in the ring of
Gauss integers $\Z[\1]$ is defined as follows: the factor of $z_1$ and
$z_2$ equals $[\frac {z_1}{z_2}]$ and the remainder equals $z_1 -
[\frac {z_1}{z_2}]z_2$.

\begin{zadacha}
Prove that $\left|z_1 - \left[\frac {z_1}{z_2}\right]z_2\right| <
|z_2|$.
\end{zadacha}

\begin{zadacha} 
Prove that in the ring of Gauss integers $\Z[\1]$ Euclid's algorithm
always terminates.
\end{zadacha}

\begin{ukazanie} Use the previous problem.
Deduce that with every step of the Euclid's algorithm a quantity
$\min(|z_1|^2, |z_2|^2)$ decreases.
\end{ukazanie}

Let $R=\Z[\sqrt{-n}]$ or $R=\widetilde{\Z[\sqrt{-3}]}$. For any $z\in
\C$ let us denote by $[z]_R$ a point of a complex plane corresponding
to point from $R$ closest to $z$. If there are several such points let
us take a point with greatest $Re [z]_R$, if still there are several
such points, let us take one with the greatest $Im [z]_R$. Define the
division of  $z_1$ by $z_2$ with remainder in such a way that the
factor of $z_1$ and $z_2$ is $[\frac{z_1}{z_2}]_R$ and the remainder
is  $z_1 - [\frac {z_1}{z_2}]_Rz_2$.

\begin{zadacha}[*] Prove that if 
$n=1$ then it is the usual division with remainder in $\Z[\1]$
\end{zadacha}

\begin{zadacha}[*] 
Let  $\left|z - [z]_R\right|<1$ for all $z\in \C$. Prove that with
every step of the Euclid's algorithm a quantity 
$|z_2|^2$ decreases.
\end{zadacha}

\begin{zadacha}[*] 
Let for any point $z\in \C$ there exist $r\in R$ such that
$|r-z|<1$. Prove that $R$ is Euclidean.
\end{zadacha}

\begin{zadacha}[*] 
Prove that the following rings are Euclidean: $\Z[\sqrt{-2}]$,
$\widetilde{\Z[\sqrt{-3}]}$.
\end{zadacha}

\begin{zadacha} Decompose the number 2 into prime multipliers in
$\Z[\1]$.
\end{zadacha}

\begin{ukazanie} Use the problem~\ref{prime.gauss}.
\end{ukazanie}

\begin{zadacha}[*] Decompose the numbers 3, 5, 7 into prime
 multipliers in $\Z[\1]$.
\end{zadacha}

\begin{zadacha}[*] Prove that a prime number in $\Z$ of the form $p=4k+3$
is prime in $\Z[\1]$.
\end{zadacha}

\begin{ukazanie} Prove that $p$ cannot be represented as a sum of squares.
\end{ukazanie}

\begin{zadacha} Let $z= a + b \1$ be a Gauss integer which is not
  divisible by $1 + \1$. Suppose that $a$ and $b$ are coprime. Prove
  that $z$ and $\bar z$ are coprime. 
\end{zadacha}

\begin{ukazanie} Prove that if $a$ and $b$ are coprime in $\Z$
  then 2 can be represented as a linear combination $a + b \1$, $a - b
\1$.
\end{ukazanie}

\begin{zadacha}[!] Let $a, b, c$ be coprime numbers such that
  $a^2 + b ^2 = c^2$. Prove that $c= |z|^2$ for some $z\in\Z[\1]$.
\end{zadacha}

\begin{ukazanie} Use the fact that  $c^2 = (a + b \1)(a - b \1)$ and 
 $a, b$ are coprime. Apply the uniqueness of prime multipliers
  decomposition in $\Z[\1]$ and deduct that every prime multiplier of
  $a + b \1$, $a - b \1$ appears twice in the decomposition. 
\end{ukazanie}

\begin{zadacha}[!] Find all triples of integer numbers $a, b, c$ such
  that $a^2 + b ^2 = c^2$ (``find'' means ``write a formula that gives
  all such triples when one substitutes its variables with integer
  numbers'').
\end{zadacha}

\begin{ukazanie} Use the previous problem.
\end{ukazanie}

\begin{zadacha}[*] Find all triples of coprime numbers  $a, b,
c$ such that $a^2 + 2 b ^2 = c^2$.
\end{zadacha}

\begin{zadacha} Use the uniqueness of prime multipliers decomposition
  in $\Z[\sqrt{-2}]$
\end{zadacha}

\begin{zadacha}[**] Find all triples of coprime numbers $a, b,
  c$ such that $a^2 + 3 b ^2 = c^2$.
\end{zadacha}

\end{document}
