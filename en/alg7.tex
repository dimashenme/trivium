
\documentclass[12pt]{article}

\usepackage{theorem,amsmath,amssymb}



\addtolength{\topmargin}{-23mm}
\addtolength{\textheight}{60mm}
\addtolength{\oddsidemargin}{-20mm}
\addtolength{\textwidth}{40mm}

\def\eqref#1{(\ref{#1})}
\newcommand{\goth}{\mathfrak}
\newcommand{\arrow}{{\:\longrightarrow\:}}
\def\1{\sqrt{-1}\:}
\newcommand{\restrict}[1]{{\left|_{{\phantom{|}\!\!}_{#1}}\right.}}

\renewcommand{\bar}{\overline}
\renewcommand{\phi}{\varphi}
\renewcommand{\epsilon}{\varepsilon}
\renewcommand{\geq}{\geqslant}
\renewcommand{\leq}{\leqslant}

\def\rad{\operatorname{\sf rad}}
\def\tr{\operatorname{\sf tr}}
\def\rk{\operatorname{\sf rk}}
\def\Alt{\operatorname{\sf Alt}}
\def\Sym{\operatorname{\sf Sym}}
\def\Id{\operatorname{\sf Id}}
\def\Hom{\operatorname{Hom}}
\def\Map{\operatorname{Map}}
\def\Gal{\operatorname{Gal}}
\def\Aut{\operatorname{Aut}}
\newcommand{\End}{\operatorname{End}}
\newcommand{\Mat}{\operatorname{Mat}}

\newcommand{\coker}{\operatorname{Coker}}

\def\chpoly{\operatorname{\sf Chpoly}}
\def\minpoly{\operatorname{\sf Minpoly}}

\def\cchar{\operatorname{\sf char}}

\def\Z{{\mathbb Z}}
\def\R{{\mathbb R}}
\def\C{{\mathbb C}}
\def\Q{{\mathbb Q}}
\def\N{{\mathbb N}}
\def\F{{\mathbb F}}

\def\Re{\operatorname{Re}}
\def\Im{\operatorname{Im}}

\makeatletter
\theoremstyle{definition}

\newtheorem{zadacha}{������}[section]
\newtheorem{opredelenie}{�����������}[section]
\newtheorem*{ukazanie}{��������}%[section]
\newtheorem*{zamechanie}{���������}%[section]

%\renewcommand{\labelenumi}{\ralph{enumi}.}
\renewcommand{\labelenumi}{\alph{enumi}.}
\newcommand{\subs}[1]{{\bigskip\centerline{\bf\large #1}\bigskip}}
\newcommand{\sttr}{{\bf(*)}}
\newcommand{\shrk}{{\bf(!)}}
\newcommand{\doublesttr}{{\bf(**)}}

\newcommand{\listok}[2]{%
\setcounter{page}{1}
\renewcommand{\@oddhead}{\hfil #2 \hfil}
\renewcommand{\@evenhead}{\hfil #2 \hfil}
\section*{#2}
\refstepcounter{section}
\setcounter{section}{#1}
}

\@addtoreset{equation}{section}
\renewcommand{\theequation}{\thesection.\arabic{equation}}

\let\oldllim=\lim
\def\lim{\oldllim\limits}
\makeatother


\begin{document}

%%%%%%%%%%%%%%%%%%%%%%%%%%%%%%%%%%%%%%%%%%%%%%%%

\listok{7}{ALGEBRA 7: matrices and determinants}

%%%%%%%%%%%%%%%%%%%%%%%%%%%%%%%%%%%%%%%%%%%%%%%%

We suppose that all vector spaces are vector spaces over a field $k$. 

\begin{zadacha}
Let $v_1, \dots, v_n\in V$, $w_1, \dots, w_m\in W$ be bases in 
vector spaces $V$ and $W$.  Consider a homomorphism $e_i^j$ from $V$
to $W$ that maps $v_i$ to $w_j$ and maps $v_k$ to zero for
$k \neq i$. Prove that $e^i_j$ form a basis in the space of
homomorphisms $\Hom(V,W)$. 
\end{zadacha}

\begin{opredelenie}
In the previous problem setting consider a homomorphism $\gamma\in
\Hom(V,W)$. Consider
$\gamma= \gamma^i_j e^i_j$, $\gamma^i_j \in k$.
The matrix
$$
\begin{pmatrix}
\gamma^1_1 & \dots & \gamma^1_n\\
\vdots & \ddots & \vdots\\
\gamma^m_1 & \dots & \gamma^m_n
\end{pmatrix}
$$
is called the {\bf matrix of the homomorphism $\gamma$}.
\end{opredelenie}

\begin{zadacha}
Consider homomorphisms $a\in \Hom(U,V)$, $b\in \Hom(V,W)$
defined by the matrices $(a^i_j)$, $(b^k_l)$. Prove that the
composition of $a$ and $b$ is defined by the matrix $c^i_k =
\sum_{j} a^i_j b^j_k$.
\end{zadacha}

\begin{zamechanie}
Note that the matrix product formula makes sense for matrices of
elements of an arbitrary ring.
\end{zamechanie}

\begin{zadacha}
Consider the space $A$ of square matrices of the size $n\times n$,
with the multiplication $A\times A \arrow A$ defined by the
formula $(a^i_j)\circ (b^k_l)\arrow \sum_{j} a^i_j b^j_k$.
Prove that this is an algebra with unit. Prove that this algebra is
isomorphic to the algebra of linear operators from $k^n$ to $k^n$. 
\end{zadacha}

\begin{opredelenie}
This algebra is called the {\bf matrix algebra} and is denoted
$\Mat(n)$.  The unit element of this algebra (the diagonal matrix
with $a_i^i=1$) is called the {\bf identity matrix} and is denoted 
$\Id$.
\end{opredelenie}

\begin{zadacha}
  Consider a linear operator $f\in\Hom(V,V)$ and let $v_1, \dots, v_n$
  be a basis of $V$ and $(f^i_j)$ be the matrix of $f$.  Consider
  another basis $v_1', \dots, v_n'$ of $V$. Prove that there exists a
  unique operator $g$ that maps $v_i$ to $v_i'$, and $g$ is
  invertible. Let $(g^i_j)$, $((g^{-1})^i_j)$ be the matrices of $g$
  and $g^{-1}$. Prove that $f$ is defined by the matrix $h^i_j:=
  (g^i_j)\circ(f^i_j) \circ((g^{-1})^i_j)$ in the basis $v_1', \dots,
  v_n'$.
\end{zadacha}

\begin{opredelenie}
In that case the matrices $(h^i_j)$, $(f^i_j)$ are said to be {\bf
  equivalent}. 
\end{opredelenie}

\begin{zadacha}
Find all the matrices equivalent to $c\Id$ where $c\in k$.
\end{zadacha}

\begin{zadacha}[!]
Consider a matrix $E(i,j)$ 
$$
\begin{pmatrix}
0 &\dots&0&\dots &0\\
\vdots &\ddots & \vdots &\ddots & \vdots\\
0 & \dots &1 &\dots &0\\
\vdots &\ddots & \vdots &\ddots & \vdots\\
0 &\dots&0&\dots &0
\end{pmatrix},
$$
which has $1$ on the position $i,j$ and has $0$ everywhere else. What
are the values of $i, j$, $i',j'$ that make the matrices $E(i,j)$ and
$E(i',j')$ equivalent?
\end{zadacha}

\begin{zadacha}[!] 
Consider a matrix $A$ which is equivalent to $E(i,j)$.
Prove that all rows of $A$ are proportional.
Prove that all columns of $A$ are proportional.
\end{zadacha}

\begin{zadacha}[*]
Prove that if all rows and columns of $A$ are proportional, then $A$
is equivalent to $E(i,j)$.
\end{zadacha}

\begin{opredelenie}
  Consider a vector space $V$ and an endomorphism $A \in \End(V)$ over it
  (i.e.\ a homomorphism from $V$ to itself) and its dual space $V^*$.
  An operator $A^*:\; V^*\arrow V^*$ that maps a linear functional
  $\gamma\in V^*$ to the linear functional $A^*(\gamma)(v) =
  \gamma(A(v))$ is called a {\bf conjugate operator} for $A$.
\end{opredelenie}

\begin{zadacha}\label{lambda.dual}
  Consider a finite-dimensional vector space $V$ and its dual space
  $V^*$. Construct the natural isomorphism between $\Lambda^k(V)^*$ and
  $\Lambda^k(V^*)$.
\end{zadacha}

\begin{zamechanie}
  ``Natural'' means that it does not require any extra choice (choice
  of base, for example). In this situation, a natural isomorphism
  $\Lambda^k(V)^*\cong\Lambda^k(V^*)$ is permutable with the standard
  action of $GL(V)$ on $\Lambda^k(V)^*$, $\Lambda^k(V^*)$. The spaces
  $V$ and $V^*$ are isomorphic, but one can prove that there is no
  $GL(V)$-invariant isomorphism $V \cong V^*$. In other words it is
  {\it impossible} to construct a natural homomorphism $V \cong V^*$.
\end{zamechanie}

\begin{zadacha}[!]
  Consider a vector space $V$, an endomorphism $A \in \End(V)$ and 
  the conjugate operator $A^*$. Prove that $\det A^* = \det A$.
\end{zadacha}

\begin{ukazanie}
Use the previous problem.
\end{ukazanie}

\begin{opredelenie}
Consider a square matrix $(A^i_j)$ and a matrix $(B^i_j)$, that is
constructed from $(A^i_j)$ by reflecting it over the diagonal: $B^i_j
=A^j_i$.  Then $(B^i_j)$ is called the {\bf transposed matrix} of
$(A^i_j)$, and is denoted $(A^i_j)^\bot$.
\end{opredelenie}

\begin{zadacha}[!]
  Consider a basis $v_1, \dots, v_n$ in $V$ and a dual basis
  $v^1,\dots, v^n$ in $V^*$ ($v^i$ maps $v_i$ to 1 and maps other
  $v_j$s to zero).  Consider an operator $A \in \End(V)$ and its matrix
  $(A^i_j)$. Prove that $A^*$ is given as the matrix $(A^i_j)^\bot$.
\end{zadacha}

\begin{opredelenie}
  Consider a nondegenerate bilinear symmetric form $g$ defined on a
  vector space $V$. An operator $A\in \End(V)$ is called {\bf
    orthogonal with respect to $g$} (or simply {\bf orthogonal}) iff
  $g(Av, Av) = g(v,v)$ for any $v\in V$.
\end{opredelenie}

\begin{zadacha}[!]
Prove that any orthogonal operator is invertible.
\end{zadacha}

\begin{zadacha}[!]
  Consider a linear operator $A\in \End(V)$ on a vector space endowed
  with a nondegenerate bilinear symmetric form $g$.  Identify $V$ and
  $V^*$ using $g$. Then the dual operator $A^*$ can be considered as an
  endomorphism of $V$. Prove that a linear operator $A$ is orthogonal
  iff $A^{-1} = A^*$.
\end{zadacha}

\begin{zadacha}[!]\label{det.orth}
  Prove that the determinant of an orthogonal operator equals $\pm 1$.
\end{zadacha}


\begin{opredelenie}
A nondegenerate bilinear antisymmetric form 
(see ALGEBRA 3) is called a {\bf symplectic form}.
\end{opredelenie}

\begin{zadacha}[*]
  Consider a vector space $V$ with a symplectic form $\omega$ defined
  on it. An operator $A\in \End(V)$ is called {\bf symplectic}, if it
  preserves $\omega$, i.e.\ if $\omega(Av, Av) = \omega(v,v)$.  Prove
  that any symplectic operator has the determinant 1.
\end{zadacha}

\begin{zadacha}[!]
Consider a two-dimensional vector space $V$ over $\R$ and let $A$ be
the matrix
$$
\begin{pmatrix}
a & b\\
c & d
\end{pmatrix}.
$$
Consider the matrix
$$
A'= \begin{pmatrix}
d & b\\
-c & a
\end{pmatrix}.
$$
Prove that $A A'= \Delta \Id$,
where $\Delta\in k$ is the number
$ad-bc$. Prove that $A$ is invertible iff $\Delta\neq 0$.
\end{zadacha}

\begin{zadacha}[!]
In the previous problem setting prove that  $\Delta$ equals to $A$
determinant .
\end{zadacha}

\begin{zadacha} 
Consider a two-dimensional vector space $V$  over  $\R$ endowed with a
positive  bilinear symmetric form. Let $A$ be an orthogonal operator
and its matrix in an orthonormal basis has a form
$$
\begin{pmatrix}
a & b\\
c & d
\end{pmatrix}.
$$
It follows from the Problem~\ref{det.orth} that $\det A = \pm 1$.
\begin{enumerate}
\item Suppose $\det A=1$.
Prove that  $b=-c$, $a =d$ and $a^2+b^2=1$.

\item Suppose $\det A=-1$.  
Prove that  $b=c$, $a =-d$ and $a^2+b^2=1$.
\end{enumerate}
\end{zadacha}

\begin{zadacha}[*] Use the statement of the previous problem to
describe (in terms of 2x2 matrices) the group of movements of a
plane which preserve the origin. Prove that this is a dihedral
group (cf. ALGEBRA 1).
\end{zadacha}

\begin{opredelenie}
  Consider a matrix $(A^i_j)$.  One says that the matrix $(B^i_j)$
  is obtained from $(A^i_j)$ using the {\bf row-wise Gauss
    transformation}, if $(B^i_j) = (A^i_j)\circ E$, where $E$ is the
 matrix either of the following form:
\begin{equation}\label{gauss.1}
\begin{pmatrix}
1 &\hdotsfor{8}\\
\hdotsfor{1} &1 &\hdotsfor{7}\\
\hdotsfor{9}\\
\hdotsfor{3} &0 &\dots &1 &\hdotsfor{3}\\
\vdots &\vdots &\vdots 
&\vdots &\ddots &\vdots 
&\vdots &\vdots &\vdots \\
\hdotsfor{3} &1 &\dots &0 &\hdotsfor{3}\\
\hdotsfor{9}\\
\hdotsfor{7} &1 &\hdotsfor{1}\\
\hdotsfor{8} &1
\end{pmatrix},
\end{equation}
or of the following form:
\begin{equation}\label{gauss.2}
\begin{pmatrix}
1 &\hdotsfor{8}\\
\hdotsfor{1} &1 &\hdotsfor{7}\\
\hdotsfor{9}\\
\hdotsfor{3} &1 &\dots &\lambda &\hdotsfor{3}\\
\vdots &\vdots &\vdots 
&\vdots &\ddots &\vdots 
&\vdots &\vdots &\vdots \\
\hdotsfor{3} &0 &\dots &1 &\hdotsfor{3}\\
\hdotsfor{9}\\
\hdotsfor{7} &1 &\hdotsfor{1}\\
\hdotsfor{8} &1
\end{pmatrix}
\end{equation}
(dots denote zeroes). 
If $(B^i_j) = E \circ (A^i_j)$, where $E$ is as above, then one says
that $(B^i_j)$ is obtained from $(A^i_j)$ using the {\bf column-wise
  Gauss transformation}.
\end{opredelenie}

\begin{zadacha} Prove that the row-wise Gauss transformation can be
described in terms of the following matrix operations: $(B^i_j)$ is
obtained from $(A^i_j)$ by permuting of rows or by adding the
$j$-th row multiplied by $\lambda$ to the $i$-th. What operations
can be used to describe the column-wise Gauss transformation?
\end{zadacha}

\begin{zadacha}
Prove that a matrix of the form  \eqref{gauss.1} has determinant -1
and that a matrix of the form \eqref{gauss.2} has determinant 1.
\end{zadacha}

\begin{zadacha}[!]
Prove that a Gauss transformation of the form \eqref{gauss.2}
does not change the determinant but a transformation of the form
\eqref{gauss.1} multiplies it by -1.
\end{zadacha}

\begin{opredelenie}
A matrix $(A^i_j)$ is called {\bf upper triangular},
if $A^i_j=0$ when $i<j$:
$$
\begin{pmatrix}
*&*&* &\hdotsfor{1} &*&*&*\\
0&*&* &\hdotsfor{1} &*&*&*\\
0&0&* &\hdotsfor{1} &*&*&*\\
\vdots&\vdots&\vdots&
\ddots
&\vdots&\vdots&\vdots\\
0&0&0 &\hdotsfor{1} &*&*&*\\
0&0&0 &\hdotsfor{1} &0&*&*\\
0&0&0 &\hdotsfor{1} &0&0&*
\end{pmatrix}.
$$
A matrix is called 
{\bf diagonal}, if $A^i_j=0$ when $i \neq j$.
\end{opredelenie}

\begin{zadacha}[!]
  Consider an upper triangular matrix $(A^i_j)$ of the size $n \times
  n$.  Prove that $\det (A^i_j)$ equals to the product of all the
  diagonal coefficients:
\[ \det (A^i_j) = \prod_i A^i_i.
\]
\end{zadacha}

\begin{zadacha}\label{gss}
\begin{enumerate}
\item Prove that any matrix can be brought into upper triangular form
  using row-wise Gauss transformations; 
\item Prove that any matrix can be brought into diagonal form
  using row-wise and column-wise Gauss transformations.
\end{enumerate}
\end{zadacha}

\begin{zamechanie}
Since Gauss transformations preserve the determinant (up to $\pm 1$),
one can compute the determinant of a square matrix by bringing it to
diagonal form and multiplying the coefficients on the diagonal.
\end{zamechanie}

\begin{zadacha}[*]
Consider a Euclidean ring $A$ (cf. ALGEBRA 2) such that any element
$a \in A$ admits decomposition into prime factors. Solve the
Problem~\ref{gss} for matrices with elements from $A$.
\end{zadacha}

\begin{ukazanie}
First consider matrices $(a^i_j)$ of the size $1 \times 2$, then
prove the statement by induction for matrices of the size $1 \times
n$ (which is the same as $n \times 1$).
Prove that after the matrix is brought into upper triangular form, the
only non-zero element will be GCD$(a^1_1,\dots,a^1_n)$.
Consider now an arbitrary matrix of the size $m \times n$ and permute
the columns and the rows in such a way that $a^1_1$ be non-zero. 
To prove (b) apply the Gauss transformation to rows, columns then once
more to rows, once more to columns etc. and obtain a matrix where
$a^1_1 \neq 0$ and such that all other elements of the first column
and the first row are zeroes. 
\end{ukazanie}

%%%%%%%%%%%%%%%%%%%%%%%%%%%%%%%%%%%%%%%%%%%%%%%%
\subs{Grassmann algebra and minors of matrices}
%%%%%%%%%%%%%%%%%%%%%%%%%%%%%%%%%%%%%%%%%%%%%%%%


\begin{zadacha}[!]
Consider a basis $v_1, \dots, v_n$ of a vector space $V$, then
$v_{i_1} \wedge v_{i_2} \wedge \dots \wedge v_{i_k}$, 
$i_1< i_2< \dots < i_k$ is the corresponding basis in $\Lambda^k(V)$.
Consider a matrix $A\in \End V$, and $A(i_1, i_2, \dots, i_k;i_1',
i_2', \dots i_k')$, the coefficients of the matrix of the endomorphism
induced by $A$ on $\Lambda^k(V)$ in the basis described above. Prove
that $A(i_1, i_2, \dots, i_k;i_1', i_2', \dots, i_k')$ is the
determinant of the matrix which is obtained from $A$ after all rows
except $i_1$-th, $i_2$-th, $\ldots, i_k$-th and all columns except
$i_1'$-th, $i_2'$-th, $\ldots, i_k'$-th has been removed from it.
\end{zadacha}

\begin{zamechanie}
This determinant is called the {\bf minor} of the matrix $A$.
\end{zamechanie}

\begin{ukazanie}
  Take the composition of $A$ with an operator that maps $v_{i_l}$ to
  $v_{i'_l}$, and reduce the problem to the case $i_l=i'_l$. Prove
  that the coefficients $A(i_1, i_2, \dots, i_k;i_1', i_2', \dots,
  i_k')$ do not depend on rows except the $i_1$-th, $i_2$-th, $\ldots,
  i_k$-th rows, and on columns except the $i_1'$-th, $i_2'$-th,
  $\ldots, i_k'$-th columns. Then put $A^i_j=0$ if $i$ and $j$ do not
  belong to $\{ i_1, i_2, \dots, i_k\}$. Thus you have reduced the
  problem to the case when $V= V_1 \oplus V_2$ and $A$ is of the form
  $B\oplus 0_{V_2}$ where $B\in \End(V_1)$ and $0_{V_2}$ acts on $V_2$
  by mapping all vectors to $0$. In this situation one can apply the
  formula $\Lambda^*(V) = \Lambda^*(V_1)\otimes \Lambda^*(V_2)$ to get
  the desired result.
\end{ukazanie}

\begin{opredelenie}
Consider a linear operator $A\in \End(V)$. Consider the endomorphism
induced by $A$ on $\Lambda^*(V)$. Consider the biggest number $N$
such that this endomorphism is non-zero on $\Lambda^N(V)$. This number 
$N$ is called the {\bf rank of the linear operator $A$} (denoted $\rk
A$). If $A$ is represented by a matrix $(A^i_j)$ then $\rk A$ is
called the rank of this matrix.
\end{opredelenie}

\begin{zadacha}[!]
  Consider an operator $A$ that induces the zero action on
  $\Lambda^k(V)$.  Prove that $A$ induces the zero action on
  $\Lambda^l(V)$ for any $l>k$.
\end{zadacha}

\begin{zadacha}
Prove that the rank of a matrix is the size of its biggest non-zero
minor.
\end{zadacha}

\begin{zadacha} 
Prove that the rank of an operator $A$ is the biggest number $N$ such
that there are vectors $v_1, \dots, v_N$ such that $A(v_1), \dots,
A(v_N)$ are linearly independent. 
\end{zadacha}

\begin{zadacha}[!]
Prove that the rank of  an operator $A$ is the dimension of its image.
\end{zadacha}

\begin{zadacha} 
Consider a matrix of rank 1. Prove that all its rows are
proportional. Prove that all its columns are proportional. 
\end{zadacha}

\begin{zadacha} 
Prove that $\rk A = \rk A^*$.
\end{zadacha}

\begin{ukazanie}
Use the Problem~\ref{lambda.dual}.
\end{ukazanie}

\begin{opredelenie}
A bilinear form
$\mu:\; V_1 \otimes V_2 \arrow k$ is called
{\bf nondegenerate pairing} if for every non-zero $v_1\in V_1$
there is a vector $v_1'\in V_2$ such that 
$\mu(v_1, v_1')\neq 0$ and for any non-zero
$v_2\in V_2$ there is a vector $v_2'\in V_1$ such that
$\mu(v_2, v_2')\neq 0$.
\end{opredelenie}

\begin{zadacha} 
Consider finite-dimensional vector spaces $V_1$, $V_2$.
Prove that a nondegenerate pairing $\mu:\; V_1 \otimes V_2 \arrow k$
defines an isomorphism between $V_1$ and $V_2^*$ and any isomorphism
between those spaces is defined in this way.
\end{zadacha}

\begin{zadacha}[!]
Consider an $n$-dimensional vector space $V$.
Construct the natural isomorphism
\[
\Lambda^k(V)^*\cong\Lambda^{n-k}(V)\otimes \det V^*
\]
($\det V$ denotes the one-dimensional vector space
$\Lambda^{n}(V)$).
\end{zadacha}

\begin{ukazanie}
Use the previous problem.
\end{ukazanie}

\begin{zadacha}
Consider an $n$-dimensional vector space $V$ with the basis
 $v_1, v_2, \dots, v_n$ and an operator $A\in \End V$.
Consider the basis $w_1, w_2, \dots, w_n$ 
in $\Lambda^{n-1}(V)$ where
$w_k= v_1\wedge v_2 \wedge \dots v_{k-1}\wedge v_{k+1}
\wedge \dots$
(there are all $v_i$ in the product except one). 
Consider the matrix $(A^i_j)$ of $A$ and consider 
$\check A^i_j$, the minor that is obtained from 
$A$ after $i$-th row and $j$-th column have been removed. 
Prove that $A$ acts on $\Lambda^{n-1}(V)$ as the matrix 
 $(\check A^i_j)$.
\end{zadacha}

\begin{zadacha} 
In the previous problem setting consider a nondegenerate bilinear
pairing 
\[ 
V \otimes \Lambda^{n-1}(V) \arrow \det V,
\]
defined by the form $v\otimes w \arrow v\wedge w$.  Choose the
isomorphism $k \cong \det V$ such that $v_1\wedge v_2 \wedge \dots
\wedge v_n$ is mapped to $1$. This gives a nondegenerate pairing
defined on $V$ and $\Lambda^{n-1}(V)$. Prove that the basis in
$\Lambda^{n-1}(V)$ dual to $v_1, v_2, \dots, v_n$ is $w_1, -w_2, w_3,
-w_4, \dots $.  Prove that $A$ acts on $\Lambda^{n-1}(V)$ by the
matrix $((-1)^{i+j}\check A^i_j)$ in this basis.
\end{zadacha}

\begin{zadacha} 
Consider a nondegenerate bilinear pairing $\mu:\; V \otimes V'\arrow
k$ and endomorphisms $A\in \End V$ and $B\in \End V'$ such that
$\mu(A v, B v') = \mu(v,  v')$ for all $v, v'\in V, V'$. Choose dual
bases in $V$, $V'$ and suppose $(\alpha^i_j)$ and
$(\beta^i_j)$ are the matrices of $A$ and $B$. Prove that
$(\alpha^i_j)\circ (\beta^i_j)^\bot =\Id$.
\end{zadacha}

\begin{zadacha}[!]
Consider an $A\in \End V$ where $V$ is an $n$-dimensional vector space
with a basis $v_1, v_2, \dots, v_n$ and $(A^i_j)$ is the matrix of the
operator $A$. Prove that $A$ is invertible iff $\det A\neq 0$. 
Prove that
\[
A^{-1} = \frac 1{\det A} ((-1)^{i+j}\check A^i_j)^\bot.
\]
\end{zadacha}

\begin{ukazanie}
Prove that for the natural pairing form
\[ 
V \otimes \Lambda^{n-1}(V) \stackrel \mu \arrow \det V,
\]
it holds that $\mu(A(v), A(w)) = \det A \mu(v, w)$, where $A(w)$
denote the natural action of $A$ on $\Lambda^{n-1}(V)$. Then use the
previous problem for $(A^i_j)=(\alpha^i_j)$, $\frac 1{\det A}
((-1)^i\check A^i_j) = (\beta^i_j)^\bot$.
\end{ukazanie}

\begin{zamechanie}
  We have obtained the well-known formula for calculation of the
  inverse matrix by expansion by minors.  The geometric meaning of
  this formula can be explained as follows: minors of a matrix are (by
  definition) the matrix coefficients of the action of this matrix on
  $\Lambda^{n-1}(V)$ and the natural pairing between $V$ and
  $\Lambda^{n-1}(V)$ is multiplied by $\det A$ by the action of
  $A$. This allows for the calculation of $A^{-1}$ using $\det A$ and
  $\check A$.
\end{zamechanie}

%%%%%%%%%%%%%%%%%%%%%%%%%%%%%%%%%%%%%%%%%%%%%%%%
\subs{Calculation of determinant}
%%%%%%%%%%%%%%%%%%%%%%%%%%%%%%%%%%%%%%%%%%%%%%%%

\begin{zadacha}[!]
Consider the matrix $(A^i_j)$  of a linear operator $A$.
Prove that $\det A$ is equal to 
\[
\sum_{\sigma\in S_n}\operatorname{\sf sgn}(\sigma) A^1_{\sigma_1} 
A^2_{\sigma_2} \dots A^n_{\sigma_n}
\]
where $(\sigma_1, \sigma_2, \dots, \sigma_n)\in S_n$ is a permutation,
the sum is over the elements of the group of all permutations and
$\operatorname{\sf sgn}$ is the sign of the permutation $\sigma$. 
\end{zadacha}

\begin{ukazanie}
Use the explicit formula (one that uses the sum over the elements of
$S_n$) from ALGEBRA 6 for the tensor  
$v_1\wedge v_2 \wedge \dots \wedge v_n$

\end{ukazanie}

\begin{zamechanie} 
The determinant is usually defined using this formula.
\end{zamechanie}

\begin{zadacha}
Consider the matrix $(A^i_j)$ of a linear operator $A$.
Prove that $\det A$ can be calculated as follows:
\[
A^1_1 \check A^1_1 - A^1_2\check A^1_2 + A^1_3\check A^1_3 \dots
\]
where $\check A^i_j$ are minors that are obtained after removing the 
$i$-the row and $j$-th column.
\end{zadacha}

\begin{zamechanie}
This procedure is called {\bf determinant expansion along a row}.
\end{zamechanie}

\begin{zadacha}[*]
(Vandermonde determinant)
Consider the matrix
$$
\begin{pmatrix}
1 &1 &1 &\dots &1\\
t_1 &t_2 &t_3 &\dots &t_n\\
t_1^2 &t_2^2 &t_3^2 &\dots &t_n^2\\
\hdotsfor{5}\\
t_1^{n-1} &t_2^{n-1} &t_3^{n-1} &\dots &t_n^{n-1}
\end{pmatrix},
$$
where $n>1$. Prove that its determinant is
$\prod_{i<j} (t_i-t_j)$.
\end{zadacha}

\begin{zadacha}[*]\label{_P_prod_Q_i_kornej_Zadacha_}
Consider the matrix
$$
\begin{pmatrix}
t &x_1 &x_2 &x_3 &\dots &x_n\\
t^2 &x_1^2 &x_2^2 &x_3^2 &\dots &x_n^2\\
t^4 &x_1^4 &x_2^4 &x_3^4 &\dots &x_n^4\\
\hdotsfor{6}\\
t^{2^{n}} &x_1^{2^{n}} &x_2^{2^{n}} &x_3^{2^{n}} &\dots &x_n^{2^{n}}
\end{pmatrix},
$$
and denote its determinant as $P_n(t, x_1, ... x_n)$. Suppose that
this matrix is over the field $\Z/2\Z$. Prove that $P_n(t, x_1, \dots,
x_n)$ becomes zero if one takes $t=\sum \alpha_i x_i$
to be an arbitrary linear combination of $x_i$. Deduce from B\'ezout's
theorem that
\[ 
 P_n(t, x_1, \dots, x_n) = Q(x_1, \dots, x_n)\prod (t-\sum
 \alpha_i x_i),
\]
where $\alpha_i\in \Z/2\Z$, and $Q \in \Z/2\Z[x_1, \dots, x_n]$ is a
polynomial. 
\end{zadacha}


\begin{ukazanie}
Use long division of $P_n$ by $t-\sum \alpha_i x_i$. If you get a
non-zero value, then if you substitute $t$ for $t=\sum \alpha_i x_i$
in $P(t)$ then you will also get a non-zero value.
\end{ukazanie}

\begin{zadacha}[*]
Prove in the previous problem setting that $Q = P_{n-1}(x_n)$.
\end{zadacha}

\begin{zadacha}[*]
Deduce from the previous problem that
$Q(x_1, \dots, x_n)\neq 0$.
\end{zadacha}

\begin{zadacha}[*]
(Dickson's theorem)
Consider the polynomial 
$$
F_n(t)=\prod (t-\sum
 \alpha_i x_i)\in \Z/2\Z[x_1, \dots, x_n].
$$ 
Prove that 
\[ 
F_n(t) = t^{2^n} + \sum^{n-1}_{i=0} c_{n,i} t^{2^i},
\]
where $c_{n,i}\in \Z/2\Z[x_1, \dots, x_n]$ are polynomials in $x_1,
\dots, x_n$. 
\end{zadacha}

\begin{ukazanie}
Use the previous problem and problem~\ref{_P_prod_Q_i_kornej_Zadacha_}.
\end{ukazanie}

\begin{zamechanie}
Polynomials $c_{n,i}\in \Z/2\Z[x_1, \dots, x_n]$
are called {\bf Dickson's invariants}.
\end{zamechanie}

\begin{zadacha}[*]
Consider the coefficients $Q_r$ (which are $c_{n,i}$ according to
Dickson's theorem) of the polynomial $F_n(t)$ as elements of the
symmetric algebra $S^*(V)$  where $V$ is the vector space over the
field $\Z/2\Z$ with basis $x_1, \dots, x_n$. Consider the action of
the group $GL(V)$ of invertible linear operators on $V$ and extend it
naturally (by multiplicativity) over the symmetric algebra.
Prove that $Q_r$ is invariant with respect to $GL(V)$:
\[
Q_r(x_1, x_2, \dots, x_n) = Q_r(h(x_1), h(x_2), \dots, h(x_n))
\]
where $h\in GL(V)$ is an arbitrary invertible endomorphism.
\end{zadacha}

\begin{zamechanie}
Consider the subring of $GL(V)$-invariant polynomials in the
polynomials ring $S^*(V)$. Dickson (1911) proved that this ring is the
ring of polynomials with generators $c_{n,i}$.
Consult
\begin{quote}
A PRIMER ON THE DICKSON INVARIANTS, 
Contemporary Mathematics 19 (1983), 421-434.
{\tt http://www.math.purdue.edu/$\sim{}$wilker/papers/dickson.pdf}
\end{quote}
for details.
\end{zamechanie}

\end{document}
