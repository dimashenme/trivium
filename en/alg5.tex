\documentclass[12pt]{article}

\usepackage{theorem,amsmath,amssymb}



\addtolength{\topmargin}{-23mm}
\addtolength{\textheight}{60mm}
\addtolength{\oddsidemargin}{-20mm}
\addtolength{\textwidth}{40mm}

\def\eqref#1{(\ref{#1})}
\newcommand{\goth}{\mathfrak}
\newcommand{\arrow}{{\:\longrightarrow\:}}
\def\1{\sqrt{-1}\:}
\newcommand{\restrict}[1]{{\left|_{{\phantom{|}\!\!}_{#1}}\right.}}

\renewcommand{\bar}{\overline}
\renewcommand{\phi}{\varphi}
\renewcommand{\epsilon}{\varepsilon}
\renewcommand{\geq}{\geqslant}
\renewcommand{\leq}{\leqslant}

\def\rad{\operatorname{\sf rad}}
\def\tr{\operatorname{\sf tr}}
\def\rk{\operatorname{\sf rk}}
\def\Alt{\operatorname{\sf Alt}}
\def\Sym{\operatorname{\sf Sym}}
\def\Id{\operatorname{\sf Id}}
\def\Hom{\operatorname{Hom}}
\def\Map{\operatorname{Map}}
\def\Gal{\operatorname{Gal}}
\def\Aut{\operatorname{Aut}}
\newcommand{\End}{\operatorname{End}}
\newcommand{\Mat}{\operatorname{Mat}}

\newcommand{\coker}{\operatorname{Coker}}

\def\chpoly{\operatorname{\sf Chpoly}}
\def\minpoly{\operatorname{\sf Minpoly}}

\def\cchar{\operatorname{\sf char}}

\def\Z{{\mathbb Z}}
\def\R{{\mathbb R}}
\def\C{{\mathbb C}}
\def\Q{{\mathbb Q}}
\def\N{{\mathbb N}}
\def\F{{\mathbb F}}

\def\Re{\operatorname{Re}}
\def\Im{\operatorname{Im}}

\makeatletter
\theoremstyle{definition}

\newtheorem{zadacha}{������}[section]
\newtheorem{opredelenie}{�����������}[section]
\newtheorem*{ukazanie}{��������}%[section]
\newtheorem*{zamechanie}{���������}%[section]

%\renewcommand{\labelenumi}{\ralph{enumi}.}
\renewcommand{\labelenumi}{\alph{enumi}.}
\newcommand{\subs}[1]{{\bigskip\centerline{\bf\large #1}\bigskip}}
\newcommand{\sttr}{{\bf(*)}}
\newcommand{\shrk}{{\bf(!)}}
\newcommand{\doublesttr}{{\bf(**)}}

\newcommand{\listok}[2]{%
\setcounter{page}{1}
\renewcommand{\@oddhead}{\hfil #2 \hfil}
\renewcommand{\@evenhead}{\hfil #2 \hfil}
\section*{#2}
\refstepcounter{section}
\setcounter{section}{#1}
}

\@addtoreset{equation}{section}
\renewcommand{\theequation}{\thesection.\arabic{equation}}

\let\oldllim=\lim
\def\lim{\oldllim\limits}
\makeatother


\begin{document}

%%%%%%%%%%%%%%%%%%%%%%%%%%%%%%%%%%%%%%%%%%%%%%%%%%%%%%%%%%%%

\listok{5}{ALGEBRA 5: Algebras over a field}

%%%%%%%%%%%%%%%%%%%%%%%%%%%%%%%%%%%%%%%%%%%%%%%%%%%%%%%%%%%%

%%%%%%%%%%%%%%%%%%%%%%%%%%%%%%%%%%%%%%%%%%%%%%%%
\subs{Algebras over a field}
%%%%%%%%%%%%%%%%%%%%%%%%%%%%%%%%%%%%%%%%%%%%%%%%

From now on we work with a fixed field $k$. 

Recall that a mapping $(V_1\times V_2) \overset{\mu}{\arrow}
V_2$ of vector spaces is called {\bf bilinear}, if for any
$v_1\in V_1$, $v_2 \in V_2$, the mappings
\[ 
\mu(v_1, \cdot ):\; V_2 \arrow V_3, \ \mu(\cdot, v_2):\; V_1 \arrow V_3
\]
(one argument is fixed, another takes values in $V_2$, $V_1$
respectively) are linear. One says that bilinear mapping is a
mapping which is ``linear in both arguments''.  A symbol of tensor
multiplication is used to denote bilinear mappings, for example, the
mapping just mentioned is denoted
\[ 
\mu:\; V_1\otimes V_2 \arrow V_3.
\]

\begin{opredelenie}
Let $A$ be a vector space over a field $k$ and $\mu:\; A
\otimes A \arrow A$ be a bilinear operation (``multiplication'').  $(A, \mu)$
is called an {\bf algebra over $k$}, if $\mu$ is associative:
\[
\mu(a_1, \mu(a_2, a_3)) = \mu(\mu(a_1, a_2), a_3)).
\]
Multiplication in an algebra is usually denoted like this: $a\cdot
b$. If there is an element $1$ in an algebra such that $\mu(1,a) =
\mu(a,1) =a$ for all $a\in A$ then this element is called a  {\bf
  unit}, and an algebra is called {\bf algebra with unit}. A {\bf
  homomorphism} $r:\; A \arrow A'$ of algebras is a linear mapping
which preserves multiplication and an {\bf isomorphism} of algebras is
an invertible homomorphism.  {\bf Subalgebra} of an algebra is a
linear subspace which is closed under multiplication.
\end{opredelenie}

\begin{zadacha} 
Consider an algebra with unit such that for all $a,b$ it is true that 
$\mu(a,b) = \mu(b,a)$. Prove that this is a (commutative) ring.
\end{zadacha}

\begin{zadacha} 
Give an example of an algebra without a unit.
\end{zadacha}

\begin{zadacha}
Prove that unit is unique.
\end{zadacha}

\begin{zadacha}[*]
Give an example of non-commutative algebra with unit.
\end{zadacha}

\begin{zadacha} 
Let $V$ be a vector space and $\End(V)$ be a space of linear
homomorphisms from $V$ to $V$ with an operation of composition. Prove
that $\End(V)$ is an algebra.
\end{zadacha}

\begin{opredelenie}
$\End(V)$ is called a {\bf matrix algebra} and is denoted $\Mat(V)$. 
\end{opredelenie}

\begin{zadacha}
Is $\Mat(V)$ commutative?
\end{zadacha}

\begin{zadacha}  Consider an isomorphism of matrix algebras
$\Mat(V) \cong \Mat(V')$. 
\begin{enumerate}
\item Suppose that $V$, $V'$ are finite-dimensional.
Prove that $V$, $V'$ are isomorphic. Find a set of all isomorphisms
 $a:\; V \arrow V'$ which are compatible with a given isomorphism 
 $\Mat(V) \cong \Mat(V')$.

\item\sttr Prove that $V\cong V'$ for any $V$, $V'$ (possibly
  infinite- dimensional). Use Zorn's Lemma. \footnote{We mean the
    following statement. Suppose that a system $S$ of subsets of $V$
    is defined, and suppose that it satisfies the following
    conditions:
\begin{enumerate}
\item For any $V_\alpha\in S$ which is not equal to all $V$, there is
  a subset $V_{\alpha'}$ in $S$ which contains $V_\alpha$ but is not
  equal to it.

\item Consider a collection  $S'\subset S$ of subsets of 
 $V$ such that any $V_\alpha, V_{\alpha'}\in S'$ are contained one in
 another: either  $V_\alpha\subset V_{\alpha'}$, or
$V_{\alpha'}\subset V_{\alpha}$. Then a union 
 $V_{\alpha_i}\in S'$ also belongs to $S$
\end{enumerate}
If these conditions hold $V$ is contained in $S$.
Zorn's lemma is a corollary of the axiom of choice.
}

\item\doublesttr Is it possible to prove (b) without axiom of choice?
\end{enumerate}
\end{zadacha}

\begin{zadacha}[!]
Consider an algebra $A$ with unit. Prove that $A$ can be realized as a
subalgebra of $\Mat V$ for some vector space $V$ (possibly infinite
dimensional).
\end{zadacha}

\begin{ukazanie}
Take $V=A$.
\end{ukazanie}

\begin{opredelenie} 
An algebra $A$ with a unit is called  a {\bf division algebra}, if 
$A\backslash 0$ is a multiplication group. In other words $A$
is called a division algebra, if all non-zero elements $A$ are
invertible. 
\end{opredelenie}

\begin{zadacha}\label{quat}
Let $\mathbb H$ be a four dimensional vector space over $\R$ with a
basis $1, I, J, K$. Prove that there exists a unique algebra structure
on $\mathbb H$  such that
\begin{enumerate}
\renewcommand{\labelenumi}{\arabic{enumi}.}
\item $1\cdot a=a$ for all $a\in \mathbb H$,

\item $I^2=J^2=K^2=-1$,

\item $I \cdot J = - J\cdot I =K$.
\end{enumerate}
\end{zadacha}

\begin{opredelenie}
Algebra ${\mathbb H}$ is called a {\bf quaternion algebra}.
\end{opredelenie}

\begin{zadacha}[!] 
Consider ``complex conjunction'' map $z\arrow \bar z$,
defined on ${\mathbb H}$ as follows
\[ 
a + b I + c J + d K \arrow a - b I - c J - d K.
\]
Prove that $\overline{z_1 z_2} = \bar z_2 \bar z_1$.
\end{zadacha}

\begin{zadacha} 
Prove that $z \bar z= a^2 + b^2 + c^2 + d^2$, if $z=a + b I + c
J + d K$.
\end{zadacha}

\begin{zadacha}[!] 
Prove that $\mathbb H$ is a division algebra.
\end{zadacha}

\begin{ukazanie}
Use the argument that was used to prove invertibility of complex
numbers. 
\end{ukazanie}

\begin{zadacha} 
  Replace the equality $I^2=J^2=K^2=-1$ with $I^2=-1$, $J^2=K^2=1$ in
  the statement of the problem~\ref{quat}, replace the second equality
  with $I \cdot J \cdot K=-1$. Prove that you still get an algebra
  structure on $\R^4$ (this algebra is called the {\bf algebra of
    para-quaternions}). Is it a division algebra?
\end{zadacha}

\begin{zadacha}[*]
Prove that the algebra of para-quaternions is isomorphic to $\Mat(\R^2)$.
\end{zadacha}

\begin{zadacha}
Prove that a finite-dimensional algebra $A$ with unity is a division
algebra iff it has no divisors of zero.
\end{zadacha}

%%%%%%%%%%%%%%%%%%%%%%%%%%%%%%%%%%%%%%%%%%%%%%%%%%%%%%%%%%%%%%%%%%%%%%%%
\subs{Algebras defined by generators and relations}
%%%%%%%%%%%%%%%%%%%%%%%%%%%%%%%%%%%%%%%%%%%%%%%%%%%%%%%%%%%%%%%%%%%%%%%%

Consider a vector space $V$ over $k$. A multilinear form
$\phi$ on $V$ is a mapping $V\times V \times V \times \dots
\arrow k$ which is linear in each of its arguments. We denote it like
this:
\[ 
\phi:\; V\otimes V \otimes V \otimes \dots \arrow k.
\]
If $\phi$, $\psi$ are multilinear $i$-form and multilinear $j$-form
then the mapping 
\[ 
\phi\otimes \psi :\; \underbrace{V\times V \times V \times \dots}_{i+j}
   \arrow k,
\]
defined as
\[ 
(\phi\otimes \psi)(v_1, v_2, \dots, v_{i+j}) = \phi( v_1, \dots, v_i) \phi(
   v_{i+1}, \dots, v_{i+j})
\]
is apparently multilinear. This defines multiplication on the space of
multilinear forms.

\begin{zadacha}
Prove that a direct sum $\oplus_i {\cal M}^i V$ of spaces of
$i$-linear forms ${\cal M}^i V$ forms an algebra with respect to
multiplication as defined above.
\end{zadacha}

\begin{zadacha}
  Let $V$ be finite-dimensional. Prove that any element of the algebra
  of multilinear forms can be represented as a linear combination of
  products of elements of $V^*$ (we say that the algebra is {\bf
    generated by $V^*$}).
\end{zadacha}

\begin{opredelenie}
  Let $V$, $W$ be vector spaces over $k$. Consider a space $U= \langle
  V\times W\rangle$, freely generated by pairs of vectors $v, w\in V,
  W$. We will denote vectors from $U$ which correspond to $v, w$ as
  $v\otimes w$. Let us take a quotient of $U$ by a subspace generated
  by the following elements:
\begin{align*}
(\lambda v)\otimes w &- \lambda (v
  \otimes w), \qquad v\otimes (\lambda w ) - \lambda (v
  \otimes w), 
\qquad\qquad \lambda\in k\\
(v+v')\otimes w &- v\otimes w - v'\otimes w, \qquad
   v\otimes (w+w') - v\otimes w - v\otimes w'.
\end{align*}
The quotient space we obtain is called a {\bf tensor product of $V$ and
  $W$} and is denoted $V\otimes W$.
\end{opredelenie}

\begin{zadacha}[!]
  Prove that $(V\otimes W)^*$ is naturally isomorphic to a space of
  bilinear forms over $(V, W)$.
\end{zadacha}

\begin{zadacha}
  Find a number of dimensions of $V\otimes W$ when $\dim V=n$, $\dim
  W=m$.  Prove that $V\otimes W^*$ is naturally isomorphic to a space
  of homomorphisms from $W$ to $V$.
\end{zadacha}

\begin{zadacha}[!]
Prove that  $U\otimes (V\otimes W)$ is canonically isomorphic to
$(U\otimes V)\otimes W$.
\end{zadacha}

\begin{zamechanie}
This statement allows to omit brackets: we write
$U\otimes V\otimes W$ which can be interpreted with any possible
bracketing.
\end{zamechanie}

\begin{zamechanie}
A tensor product $V$ with itself $i$ times is denoted as 
$V^{\otimes i}$. An isomorphism of associativity constructed above
allows to endow $\bigoplus_i V^{\otimes i}$ with associative
multiplication 
\[
V^{\otimes i}\times V^{\otimes j}\arrow V^{\otimes {j+j}}
\]
\end{zamechanie}

\begin{opredelenie}
  The {\bf free} (or {\bf tensor}) algebra, generated by $V$ is an
  algebra $\bigoplus_i V^{\otimes i}$ with multiplication as defined
  above. This algebra is denoted as $T(V)$. $V^{\otimes 0}$ is
  naturally interpreted as $k$. It follows that $T(V)$ is an algebra
  with unit.
\end{opredelenie}

\begin{zadacha}[!]
  Let $V$ be a finite-dimensional vector space. Prove that $T(V)$ is
  isomorphic to the algebra of multilinear forms on $V^*$.
\end{zadacha}

\begin{zadacha}[*]
  Consider a linear mapping from $V$ into some algebra $A$. Prove that
  is can be uniquely extended to a homomorphism of algebras
  $T(V)\arrow A$.
\end{zadacha}

\begin{zadacha}[!]\label{_tensor_monomials_Zadacha_}
  Let $\langle x_i\rangle $ be a basis in $V$. Prove that all the
  monomials of the form $x_{i_1} x_{i_2} x_{i_3}\cdots$ define a basis
  in $T(V)$.
\end{zadacha}

\begin{zadacha}[!]
  Consider a vector space $V$ over $k$ (a ``generator space'') and a
  subspace $W\subset T(V)$ (a ``relations space''). Consider a
  quotient space of $T(V)$ over the space $T(V) W T(V)$ generated by the
  vectors of the form $v w v'$ where $w\in W$. Let this space be
  nonempty. Prove that this quotient space carries a natural structure
  of an algebra with unit.
\end{zadacha}

\begin{opredelenie}
In the previous problem setting let $x_i$ be a basis in $V$ and $w_i$
be a basis in $W$. Every relation $w_i=0$ can written down using a
non-commutative polynomial of the form
\[ 
\sum_I \alpha_{i_1, \dots, i_n} x_{i_1}x_{i_2}\dots =0
\]
where the sum is taken over some set of multiindices and $\alpha_{i_1,
  \dots, i_n}$ are coefficients from the field $k$.  An algebra
$T(V)/T(V) W T(V)$ is called an {\bf algebra defined by generators
  $v_i$ and relations $w_i$}.
\end{opredelenie}

\begin{zadacha}
  Prove that any algebra with unit $A$ can be defined by generators
  and relations. Prove that if $A$ is finite-dimensional then
  generator space $V$ and relation space $W$ can be made
  finite-dimensional too.
\end{zadacha}

\begin{ukazanie}
Take $A=V$.
\end{ukazanie}

\begin{opredelenie}
  An algebra is called {\bf finitely generated} if it can be defined
  by generators and relations in such a way that relations generator
  space is finite-dimensional $V$. An algebra is called {\bf finitely
    presented} if it can be defined in such a way that relations space
  $W$ is also finite-dimensional.
\end{opredelenie}

\begin{zadacha}
Give an example of an algebra that is not finitely presented.
\end{zadacha}

\begin{zadacha}[*]
Is it true that any finitely generated algebra is finitely presented?
\end{zadacha}

\begin{zadacha} 
Prove that algebra $\Mat(\R^2)$ is finitely presented.
\end{zadacha}

\begin{zadacha} 
Define an algebra of Laurent polynomials $k[t, t^{-1}]$ by generators
and relations.
\end{zadacha}

\begin{opredelenie}
Let $V$ be a vector space with a bilinear symmetric form 
$g:\; V \otimes V \arrow \R$.  Consider an algebra  $Cl(V)$,
generated by $V$ and defined by relations of the form
\[ 
v_1\cdot v_2 + v_2\cdot v_1 = g(v_1, v_2)\cdot 1,
\] 
where $v_1, v_2$ passes through $V$. This algebra is called a   
{\bf Clifford algebra} over the field $k$.
\end{opredelenie}

\begin{zadacha}
  Define complex numbers as a Clifford algebra over $\R$.
\end{zadacha}

\begin{zadacha}
Find all Clifford algebras over $\R$ for $\dim V=1,2$.
\end{zadacha}

\begin{zadacha}[!]
Define quaternions and para-quaternions as a Clifford algebra over
$\R$. 
\end{zadacha}

\begin{zadacha}[*]
  Let the dimension of $V$ is $n$.  What is the dimension of $Cl(V)$
  as a vector space?
\end{zadacha}

\begin{zadacha}[**] Define an algebra $\Mat(\R^{2^n})$
as a Clifford algebra.
\end{zadacha}

\end{document}
