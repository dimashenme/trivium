\documentclass[12pt]{article}

\usepackage{theorem,amsmath,amssymb}



\addtolength{\topmargin}{-23mm}
\addtolength{\textheight}{60mm}
\addtolength{\oddsidemargin}{-20mm}
\addtolength{\textwidth}{40mm}

\def\eqref#1{(\ref{#1})}
\newcommand{\goth}{\mathfrak}
\newcommand{\arrow}{{\:\longrightarrow\:}}
\def\1{\sqrt{-1}\:}
\newcommand{\restrict}[1]{{\left|_{{\phantom{|}\!\!}_{#1}}\right.}}

\renewcommand{\bar}{\overline}
\renewcommand{\phi}{\varphi}
\renewcommand{\epsilon}{\varepsilon}
\renewcommand{\geq}{\geqslant}
\renewcommand{\leq}{\leqslant}

\def\rad{\operatorname{\sf rad}}
\def\tr{\operatorname{\sf tr}}
\def\rk{\operatorname{\sf rk}}
\def\Alt{\operatorname{\sf Alt}}
\def\Sym{\operatorname{\sf Sym}}
\def\Id{\operatorname{\sf Id}}
\def\Hom{\operatorname{Hom}}
\def\Map{\operatorname{Map}}
\def\Gal{\operatorname{Gal}}
\def\Aut{\operatorname{Aut}}
\newcommand{\End}{\operatorname{End}}
\newcommand{\Mat}{\operatorname{Mat}}

\newcommand{\coker}{\operatorname{Coker}}

\def\chpoly{\operatorname{\sf Chpoly}}
\def\minpoly{\operatorname{\sf Minpoly}}

\def\cchar{\operatorname{\sf char}}

\def\Z{{\mathbb Z}}
\def\R{{\mathbb R}}
\def\C{{\mathbb C}}
\def\Q{{\mathbb Q}}
\def\N{{\mathbb N}}
\def\F{{\mathbb F}}

\def\Re{\operatorname{Re}}
\def\Im{\operatorname{Im}}

\makeatletter
\theoremstyle{definition}

\newtheorem{zadacha}{������}[section]
\newtheorem{opredelenie}{�����������}[section]
\newtheorem*{ukazanie}{��������}%[section]
\newtheorem*{zamechanie}{���������}%[section]

%\renewcommand{\labelenumi}{\ralph{enumi}.}
\renewcommand{\labelenumi}{\alph{enumi}.}
\newcommand{\subs}[1]{{\bigskip\centerline{\bf\large #1}\bigskip}}
\newcommand{\sttr}{{\bf(*)}}
\newcommand{\shrk}{{\bf(!)}}
\newcommand{\doublesttr}{{\bf(**)}}

\newcommand{\listok}[2]{%
\setcounter{page}{1}
\renewcommand{\@oddhead}{\hfil #2 \hfil}
\renewcommand{\@evenhead}{\hfil #2 \hfil}
\section*{#2}
\refstepcounter{section}
\setcounter{section}{#1}
}

\@addtoreset{equation}{section}
\renewcommand{\theequation}{\thesection.\arabic{equation}}

\let\oldllim=\lim
\def\lim{\oldllim\limits}
\makeatother


\begin{document}

\listok{1}{GEOMETRY 1: real numbers.}

%%%%%%%%%%%%%%%%%%%%%%%%%%%%%%%%%%%%%%%%%%%%%%%%%%%%%%%%%%%%

You are supposed to know what field is, consult ALGEBRA-1 for the
definition.

%%%%%%%%%%%%%%%%%%%%%%%%%%%%%%%%%%%%%%%%%%%%%%%%%%%%%%%%%%%%
\subs{Cauchy sequences.}
%%%%%%%%%%%%%%%%%%%%%%%%%%%%%%%%%%%%%%%%%%%%%%%%%%%%%%%%%%%%

Real numbers are usually considered as something that can be
approximated by rational numbers, for example, one can regard a real
number $a$ as infinite decimal fraction $a_0,a_1a_2\dots$, the finite
fragments of that fraction $a_0,a_1a_2 \dots a_n$ are then
approximations of $a$. Some fractions are declared equivalent, for
example, $1,00000\dots$ and $0,9999 \dots$. It turns out that it is
easier to rigorously define real numbers and operations on them when
not only decimal fractions but just any sequences of rational numbers
which approximate a given real number are considered. And again it
should be taken into account that different sequences can be
equivalent (when they approximate one and the same number). It appears
quite logical to {\bf define} a real number as a set of sequences of
rational numbers that approximate it. This is Cauchy approach to
real numbers definition.

\begin{opredelenie} We will say that something holds for {\bf almost
    all} elements of a set if it holds for all elements except finite
number of them. Let $\{a_i\}= a_0, a_1, a_2, \ldots$ be a sequence of
rational numbers. One says that $\{a_i\}$ is a {\bf Cauchy sequence}
if for any rational number $\epsilon >0$ there exists an interval $[x,
y]$ of length $\epsilon$ which contains almost all $\{a_i\}$.
\end{opredelenie}

\begin{zadacha}
Let $a$ be a rational number. Prove that a constant sequence
$a,a,\dots$ is a Cauchy sequence.
\end{zadacha}

We will denote such a sequence by $\{a\}$.

\begin{zadacha} Let $\{a_i\}$ be a Cauchy sequence. Let us permute
  arbitrarily its elements $a_i$. Prove that we obtain a Cauchy
  sequence then.
\end{zadacha}

\begin{zadacha} Consider a sequence $\{a_i\}$ of rational numbers from
  an interval $I = [a, b]$, $a, b \in \Q$. Prove that one can select a
  subsequence out of $\{a_i\}$ which is a Cauchy subsequence.
\end{zadacha}

\begin{ukazanie} Let us split the interval $I_0=[a, b]$ into two
  equal parts. One of the halves (we will denote it by $I_1$) contains
 an infinite number of elements of the sequence. Let us delete from
 $\{a_i\}$ all elements that do not belong to $I_1$ except $a_0$. Let
 then divide $I_1$ into two equal parts and repeat the procedure over
 and over again. An interval $I_k$ obtained on a $k$-th step contains
 all elements of the sequence starting from $k$-th and this interval
 is of length $\frac{b-a}{2^k}$.
\end{ukazanie}

\begin{zadacha}[!] Consider a monotonically increasing sequence 
$a_1\leq a_2 \leq a_3 \leq\dots$. All $a_i$ are bounded by some
 constant $C$: $a_i \leq C$. Prove that this is a Cauchy sequence.
\end{zadacha}

\begin{ukazanie} Use the previous problem.
\end{ukazanie}

\begin{opredelenie} Let $\{ a_i\}$, $\{b_i\}$ be Cauchy
 sequences. They are called {\bf equivalent} if a sequence $a_0, b_0,
 a_1, b_1, a_2, b_2, \ldots $ is a Cauchy sequence.
\end{opredelenie}

\begin{zadacha}
Let $a$, $b$ be two rational numbers. Prove that $\{a\}$ is equivalent
to $\{b\}$ iff $a=b$.
\end{zadacha}

\begin{zadacha} Prove that a Cauchy sequence is equivalent to any
  subsequence of it.
\end{zadacha}

\begin{zadacha}
Prove that if $\{a_i\}$ is equivalent to $\{b_i\}$ then $\{b_i\}$ is
equivalent to $\{a_i\}$.
\end{zadacha}

\begin{zadacha}[!]\label{otdeleny}
Let $\{a_i\}$, $\{b_i\}$ be two non-equivalent Cauchy sequences. Prove
that there exist two non-intersecting intervals $I_1$, $I_2$ such that
almost all $a_i$ belong to $I_1$ while almost all $b_i$ belong to
$I_2$.
\end{zadacha}

\begin{ukazanie} Apply the definition of a Cauchy sequence with
$\epsilon = \frac{1}{2^n}$ for all $n$.
\end{ukazanie}

\begin{zadacha} [!] Prove that if a sequence $\{a_i\}$
is equivalent to a sequence $\{b_i\}$ and a sequence $\{b_i\}$ is
equivalent to a sequence $\{c_i\}$ then $\{a_i\}$ is equivalent to
$\{c_i\}$ (one says that ``Cauchy sequences equivalence is
transitive'').
\end{zadacha}

\begin{opredelenie}
Let $\{a_i\}$, $\{b_i\}$ be two non-equivalent Cauchy sequences. It is
said that $\{a_i\} > \{b_i\}$ if $a_i > b_i$ for almost all $i$.
\end{opredelenie}

\begin{zadacha}\label{order}
Let $\{a_i\}$, $\{b_i\}$ be two non-equivalent Cauchy sequences. Prove
that either $\{a_i\} < \{b_i\}$ or $\{b_i\} < \{a_i\}$.
\end{zadacha}

\begin{ukazanie}
Use the problem~\ref{otdeleny}.
\end{ukazanie}

\begin{zadacha}\label{inte}
Let $\{a_i\}$, $\{b_i\}$ be two non-equivalent Cauchy sequences and
$\{a_i\} < \{b_i\}$. Prove that there exist two rational numbers $c$,
$d$ such that $\{a_i\} < \{c\} < \{d\} < \{b_i\}$.
\end{zadacha}

\begin{ukazanie}
Use the previous hint.
\end{ukazanie}

\begin{zadacha}
Let $\{a_i\} < \{b_i\}$ and $\{b_i\}$ be equivalent to
$\{c_i\}$. Prove that $\{a_i\} < \{c_i\}$.
\end{zadacha}

\begin{ukazanie}
Use the previous problem and the definition of Cauchy sequence for
$\epsilon < |c-d|$.
\end{ukazanie}

\begin{zadacha} Let $\{ a_i\}$ be a Cauchy sequence and $c\in \Q$ be a
  rational number.  Prove that the following properties are equivalent
\begin{enumerate}
\item $\{ a_i\}$ is equivalent to a sequence $\{c\}$.

\item there are infinitely many elements of a sequence $\{ a_i\}$ in
any open interval $]x, y[$ containing $c$.

\item any open interval $]x, y[$ which contains $c$ contains almost
    all elements of a sequence $\{ a_i\}$ as well.
\end{enumerate}
\end{zadacha}

\begin{opredelenie} If any of these properties holds then one says
    that $\{ a_i\}$ converges to $c$.
\end{opredelenie}

\begin{zadacha}\label{sum} 
Let $\{ a_i\}$, $\{b_i\}$ be a Cauchy sequence. Prove that $\{ a_i +
b_i \}$ and $\{ a_i - b_i \}$ are Cauchy sequences.
\end{zadacha}

\begin{zadacha} Let $\{ a_i\}$, $\{b_i\}$ be Cauchy sequences and
  $b_i$ converges to 0. Prove that $\{ a_i\}$ is equivalent to $\{ a_i
  + b_i \}$.
\end{zadacha}

\begin{zadacha} Let $\{ a_i\}$, $\{b_i\}$ be Cauchy sequences. Prove
  that $\{ a_i b_i \}$ is a Cauchy sequence.
\end{zadacha}

\begin{zadacha} Prove that if $\{b_i\}$ converges to 1 then $\{
a_i b_i \}$ is equivalent to $\{ a_i\}$.
\end{zadacha}

\begin{zadacha}\label{div}
Let $\{ a_i\}$ be a Cauchy sequence which does not contain zeros and
which does not converge to $0$. Prove that $\{ a_i^{-1}\}$ is a Cauchy
sequence.
\end{zadacha}

\begin{ukazanie} Prove that there exists a closed interval $[x, y]$
  which does not contain $0$ such that almost all $\{ a_i\}$ are
  contained in $[x, y]$. Let almost all $\{ a_i\}$ belong to an
  interval $I\subset [x, y]$ of a length $\epsilon$. Prove that all
  $\{ a_i^{-1}\}$ except a finite number belong to an interval
  $I^{-1}$ of a length $\epsilon (\min(|x|, |y|)^{-1}$.
\end{ukazanie}

\begin{opredelenie}
A set of all Cauchy sequences equivalent to a Cauchy sequence
$\{a_i\}$ is called an {\bf equivalence class} of a Cauchy
sequence. The set of all equivalence classes is called a {\bf set of
  real numbers} and is denoted by $\R$.
\end{opredelenie}

\begin{zadacha}
Prove that to correspondence $c \mapsto \{c\}$ defines an injective
mapping from a set $\Q$ of all rational numbers into $\R$.
\end{zadacha}

\begin{zadacha}[!] Prove that four arithmetic operations that we have
  defined on $\R$ in the problems~\ref{sum}-~\ref{div} define on $\R$
  the structure of a field.
\end{zadacha}

\subs{Dedekind sections}

The main disadvantage of defining real numbers using Cauchy sequences
is that there are too many Cauchy sequences and the definition appears
to be too implicit. This difficulty is rather
psychological. Nevertheless, there exists a way to overcome it, it is
to introduce more straightforward definition of real numbers that was
proposed by Dedekind.

\begin{opredelenie}
Let $R \subset \Q$ be a subset of a set of rational numbers which is
non-empty and does not equal to the whole $\Q$. One says said that $R$ is
a {\bf Dedekind section} if $a \in R$ and $b < a$ entails that $b \in
R$. Dedekind section $R$ is called {\bf closed} if there exists a
rational number $a$ such that $b \in R$ as soon as $b \leq
a$. Otherwise $R$ is called {\bf open}.
\end{opredelenie}

Let $\{a_i\}$ be a Cauchy sequence. Let us denote the set of all
rational numbers $b$ such that $\{b\} < \{a_i\}$ by $R_{\{a_i\}}$.

\begin{zadacha}
Prove that $R_{\{a_i\}}$ is a Dedekind section (i.e.\ if $b \in
R_{\{a_i\}}$ and $c < b$ then $c \in R_{\{a_i\}}$). Prove that this
section is open.
\end{zadacha}

\begin{zadacha} Let $\{ a_i\}$ and $\{b_i\}$ be equivalent Cauchy
  sequences. Prove that $R_{\{a_i\}} = R_{\{b_i\}}$.
\end{zadacha}

\begin{zadacha}
Let $\{ a_i\}$ and $\{b_i\}$ be non-equivalent Cauchy sequences and
$\{a_i\} < \{b_i\}$. Prove that $R_{\{a_i\}} \subset
R_{\{b_i\}}$ but those two sets do not coincide.
\end{zadacha}

\begin{ukazanie}
Consider the points of an interval $[c,d]$ from the
problem~\ref{inte}; which of the sets $R_{\{a_i\}}$, $R_{\{b_i\}}$ do
they belong?
\end{ukazanie}
 
\begin{zadacha}[*] Let $\{ a_i\}$, $\{b_i\}$ be two Cauchy
  sequences. Prove that they are equivalent if and only if $R_{\{
  a_i\}} = R_{\{ b_i\}}$.
\end{zadacha}

\begin{ukazanie}
Use the problem~\ref{order} (as well as the preceding problems).
\end{ukazanie}

\begin{zadacha}[*]
Let $R \subset \Q$ be an open Dedekind section. Prove that $R
= R_{\{a_i\}}$ holds for some Cauchy sequence $\{a_i\}$.
\end{zadacha}

\begin{ukazanie}
Consider an interval $I_0=[a,b]$ such that $a$ belongs to $R$ and $b$
does not. Split it into two equal parts, select the half $I_1$ with
the same property. Repeat this process and select any point of $I_i$
as $a_i$.
\end{ukazanie} 

We observe that the set of equivalence classes of Cauchy sequences is
the same thing as the set of open Dedekind sections. That is why the
real numbers can be defined as Dedekind sections. In what follows you
can use the definition that suits you best.

\begin{zadacha}[**]
Define arithmetic operations on $\R$ explicitly on Dedekind sections
without using Cauchy sequences. Check that the axioms of a field hold.
\end{zadacha}

\begin{ukazanie}
To define multiplication define first the operations  ``multiplication
by a positive real number $a$'' and ``multiplication by $-1$'',
then prove distributivity for each of them separately.
\end{ukazanie}

%%%%%%%%%%%%%%%%%%%%%%%%%%%%%%%%%%%%%%%%%%%%%%%%%%%%%%%%%%%%
\subs{Supremum and infimum}
%%%%%%%%%%%%%%%%%%%%%%%%%%%%%%%%%%%%%%%%%%%%%%%%%%%%%%%%%%%%

\begin{opredelenie} Let $A\subset \R$ be some subset of
$\R$. A set $A$ is called {\bf bounded above} if all elements of
$A$ are greater that some constant $C\in \R$.  A set $A$ is called
{\bf bounded below} if all elements of $A$ are less than some constant
$C\in \R$. A set $A$ is called {\bf bounded} if it bounded above and
bounded below.
\end{opredelenie}

\begin{opredelenie} Let $A\subset \R$ be some subset of 
$\R$. Infimum of $A$ (notation: $\inf A$) is by definition a number $c
\in \R$ such that $c \leq a$ for all $a\in A$ and in any open interval
$]x, y[$ containing $c$ there are elements of $a$. Supremum of $A$
(notation: $\sup A$) is by definition a number $c \in \R$ such that $c
\geq a$ for all $a\in A$ and in any open interval $]x, y[$ containing
$c$ there are elements of $a$.
\end{opredelenie}

\begin{zadacha} Prove that $\inf A$ and $\sup A$ are unique (if they
  exist).
\end{zadacha}

\begin{zadacha}[!] Let $A$ be a set bounded above. Prove that $\sup A$
 exists.
\end{zadacha}

\begin{ukazanie} Consider every $a\in A$ as Dedekind sections,
i.e.\ subsets of $\Q$. Consider their union $R$; since every $a \leq C$
this will be Dedekind section too. Prove that $\inf A = R$.
\end{ukazanie}

\begin{zadacha}[!] Let $A \subset \R$ a set bounded below.
Prove that $\inf A$ exists.
\end{zadacha}

\begin{zamechanie} Let $A\subset \R$ is not bounded above (below).
It is denoted by $\inf A = -\infty$ ($\sup A = \infty$).
\end{zamechanie}

\end{document}
