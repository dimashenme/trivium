\documentclass[12pt]{article}

\usepackage{theorem,amsmath,amssymb}



\addtolength{\topmargin}{-23mm}
\addtolength{\textheight}{60mm}
\addtolength{\oddsidemargin}{-20mm}
\addtolength{\textwidth}{40mm}

\def\eqref#1{(\ref{#1})}
\newcommand{\goth}{\mathfrak}
\newcommand{\arrow}{{\:\longrightarrow\:}}
\def\1{\sqrt{-1}\:}
\newcommand{\restrict}[1]{{\left|_{{\phantom{|}\!\!}_{#1}}\right.}}

\renewcommand{\bar}{\overline}
\renewcommand{\phi}{\varphi}
\renewcommand{\epsilon}{\varepsilon}
\renewcommand{\geq}{\geqslant}
\renewcommand{\leq}{\leqslant}

\def\rad{\operatorname{\sf rad}}
\def\tr{\operatorname{\sf tr}}
\def\rk{\operatorname{\sf rk}}
\def\Alt{\operatorname{\sf Alt}}
\def\Sym{\operatorname{\sf Sym}}
\def\Id{\operatorname{\sf Id}}
\def\Hom{\operatorname{Hom}}
\def\Map{\operatorname{Map}}
\def\Gal{\operatorname{Gal}}
\def\Aut{\operatorname{Aut}}
\newcommand{\End}{\operatorname{End}}
\newcommand{\Mat}{\operatorname{Mat}}

\newcommand{\coker}{\operatorname{Coker}}

\def\chpoly{\operatorname{\sf Chpoly}}
\def\minpoly{\operatorname{\sf Minpoly}}

\def\cchar{\operatorname{\sf char}}

\def\Z{{\mathbb Z}}
\def\R{{\mathbb R}}
\def\C{{\mathbb C}}
\def\Q{{\mathbb Q}}
\def\N{{\mathbb N}}
\def\F{{\mathbb F}}

\def\Re{\operatorname{Re}}
\def\Im{\operatorname{Im}}

\makeatletter
\theoremstyle{definition}

\newtheorem{zadacha}{������}[section]
\newtheorem{opredelenie}{�����������}[section]
\newtheorem*{ukazanie}{��������}%[section]
\newtheorem*{zamechanie}{���������}%[section]

%\renewcommand{\labelenumi}{\ralph{enumi}.}
\renewcommand{\labelenumi}{\alph{enumi}.}
\newcommand{\subs}[1]{{\bigskip\centerline{\bf\large #1}\bigskip}}
\newcommand{\sttr}{{\bf(*)}}
\newcommand{\shrk}{{\bf(!)}}
\newcommand{\doublesttr}{{\bf(**)}}

\newcommand{\listok}[2]{%
\setcounter{page}{1}
\renewcommand{\@oddhead}{\hfil #2 \hfil}
\renewcommand{\@evenhead}{\hfil #2 \hfil}
\section*{#2}
\refstepcounter{section}
\setcounter{section}{#1}
}

\@addtoreset{equation}{section}
\renewcommand{\theequation}{\thesection.\arabic{equation}}

\let\oldllim=\lim
\def\lim{\oldllim\limits}
\makeatother


\begin{document}

%%%%%%%%%%%%%%%%%%%%%%%%%%%%%%%%%%%%%%%%%%%%%%%%

\listok{6}{GEOMETRY 6: Point-set topology: product of spaces}

%%%%%%%%%%%%%%%%%%%%%%%%%%%%%%%%%%%%%%%%%%%%%%%%

\begin{opredelenie}
  Consider a topological space $M$ and a collection $B$ of open
  subsets of $M$. The collection $B$ is called a {\bf prebase} of the
  topology on $M$, if every open set can be obtained as a union
  (potentially infinite) of finite intersections of open subsets from
  $B$.
\end{opredelenie}

\begin{zadacha} Consider $\R$ with discrete topology. Prove that it
  does not have a countable prebase.
\end{zadacha}

\begin{zadacha}[!]\label{count}
  Consider a topological space $M$ with a countable prebase. Prove that
  $M$ has a countable base.
\end{zadacha}

\begin{zadacha}[*]
Consider a finite set $M$, $|M|=2^n$ with discrete topology and a
prebase $B$ of $M$. Prove that $|B| \geq 2n$. Find a prebase that has
$2n$ elements.
\end{zadacha}

\begin{zadacha} 
  Consider $\R$ with natural topology and let $B$ be the set of all
  intervals such that their end-points are finite binary
  fractions. Prove that $B$ is a base of topology of $\R$.
\end{zadacha}

\begin{zadacha}
  Consider a collection $B$ of subsets of a set $M$ such that $\cup B=M$.
  Consider all subsets of $M$ that are finite intersections and
  arbitrary unions of elements of $B$, as well as $M$ and $\emptyset$.
  Prove that these sets define a topology on $M$.
\end{zadacha}

\begin{opredelenie}
  This topology is called the {\bf topology defined by the prebase $B$}.
\end{opredelenie}

\begin{opredelenie}
  Consider topological spaces $M_1$ and $M_2$. Consider a topology $S$
  on $M_1 \times M_2$ defined by the prebase of subsets of the form
  $U_1\times M_2$, $M_1\times U_2$ where $U_1$, $U_2$ are open in
  $M_1$, $M_2$ respectively. Then $(M_1\times M_2, S)$ is called the
  {\bf product of $M_1$ and $M_2$}.
\end{opredelenie}

\begin{zadacha}
  Prove that the natural projection $M_1 \times M_2\arrow M_1$ is
  continuous. Prove that sets of the form $U_1\times U_2$ define a
  base of the topology of $M_1 \times M_2$.
\end{zadacha}

\begin{zadacha}\label{_product_nepre_Zadacha_}
Consider mappings of topological space 
$X\overset{\gamma_1}{\arrow} M_1$, $X\overset{\gamma_2}{\arrow}
M_2$. Prove that they are continuous iff the product
\[
X\overset{\gamma_1\times \gamma_2}{\arrow} M_1\times M_2
\]
is continuous.
\end{zadacha}

\begin{zadacha} 
Consider topological spaces $M_1$, $M_2$ that have one of the
properties from the list below. Prove that $M_1\times M_2$ has
the same property.
\begin{enumerate}
\item Separation axiom $T_1$.

\item\shrk{} Hausdorff separation axiom ($T_2$).

\item Separation axiom $T_3$.

\item Being separable.

\item\shrk{} Having a countable neighborhood base for every point.

\item Having a countable base.
\end{enumerate}
\end{zadacha}

\begin{zadacha}[**]
Does this hold for the separation axiom $T_4$? What about $T_4+T_1$?
\end{zadacha}

\begin{opredelenie}
The mapping $x\overset{\Delta}{\arrow} (x,x)\in X\times X$
is called the {\bf diagonal embedding} and its image is called the
{\bf diagonal} in $X\times X$.
\end{opredelenie}

\begin{zadacha}
Prove that the diagonal embedding is a homeomorphism onto its image
(supposing that the topology on $\Delta\subset X\times X$ is induced
from $X\times X$).
\end{zadacha}

\begin{ukazanie}
Use the Problem~\ref{_product_nepre_Zadacha_}.
\end{ukazanie}

\begin{zadacha}
Prove that $X$ satisfies the $T_1$ separation axiom iff the diagonal
is the intersection of all open sets that contain it.
\end{zadacha}

\begin{zadacha}[!]
Prove that $X$ is Hausdorff iff the diagonal is closed in $X\times
X$. 
\end{zadacha}

\begin{zadacha}[*]
Suppose that the graph $\Gamma\subset X\times Y$ of a mapping of
topological spaces $X\overset{\gamma}{\arrow} Y$ is closed. Is it true
that $\gamma$ is continuous?
\end{zadacha}

\begin{zadacha}[!]
  Consider a morphism of topological spaces $X\overset{\gamma}{\arrow}
  Y$ and suppose $X$ is Hausdorff. Prove that the graph of $\gamma$ is
  closed.
\end{zadacha}

\begin{zadacha} 
Consider metric spaces $M_1$, $M_2$ and their product $M=M_1\times
M_2$, and let $d$ be one of the functions defined on $M\times M$
listed below. Prove that $d$ defines a metric on $M$.
\begin{enumerate}
\item $d((m_1, m_2), (m'_1, m'_2)) = d(m_1, m_2) + d(m_1', m_2')$

\item $d((m_1, m_2), (m'_1, m'_2)) = \max(d(m_1, m_2), d(m_1', m_2'))$

\item\shrk{} $d((m_1, m_2), (m'_1, m'_2)) = \sqrt{d(m_1,
m_2)^2+d(m_1', m_2')^2}$
\end{enumerate}
\end{zadacha}

\begin{zadacha}[!] 
  Prove that all the three metric structures from the previous problem
  define the same topology on $M_1\times M_2$. Prove that this
  topology is equivalent to the topology of the product $M_1\times
  M_2$ considered as a product of topological spaces.
\end{zadacha}

%%%%%%%%%%%%%%%%%%%%%%%%%%%%%%%%%%%%%%%%%%%%%%%%
\subs{Tychonoff cube and Hilbert cube}
%%%%%%%%%%%%%%%%%%%%%%%%%%%%%%%%%%%%%%%%%%%%%%%%

\begin{opredelenie}
  Consider a (possibly uncountable) index set $I$ and the set $M=X^I$
  of all mappings from $I$ to a fixed topological space $X$. One can
  regard $X^I$ as a set of sequences of points of $X$ indexed by $I$
  or as an infinite product of $X$ with itself. Denote by $W(i, U)\subset
  X^I$ the set of all mappings $I\arrow X$ that map a fixed index $i$
  to an element from a subset $U\subset X$. Define a prebase $B$ of
  topology on $X^I$ in the following way: let $V\in B$ if $V=W(i, U)$
  for some index element $i\in I$ and some open subset $U\subset
  X$. This topology is called {\bf weak}.
\end{opredelenie}

\begin{zadacha}[!]
Consider a sequence of points $\alpha_1, \alpha_2, \dots $ in
$X^I$. Prove that it converges iff the sequence $\alpha_k(i)$
converges for every index $i\in I$.
\end{zadacha}

\begin{zamechanie}
  The previous problem statement is often expressed as follows: ``a
  space $X^I$ with weak topology is the set of mappings from $I$ to $X$
  with the pointwise convergence topology''.
\end{zamechanie}

\begin{opredelenie}
Consder an index set $I$.  The space  $[0,1]^I$ with the weak topology
is called a {\bf Tychonoff cube}.
\end{opredelenie}

\begin{zadacha} 
Consider a set of continuous functions $\alpha_i:\; M \arrow [0,1]$
indexed by a set $I$.  Prove that the mapping of the form
\[ 
\prod \alpha_i:\; m \arrow \prod_{i\in I}\alpha_i(m)
\]
from $M$ to Tychonoff cube $[0,1]^I$ is continuous. 
\end{zadacha}

\begin{zadacha} 
Prove that any point of a Tychonoff cube is closed.
\end{zadacha}

\begin{zadacha}[*]
Prove that a Tychonoff cube satisfies $T_2$ and $T_3$ separation axioms.
\end{zadacha}

\begin{zadacha}[!]
Consider a Tychonoff cube $[0,1]^I$ where $I$ is countable. Prove that
it has a countable base.
\end{zadacha}

\begin{ukazanie}
  Prove that the collection of all $U=W(i, ]a,b[)$ with $a,b$ rational
  numbers defines a countable prebase in $[0,1]^I$ and use the
  Problem~\ref{count}.
\end{ukazanie}

\begin{zadacha}[**]
  Prove that if the index set $I$ has the cardinality greater than or
  equal to continuum then the Tychonoff cube $[0,1]^I$ is
  non-separable.
\end{zadacha}

\begin{ukazanie}
Consider a countable subset $W$ of a Hausdorff space. Prove that the
cardinality of the closure of $W$ is not greater than continuum.
\end{ukazanie}

\begin{zadacha}[!]
Consider a set $M=[0,1]^\N$ of sequences of real numbers in
$[0,1]$ indexed by $\N$. Consider the function $d:\; M\times M\arrow \R$,
\[ 
d(\{\alpha_i\}, \{\beta_i\})= \sqrt{\sum_i i^{-2}|\alpha_i - \beta_i|^2}.
\]
Prove that this function is well-defined and defines a metric on
$[0,1]^\N$. 
\end{zadacha}

\begin{opredelenie}
A metric space $[0,1]^\N$ with the metric defined as above is called 
{\bf Hilbert cube}.
\end{opredelenie}

\begin{zadacha}[!]
  Consider a sequence $\{\alpha_i(n)\}$ of points of $[0,1]^\N$. Prove
  that it converges in the Tychonoff topology iff it converges in the
  topology of the Hilbert cube.
\end{zadacha}

\begin{zadacha}[*] 
Deduce that the identity mapping is a homeomorphism of the Tychonoff
cube and the Hilbert cube.
\end{zadacha}

\begin{zamechanie}
We actually proved that if the index set $I$ is countable then
the Tychonoff cube $[0,1]^I$ is metrizable.
\end{zamechanie}

\begin{zadacha}[*] 
  Consider a uncountable index set $I$. Is the Tychonoff cube
  $[0,1]^I$ metrizable in that case?
\end{zadacha}

%%%%%%%%%%%%%%%%%%%%%%%%%%%%%%%%%%%%%%%%%%%%%%%%%%%%%%%%%%%%
\subs{Urysohn lemma and metrization of topological spaces}
%%%%%%%%%%%%%%%%%%%%%%%%%%%%%%%%%%%%%%%%%%%%%%%%%%%%%%%%%%%%

\begin{opredelenie}
Consider two non-intersecting closed subsets $A, B\subset M$ of a
topological space $M$. A continuous function  $f:\; M \arrow
[0,1]$ is called an {\bf Urysohn function} if $f(A)=0, f(B)=1$.
\end{opredelenie}

\begin{zadacha}
  Suppose that Urysohn function exists for any two non-intersecting
  closed subsets $A, B\subset M$. Prove that $M$ satisfies the
  separation axiom $T_4$.
\end{zadacha}

\begin{zadacha}
Prove in the previous problem setting that it is possible that $M$
does not satisfy $T_1$ separation axiom.
\end{zadacha}

\begin{zadacha}[*]
Suppose $M$ satisfies $T_4$ separation axiom and $A, B\subset M$ are
non-intersecting and closed. 
Prove that there exists a sequence of neighborhoods 
$U_{p/2^q}\supset A$ indexed by rational numbers of the form
$0<p/2^q<1$ that satisfies the following conditions:
\begin{enumerate}
\renewcommand{\labelenumi}{(\roman{enumi})}
\item for all $p,q$, $B$ does not intersect $U_{p/2^q}$.

\item if $p_1/2^{q_1}< p_2/2^{q_2}$ then the closure of
  $U_{p_1/2^{q_1}}$ is contained in $U_{p_2/2^{q_2}}$.
\end{enumerate}
\end{zadacha}

\begin{ukazanie}
Use an inductive argument.
\end{ukazanie}

\begin{zadacha}[*]
In the previous problem setting define a function $f:\; M \arrow
[0,1]$ to be 
\[  
f(m) = \sup \left\{ p_2/2^{q_2} \ \ | \ \ m\notin U_{p_1/2^{q_1}}\right\}
\]
outside $A$ and equal to zero on $A$.
Prove that $f$ is continuous and that $f$ is an Urysohn function.
\end{zadacha}

\begin{ukazanie} 
Prove that the intervals of the form $]p_1/2^{q_1}, p_2/2^{q_2}[$ form
a prebase of the topology on $[0,1]$. Prove that 
\[ 
   f^{-1}(]p_1/2^{q_1}, p_2/2^{q_2}[) =
   U_{p_2/2^{q_2}}\backslash \overline{U_{p_1/2^{q_1}}}.
\]
Deduce that $f$ is continuous.
\end{ukazanie}

\begin{zamechanie}
We have proven the following ``Urysohn lemma'': if $M$ satisfies the 
$T_4$ condition, then for any two non-intersecting closed subsets of 
$M$ there is an Urysohn function.
\end{zamechanie}

\begin{zadacha}[*]
Consider a Hausdorff space $M$ with a countable base $B$,
which satisfies the $T_4$ condition and let $I$ be a set of all pairs
$U_1, U_2 \in B$ such that the closures of $U_1$, $U_2$ do not
intersect, $F_{U_1, U_2}$ are respective Urysohn functions and $F:\; M
\arrow [0,1]^I$ is a mapping to Tychonoff cube define as $F(m) = \prod
F_{U_1, U_2}$.  Prove that $F$ is continuous and injective.
\end{zadacha}

\begin{zadacha}[*]
In the previous problem setting denote the inverse mapping of $F$ as
$G:\; F(M) \arrow M$. Consider a sequence of points 
$\{x_i\}$ such that $F_{U_1, U_2}(x_i)$ converges for every pair
$(U_1, U_2)$ from $I$. Deduce that the sequence 
$\{x_i\}$ converges. Prove that $G$ is continuous.
\end{zadacha}

\begin{zadacha}[*]
  Prove that any Hausdorff space $M$ with a countable base which
  satisfies the $T_4$ condition (such space is called a {\bf Polish}
  space) is a subspace of a Hilbert cube.
\end{zadacha}

\begin{zamechanie}
We have proved the following {\bf metrization theorem}:
every Polish space is metrizable.
\end{zamechanie}

\begin{zadacha} 
Prove that any subset of a Hilbert cube is Polish.
\end{zadacha}

\begin{zadacha}[*]
Is it true that every metrizable space is a Polish space?
\end{zadacha}

\end{document}
