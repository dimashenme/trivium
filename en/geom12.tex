\documentclass[12pt]{article}

\usepackage{theorem,amsmath,amssymb,}



\addtolength{\topmargin}{-23mm}
\addtolength{\textheight}{60mm}
\addtolength{\oddsidemargin}{-20mm}
\addtolength{\textwidth}{40mm}

\def\eqref#1{(\ref{#1})}
\newcommand{\goth}{\mathfrak}
\newcommand{\arrow}{{\:\longrightarrow\:}}
\def\1{\sqrt{-1}\:}
\newcommand{\restrict}[1]{{\left|_{{\phantom{|}\!\!}_{#1}}\right.}}

\renewcommand{\bar}{\overline}
\renewcommand{\phi}{\varphi}
\renewcommand{\epsilon}{\varepsilon}
\renewcommand{\geq}{\geqslant}
\renewcommand{\leq}{\leqslant}

\def\rad{\operatorname{\sf rad}}
\def\tr{\operatorname{\sf tr}}
\def\rk{\operatorname{\sf rk}}
\def\Alt{\operatorname{\sf Alt}}
\def\Sym{\operatorname{\sf Sym}}
\def\Id{\operatorname{\sf Id}}
\def\Hom{\operatorname{Hom}}
\def\Map{\operatorname{Map}}
\def\Gal{\operatorname{Gal}}
\def\Aut{\operatorname{Aut}}
\newcommand{\End}{\operatorname{End}}
\newcommand{\Mat}{\operatorname{Mat}}

\newcommand{\coker}{\operatorname{Coker}}

\def\chpoly{\operatorname{\sf Chpoly}}
\def\minpoly{\operatorname{\sf Minpoly}}

\def\cchar{\operatorname{\sf char}}

\def\Z{{\mathbb Z}}
\def\R{{\mathbb R}}
\def\C{{\mathbb C}}
\def\Q{{\mathbb Q}}
\def\N{{\mathbb N}}
\def\F{{\mathbb F}}

\def\Re{\operatorname{Re}}
\def\Im{\operatorname{Im}}

\makeatletter
\theoremstyle{definition}

\newtheorem{zadacha}{������}[section]
\newtheorem{opredelenie}{�����������}[section]
\newtheorem*{ukazanie}{��������}%[section]
\newtheorem*{zamechanie}{���������}%[section]

%\renewcommand{\labelenumi}{\ralph{enumi}.}
\renewcommand{\labelenumi}{\alph{enumi}.}
\newcommand{\subs}[1]{{\bigskip\centerline{\bf\large #1}\bigskip}}
\newcommand{\sttr}{{\bf(*)}}
\newcommand{\shrk}{{\bf(!)}}
\newcommand{\doublesttr}{{\bf(**)}}

\newcommand{\listok}[2]{%
\setcounter{page}{1}
\renewcommand{\@oddhead}{\hfil #2 \hfil}
\renewcommand{\@evenhead}{\hfil #2 \hfil}
\section*{#2}
\refstepcounter{section}
\setcounter{section}{#1}
}

\@addtoreset{equation}{section}
\renewcommand{\theequation}{\thesection.\arabic{equation}}

\let\oldllim=\lim
\def\lim{\oldllim\limits}
\makeatother


\begin{document}

%%%%%%%%%%%%%%%%%%%%%%%%%%%%%%%%%%%%%%%%%%%%%%%%%%%%%%%%%%%%

\listok{12}{GEOMETRY 12: fundamental group and homotopies}

%%%%%%%%%%%%%%%%%%%%%%%%%%%%%%%%%%%%%%%%%%%%%%%%%%%%%%%%%%%%

%%%%%%%%%%%%%%%%%%%%%%%%%%%%%%%%%%%%%%%%%%%%%%%%
\subs{Homotopies}
%%%%%%%%%%%%%%%%%%%%%%%%%%%%%%%%%%%%%%%%%%%%%%%%

All topological spaces in this exercise sheet are assumed locally
arcwise connected and Hausdorff, unless the contrary is stated.

\begin{opredelenie}
Let $f_1, f_2:\; X \arrow Y$ be a continuous map of topological
spaces. Recall that a {\bf homotopy}
between $f_1$ and $f_2$ is a continuous map
$F:\; [0,1]\times X \arrow Y$ such that 
$F\restrict{\{0\}\times X}$ equals $f_1$, and
$F\restrict{\{1\}\times X}$ equals $f_2$. 
\end{opredelenie}

\begin{zadacha}
Prove that maps that are homotopic induce the same morphism
$\pi_1(X)\arrow \pi_1(Y)$. 
\end{zadacha}

\begin{opredelenie}
  Let $f:\; X \arrow Y$, $g:\; Y \arrow X$ be continuous maps of
  topological spaces, moreover, $f\circ g$ � $g\circ f$ are homotopic
  to identity maps from $X$ to $X$ and from $Y$ to $Y$.  Such maps are
  called {\bf homotopy equivalences } and $X$ and $Y$ are then called
  {\bf homotopy equivalent}.
\end{opredelenie}

\begin{zadacha} 
Prove that a composition of homotopy equivalence between maps is a
homotopy equivalence. Prove that a homotopy equivalence of spaces is
an equivalence relation.
\end{zadacha}

\begin{zadacha}[!]
Let $f:\; X \arrow Y$ be a homotopy equivalence.
Prove that  $f$ induces an isomorphism of fundamental groups. 
\end{zadacha}

\begin{zadacha}
Let $X\subset Y$ be a retraction. Prove that $X$ are $Y$ homotopy
equivalent. 
\end{zadacha}

\begin{zadacha}[!]
Let $X$ be a topological spaces. Prove that 
$X$ is contractible if and only if it is homotopy equivalent
to a point. 
\end{zadacha}

\begin{zadacha}[!]
Consider the connected graph $\Gamma$ which has
$n$ edges and $n$ vertices. Prove that the associated topological
space is homotopy equivalent to a circle.
\end{zadacha}

\begin{zadacha}[!]
  Let $M$ be a connected topological space and let $x,x', y, y'\in M$
  be any points.  Prove that the spaces of paths $\Omega(M,x,x')$ and
  $\Omega(M,y,y')$ are homotopy equivalent.
\end{zadacha}

\begin{ukazanie} 
  Consider a path $\gamma_{xy}$ that connects $x$ and $y$ and let
  $\gamma_{x'y'}$ be the path that connects $x'$ and $y'$. Let
  $\gamma^{-1}_{xy}(t)=\gamma_{xy}(1-t)$ and
  $\gamma^{-1}_{x'y'}(t)=\gamma_{x'y'}(1-t)$.  Consider the map
  $f:\Omega(M,x,x')\arrow\Omega(M,y,y')$ that maps any path
  $\gamma\in \Omega(M,x,x')$ into the composition
  $\gamma^{-1}_{xy}\gamma\gamma_{x'y'}$, and the analogous map
  $g:\Omega(M,y,y')\arrow\Omega(M,x,x')$ that maps
  $\gamma\in \Omega(M,y,y')$ to $\gamma_{xy}\gamma\gamma^{-1}_{x'y'}$.
  Prove that $fg$ is homotopic to the identity maps and that $gf$ is
  homotopic to the identity map.
\end{ukazanie}

%%%%%%%%%%%%%%%%%%%%%%%%%%%%%%%%%%%%%%%%%%%%%%%%%%%%%%%%%%%%
\subs{The space of paths on locally contractible spaces}
%%%%%%%%%%%%%%%%%%%%%%%%%%%%%%%%%%%%%%%%%%%%%%%%%%%%%%%%%%%%

\begin{opredelenie}
  Let $M$ be a topological space. The space $M$ is called {\bf locally
    contractible } if every point has a contractible neighbourhood.
\end{opredelenie}

\begin{zadacha} Let
$M$ be a locally contractible topological space. 
Prove that  $M$ is locally arcwise connected.
\end{zadacha}

\begin{zadacha}[*]
  Let $M$ be a geodesically connected metric space such that for some
  $\delta>0$ any two points that are at a distance $<\delta$ one from
  another are connected by a unique geodesic. Prove that $M$ is
  locally contractible.
\end{zadacha}

\begin{zadacha}
Prove that any graph is locally contractible.
\end{zadacha}

\begin{opredelenie} 
  A topological space $M$ is called a {\bf manifold of dimension $n$}
  if any point has a neighbourhood that is homeomorphic to an open
  ball in $\R^n$.
\end{opredelenie}

\begin{zamechanie}
Manifolds are easily seen to be locally contractible.
\end{zamechanie}

\begin{zadacha}[!]
Prove that a sphere $S^n$ is a manifold.
\end{zadacha}

\begin{ukazanie} 
Use the stereographic projection.
\end{ukazanie}

\begin{zadacha}
Let $M$ be contractible, $x,y\in M$. Prove that all paths  $\gamma\in
\Omega(M, x,y)$ are homotopic.
\end{zadacha}

\begin{zadacha}[!]
Let $\gamma\in \Omega(M, x,y)$ be a path in a locally contractible
space $M$ and $\{U_\alpha\}$ be the set of contractible open sets in 
 $M$. Choose a finite set in $\{U_\alpha\}$ such that it covers
 $\gamma$ (this can be done since $\gamma$ is compact). Let
 $V_1,\dots,V_n$ be the corresponding cover of 
$[0,1]$ with connected intervals where every $V_i$ is a connected
component of $\gamma^{-1}(U_i)$, and all $U_i$ are contractible. Order
$V_i$ in such a way that  $V_i$ and $V_{i+1}$ intersect at a point
$t_i$, and let $a_i := \gamma(t_i)$. Prove that any path
$\gamma'\in \Omega(M, x,y)$ such that $\gamma'(t_i)=a_i$, and
$\gamma'([t_i, t_{i+1}])\subset U_i$, is homotopic to $\gamma$.
\end{zadacha}

\begin{ukazanie} Use the previous exercise.
\end{ukazanie}

\begin{zadacha}[!] \label{_gomoto_bli_petli_Zadacha_}
Let $M$ be a locally contractible topological space, and $\gamma\in
\Omega(M, x,y)$ is a path. Prove that $\gamma$ has a neighbourhood 
${\mathcal U}\subset \Omega(M, x,y)$ such that all
$\gamma'\in {\mathcal U}$ are homotopic.
\end{zadacha}

\begin{ukazanie}
Use the previous problem.
\end{ukazanie}

\begin{zamechanie} Notice that on all compact manifolds of dimension
  $>1$ there are loops that are defined by a surjective map. Such
  loops can be constructed in the same way as the Peano curve.
\end{zamechanie}

\begin{zadacha}[!]
Let $M$ be a manifold (for instance, a sphere) 
of dimension greater than $1$, and $\gamma\in \Omega(M, x)$ be a
loop. Prove that $\gamma$ is homotopic to a loop that is not surjective.
\end{zadacha}

\begin{ukazanie}
Use the previous exercise.
\end{ukazanie}

\begin{zadacha}[!]
Let $n>1$. Prove that $n$-dimensional sphere is simply connected.
\end{zadacha}

\begin{ukazanie}
  Let $\gamma$ be a loop on a sphere.  Use the previous exercise and
  find a homotopy from $\gamma$ to a loop that maps $[0,1]$ to
  $S^n\backslash \{x\}$ where $x$ is some point. Prove that a sphere
  without a point is homeomorphic to $\R^n$, and in particular is
  contractible.
\end{ukazanie}

\begin{zadacha}[*] \label{_puti_iz_styagi_Zadacha_}
Let $M$ be contractible and let
$F:\; M\times [0,1]\arrow M$ be a homotopy from the identity map to
the constant map $M \to y \in M$. Consider the following map
$M\arrow \Omega(M, y, *)$, $t, m \arrow F(m, t)$ 
($t\in[0,1]$, $m\in M$). Prove that it is continuous.
\end{zadacha}

\begin{zadacha} 
Let $M$  be locally contractible, $x, y\in M$ be two points and
$\gamma\in \Omega(M, x, y)$ be a path. Prove that 
 $\gamma$ has a neighbourhood ${\mathcal U}\in \Omega(M, x, *)$,
such that all paths $\gamma'\in {\mathcal U}$ that connect
$x$ and $a$ are homotopic in $\Omega(M, x, a)$.
\end{zadacha}

\begin{zadacha}[*]
Let $M$ be a locally contractible topological space, and let $x \in M$
be a point, and $\Omega(M, x, *)$ be the set of all paths that start
at the point $x$ endowed with the compact-open topology.
Consider the equivalence relation on 
$\Omega(M,x, *)$: $\gamma\sim\gamma'$ if 
$\gamma$ and $\gamma'$ connect $x$ and $y$,
and homotopic in $\Omega(M, x, y)$. Consider
$\Omega(M, x, *)/\sim$ with the quotient topology.
Consider a contractible neighbourhood $U_y\ni y$,
and let $U_y\stackrel F \arrow \Omega(U_y, y, *)$ be a mapping that
was constructed in the exercise~\ref{_puti_iz_styagi_Zadacha_}. 
Let $\gamma\in \Omega(M, x, y)$ be a path and $U_y \stackrel\Psi\arrow
\Omega(M, x, *)$ be a mapping that maps $a\in U_y$ to a path
$\gamma F(a)$ (that is, to a path that is defined on
 $[0, 1/2]$ as $t \arrow \gamma(2t)$, and on
 $[1/2,1]$ as $F(a, 2t-1)$.
Prove that (for sufficiently small$U_y$) $\Psi$ composed with
$\Omega(M, x, *)\stackrel\pi\arrow\Omega(M, x, *)/\sim$ is a
homeomorphism  $U_y$ on some open 
subset in $\Omega(M, x, *)/\sim$.
\end{zadacha}

\begin{ukazanie}
  Continuity of $\Psi\circ\pi$ is obvious by construction and
  injectivity follows from the previous exercise. In order to show
  that $\Psi\circ\pi$ defines a homeomorphism $U_y$ on
  $\Psi\circ\pi(U_y)$ we need to show that prove $\Psi\circ\pi$ maps
  open sets to open sets. This is clear from the fact that the natural
  map $\Omega(M, x, *)/\sim\arrow M$, $\gamma'\arrow \gamma'(1)$ is
  continuous and defines a homeomorphism $U_y$ on its image.
\end{ukazanie}

\begin{zadacha}[*]
Consider the mapping $\Omega(M, x, *)/\sim\arrow M$ that maps a path $\gamma\in \Omega(M, x, y)$
to the point  $y=\gamma(1)$. Prove that this is a covering.
\end{zadacha}

\begin{ukazanie}
Use the previous exercise.
\end{ukazanie}

\begin{zadacha}[!]
Prove that $\Omega(M, x, *)$ is contractible.
\end{zadacha}

\begin{zadacha}[*]
Prove that $\gamma$ is a path in $\Omega(M, x, *)/\sim$.
Prove that $\gamma$ is homotopic to an image of some path from 
$\Omega(M, x, *)$.
\end{zadacha}

\begin{ukazanie}
  Prove that $\gamma$ can be lifted to a path in $\Omega(M, x, *)$
  locally and use the fact that for any point in
  $\Omega(M, x, *)/\sim$ its preimage in $\Omega(M, x, *)$ is
  connected.
\end{ukazanie}

\begin{zadacha}[*]
Deduce that $\Omega(M, x, *)/\sim$ is simply connected.
\end{zadacha}

\begin{zamechanie}
Let $(M, x)$ be a locally connected topological space
with a marked point. The universal covering of $M$ can be thus
identified  with the set of pairs ($y\in M$, homotopy class
of a path  $\gamma\in \Omega(M,x,y)$).
\end{zamechanie}

\newpage

%%%%%%%%%%%%%%%%%%%%%%%%%%%%%%%%%%%%%%%%%%%%%%%%
\subs{Free group and wedge sum}
%%%%%%%%%%%%%%%%%%%%%%%%%%%%%%%%%%%%%%%%%%%%%%%%

\begin{opredelenie}
Let $(M_1, x_1)$, $(M_2, x_2)$, $(M_3, x_3)$, $\ldots$ be a collection
(possibly infinite) of connected topological
spaces with  a marked point. Consider the quotient space of a
disconnected sum of all
$(M_\alpha, x_\alpha)$ by the equivalence relation
$\{x_1\}\sim \{x_2\}\sim \{x_3\}\sim \dots$
This quotient space is called a 
{\bf wedge sum}, denoted by
$\bigvee_\alpha (M_\alpha,x_\alpha)$.
A wedge sum can also be denoted by
$(M_1, x_1)\vee(M_2, x_2)\vee(M_3, x_3)\vee \dots$
\end{opredelenie}

\begin{zadacha} 
Assume that all $M_\alpha$ are connected (arcwise connected,
Hausdorff). Prove that the wedge sum is connected (arcwise connected,
Hausdorff). 
\end{zadacha}

\begin{zadacha}[!]
Assume that all $M_\alpha$ are connected and simply connected. Prove
that their wedge sum is simply connected.
\end{zadacha}

\begin{zadacha}[!]
  Let $\Gamma$ be a connect graph that has $n$ vertices and $n+k-1$
  edges. Prove that its associated topological space $M_\Gamma$ is
  homotopy equivalent to a wedge sum of $k$ circles.
\end{zadacha}

\begin{ukazanie} 
  Assume $\Gamma$ has an edge $r$ that connects two distinct vertices
  $v_1, v_2$. Consdireth graph $\Gamma'$ with $n-1$ vertices and
  $n+k-2$ edges that is obtained from $\Gamma$ in the following
  way. Remove an edge $r$ from $\Gamma$ and glue vertices $v_1$ and
  $v_2$ together. Prove that $M_\Gamma$ and $M_{\Gamma'}$ are homotopy
  equivalent.
\end{ukazanie}

\begin{opredelenie}
Consider a set $\{a_1, a_2, \dots\}$ of cardinality 
$N$ ($N$ by either finite or infinite). An {\bf $N$-ary tree} $D_N$ is
an infinite graph that is defined in the following way.
The verticesof $D_N$ are finite sequences of symbols $a_i$. The edges
connect vertices that correspond to  $A_1A_2 \dots A_k$
and $A_1A_2 \dots A_kA_{k+1}$ (all $A_i$ belong to 
$\{a_1, a_2, \dots\}$).
\end{opredelenie}

\begin{zadacha} Prove that every vertex 
$D_N$ has $N+1$ incoming edges.
\end{zadacha}

\begin{zadacha}[!]
Let $M_N$ be a topological space of an
$N$-ary tree, with the natural metric, defined in the beginning of
this exercise sheet.  Prove that $M_N$ is star-shaped
(any two points can be connected by a unique geodesic).
Prove that it is contractible.
\end{zadacha}

\begin{zadacha}[!]
Consider an  $2N-1$-ary tree.
Colour its edges in $N$ colours in such a way that any vertex has  2
incoming edges of each colour.
Consider the wedge sum of  $N$ circles and colour
each of the circles in a diffferent colour. Consider the mapping from
$M_{2N-1}$ to the wedge sum of $N$ circles that maps the vertices of
the graph to the vertices of the wedge sum and an edge of colour $a_i$
to the circle of the same colour. Prove that this is a universal cover.
\end{zadacha}

\begin{zadacha} 
  Let $\{a_1, a_2, \dots\}$ be a set of cardinality $N$, and let
  ${\mathcal W}$ be the set of finite sequences (``words'') of symbols
  $a_i$, $a_j^{-1}$, such that subsequences of the form $a_ia_i^{-1}$
  and $a_i^{-1}a_i$ never occur. A sequence of length 0 is denoted
  $e$. We multiply words by juxtaposing them and striking out all
  $a_ia_i^{-1}$, $a_i^{-1}a_i$ that might occur.  Prove that
  ${\mathcal W}$ forms a group.
\end{zadacha}

\begin{opredelenie}
This group is called the {\bf free group generated by 
 $\{a_1, a_2, \dots\}$} and is denoted $F_N$.
\end{opredelenie}

\begin{zadacha}
Prove that $F_1$ is isomorphic to $\Z$.
\end{zadacha}

\begin{zadacha}[!]
  Let $G$ be a group and $\{g_1, g_2, \dots \}$ be a collection of
  elements from $G$, labelled $\{a_1, a_2, \dots\}$.  Prove that there
  exists a unique homomorphism $F_N\arrow G$ that maps $a_i$ to $g_i$.
\end{zadacha}

\begin{zadacha}[!] Find a free action  of $F_N$ 
on the topological space $M_{2N-1}$ of an $2N-1$-ary tree that is
transitive on vertices.
\end{zadacha}

\begin{zadacha}[!] Prove that
$M_{2N-1}/F_N$ is a wedge sum of $N$ circles and that the fundamental
group of the wedge sum is free.
\end{zadacha}

\begin{zadacha}[!]
Prove that any (possibly infinite) graph is homotopy equivalent to a
wedge sum of circles.
\end{zadacha}

\begin{zadacha}[!]
Deduce that any subgroup of a free group is free.
\end{zadacha}

\begin{ukazanie}
Use the Galois theory of coverings.
\end{ukazanie}

\begin{zadacha}[*] 
  Let $G_1,G_2,\dots$ be a set of groups. Consider the set
  ${\mathcal W}$ of finite sequences of non-identity elements from
  different $G_i$ such that elements of the same group never occur
  next to each other. Given any sequence $A$
  of elements from $G_i$ one can obtain an element ${\mathcal W}$
  the following way. If $A$ has to successive elements from $G_i$,
  we multiply them and replace these elements with their product. If 
   $A$ ����������� an identity element of one of the groups we strike
   it out. Repeat this procedure as many times as needed in order to
   get an element from ${\mathcal W}$. The elements of ${\mathcal W}$
  can be multiplied by juxtaposing words and applying the procedure above.
  Prove that this defines a group.
\end{zadacha}

\begin{opredelenie}
  This group is called the {\bf free product of groups $G_1$, $G_2$,
    $\dots$}.
\end{opredelenie}

\begin{zadacha} Prove that the free group on 
 $N$ generators is the free product of  $N$ copies of $\Z$.
\end{zadacha}

\begin{zadacha} Prove that a free product of free groups is free. 
\end{zadacha}

\begin{zadacha}[*]
Let $(M_1, x_1), (M_2, x_2), (M_3, x_3), \dots$ be a collection of
connected topological spaces with a marked point. Prove that $\pi_1(\bigvee_\alpha(M_\alpha,x_\alpha))$
is isomorphic to a free product of groups $\pi_1(M_1, x_1),
\pi_1(M_2, x_2), \pi_1(M_3, x_3), \dots$.
\end{zadacha}

\end{document}


%%% Local Variables: 
%%% mode: latex
%%% coding: koi8-r
%%% ispell-local-dictionary: "british"
%%% TeX-master: t
%%% End: 
